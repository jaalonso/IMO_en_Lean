\chapter{Prueba de teoremas con Lean}\label{cap1}

En este primer capítulo del trabajo se hace una introducción al uso de
\href{https://leanprover-community.github.io/}{Lean}\footnote{\url{https://leanprover-community.github.io/}} y
\href{https://leanprover-community.github.io/mathlib_docs/}{mathlib}\footnote{\url{https://leanprover-community.github.io/mathlib_docs/}}
de manera que sea posible la correcta comprensión del resto del
trabajo. Estas herramientas han sido usadas con el objetivo de probar
diversos resultados matemáticos; es por eso que, previo al estudio
detallado de la estructura de una prueba y cómo realizarla, comenzaremos
introduciendo los términos de Lean y mathlib.

\section{Pilares fundamentales: Lean, IMO y mathlib}

Lean es un demostrador interactivo mediante el cual se pueden formalizar
multitud de resultados. Existen otros demostradores interactivos muy
conocidos como es
\href{https://www.cl.cam.ac.uk/research/hvg/Isabelle/index.html}{Isabelle/HOL}\footnote{\url{https://www.cl.cam.ac.uk/research/hvg/Isabelle}}
que ha sido usado en alguna asignatura
del Máster en Lógica, Computación e Inteligencia Artificial de la
Universidad de Sevilla.

El demostrador interactivo Lean es relativamente joven pues fue
desarrollado en 2013 por Leonardo de Moura en Microsoft
Research. Asimismo, existen diversas versiones de Lean, en concreto en
el trabajo ha sido utilizada la versión 3.28.00.

En este trabajo se ha usado Lean con el objetivo de formalizar problemas
correspondientes a las Olimpiadas Internacionales de Matemáticas (IMO).

Las IMO consisten en una competición de matemáticas que se realiza
anualmente para estudiantes preuniversitarios. La competición consta de
dos cuestionarios con tres problemas cada uno, formando un total de
seis. Los problemas que se proponen suelen ser de diversas áreas de las
matemáticas: álgebra, análisis, geometría, teoría de números, \dots.

Para la formalización de los problemas que se han estudiado en el
trabajo, se ha hecho uso de la librería matemática de Lean conocida como
mathlib.

La librería mathlib está formada por una gran cantidad de resultados (lemas,
teoremas, ...) y definiciones matemáticas ya formalizadas de todas las
diferentes áreas de esta ciencia.

El trabajo que se presenta ha sido desarrollado con la siguiente versión de
mathlib: 5511275718b4aa55fc66f969cbdf2065160b00bb. No obstante, durante la
realización de este trabajo se han producido multitud de problemas con las
versiones de mathlib. Estos ptoblemas son bastante comunes cuando se trabaja
con sistemas que se encuentran en continuo desarrollo, como es el caso de la
librería mathlib.

\section{Formalización de las demostraciones con Lean}

En esta sección del trabajo se va a explicar de manera detallada la
estructura de las pruebas que en el resto del trabajo se plantean y el
mecanismo mediante el cual se han llevado a cabo. Con el objetivo de que
sea más fácil e intuitivo de seguir la explicación, se va a realizar con
el desarrollo de un ejemplo muy visual que propuso Scott Morrison en
el vídeo \cite{video} en el Congreso
\href{https://leanprover-community.github.io/lftcm2020/}
     {Lean for the Curious Mathematician 2020}\footnote{\url{https://leanprover-community.github.io/lftcm2020/}}

Dividiremos en tres partes principales el procedimiento mediante el
cual se llegará a la formalización final. La primera parte se
corresponderá con el planteamiento inicial del problema. La segunda
con el desarrollo principal de la prueba y la tercera parte con la
finalización o conclusión de la misma.

Asimismo, destacar que tanto en el desarrollo de esta sección como en el
desarrollo de todo el trabajo se ha hecho uso de multitud de tácticas, las
cuales han sido detalladas en el Apéndice \ref{apentacti}.

\subsection{Planteamiento del problema}

En primer lugar, se va a enunciar en lenguaje natural el problema que se
desea formalizar en Lean. En nuestro caso es el siguiente:

\begin{teorema}[infinito\_numeros\_primos]\label{infi}
  Existen infinitos números primos; es decir que para todo número
  natural \(n\), existe un número primo \(p\) mayor o igual que
  \(n\).
\end{teorema}

Para llevar a cabo la formalización en Lean de este resultado, es
indispensable importar la librería que se necesitan. En estas librerías
hay diferentes conceptos, definiciones o lemas auxiliares que ya son
conocidos para Lean y que resultan útiles para la prueba.

En el caso del problema que se va a planear, inicialmente sólo es necesario
una teoría y es
\href{https://github.com/leanprover-community/mathlib/blob/master/src/data/nat/prime.lean}
{data.nat.prime}. En Lean la orden de importación de esta librería se plantea
como sigue:
\begin{leancode}
import data.nat.prime
\end{leancode}

\begin{nota}
  Destacar que en el proceso de realización de una prueba no siempre se sabe
  desde un principio cuáles son las librerías necesarias que habrá que importar.
  Es por eso que esto se puede hacer en cualquiero momento de la formalización,
  aunque es necesario escribirlo al comienzo del fichero.
\end{nota}

A continuación, se va a habilitar el
\href{https://leanprover.github.io/reference/other_commands.html#namespaces}
{espacio de nombre} de los números naturales en este caso. Esto se hace
con el objetivo de simplificar la notación; por ejemplo, escribiendo
\texttt{prime} en lugar de \texttt{nat.prime}. Para hacer esta
simplificación en Lean basta con:
\begin{leancode}
open nat
\end{leancode}

Ahora sí, ya se tienen las condiciones necesarias para poder enunciar el
teorema \ref{infi} y plantear la estructura de la demostración en Lean.
Veámoslo:
\begin{leancode}
theorem infinitos_numeros_primos : ∀ n, ∃ p ≥ n, prime p:=
begin
  sorry,
end
\end{leancode}

En la formalización que se ha planteado se puede observar que hemos usado la
estructura de \texttt{theorem} donde hemos llamado de la misma forma al
teorema que en \ref{infi}. También se puede observar que el enunciado del
teorema como tal comienza a partir de los dos puntos y que es totalmente
análogo al visto anteriormente. En el enunciado del teorema se ha hecho uso
de la función \texttt{prime} que se encuentra en la librería que se ha
importado.

\begin{definicion}
  Un número natural \(p\) es primo si \(p\) es mayor o igual que dos y,
  además, para todo número natural \(m\) que divide a \(p\) se tiene
  que o bien \(m\) es uno, o bien \(m\) es igual a \(p\).
  Simbólicamente:
  \begin{equation*}
    \text{primo }(p) := p ≥ 2 ∧ ∀ m (m | p → m = 1 ∨ m = p).
  \end{equation*}
\end{definicion}

La formalización en Lean de esta definición es la siguiente:
\begin{leancode}
def prime (p : ℕ) := 2 ≤ p ∧ ∀ m ∣ p, m = 1 ∨ m = p
\end{leancode}

Por otro lado, se tiene que para el planteamiento de todas las
formalizaciones que se van a presentar en este trabajo se han usado la
estructura de demostración \texttt{begin}--\texttt{end} como se puede
observar.

Finalmente, para acabar con el planteamiento inicial del problema,
mencionar el uso de la táctica \tactica{sorry}{sorry}. Esta táctica es
una herramienta mágica que acepta cualquier resultado. Más adelante
iremos viendo la utilidad de esta táctica en el proceso de formalización
de cualquier resultado.  Asimismo, destacar que cuando en algún teorema
o resultado se usa la táctica \tactica{sorry}{sorry}, Lean nos da un
aviso de ello.

\subsection{Desarrollo de la prueba}

Una vez ya se ha hecho el planteamiento del enunciado del teorema,
importado las librerías necesarias y definido las variables de trabajo
(en nuestro caso trabajamos en el conjunto de los números naturales), se
puede proceder al comienzo de la demostración como tal.

El uso de la táctica \tactica{sorry}{sorry} que hemos mencionado
anteriormente, no es que nos pruebe los resultados deseados. El
verdadero uso es que durante el proceso de formalización, se va a ir
diviendo el problema inicial en subproblemas que formalizaremos al final
y de esta forma Lean no nos devuelve error.

Entonces, se tiene que obviando el uso de la táctica
\tactica{sorry}{sorry}, el estado que tenemos actualmente de la
formalización en Lean es el siguiente:
\begin{leancode}
⊢ ∀ (n : ℕ), ∃ (p : ℕ) (H : p ≥ n), prime p
\end{leancode}

Ahora bien, como la formalización que queremos plantear comienza con un
para todo número natural \(n\), lo primero que hacemos es introducir un
número natural arbitrario. En Lean esto lo formalizamos como sigue:
\begin{leancode}
theorem infinitos_numeros_primos : ∀ n, ∃ p ≥ n, prime p:=
begin
  intro n,
  sorry,
end
\end{leancode}

Para la introducción del número natural \(n\) se ha hecho uso de la
táctica \tactica{intro}{intro} mediante la cual fijamos dicho número. Al
haber introducido el número natural \(n\), se tiene que el estado de la
prueba en Lean pasa a ser el siguiente:
\begin{leancode}
n: ℕ
⊢ ∃ (p : ℕ) (H : p ≥ n), prime p
\end{leancode}

A continuación, siguiendo con el desarrollo de la prueba, entramos en la
parte más matemática de la misma. Para ello, se tiene la \textit{idea
feliz} de considerar los siguientes dos números naturales:
\begin{align}
  &m = n!+1,\label{numm}\\
  &p = \text{mínimo factor primo de \(m\) que sea distinto de 1}.\label{nump}
\end{align}

En Lean esto se formalizaría como sigue:
\begin{leancode}
theorem infinitos_numeros_primos : ∀ n, ∃ p ≥ n, prime p:=
begin
  intro n,

  let m := factorial n + 1,
  let p := min_fac m,
  sorry,
end
\end{leancode}

Para introducir los dos números que hemos definido en (\ref{numm}) y
(\ref{nump}) se ha hecho uso de la táctica \tactica{let}{let} que
matemáticamente es equivalente a consideremos el número \(\_\) definido
de la siguiente manera.

Asimismo, a parte del uso de la táctica \tactica{let}{let}, se han
utilizado también dos funciones: la función \textbf{factorial} de un número y la
función que nos devuelve el \textbf{mínimo factor primo distinto de uno} de un
número. Estas dos funciones están formalizadas en Lean de la siguiente
manera:
\begin{leancode}
def factorial : nat → nat
  | 0        := 1
  | (succ n) := succ n * factorial n

def min_fac : ℕ → ℕ
  | 0 := 2
  | 1 := 1
  | (n+2) := if 2 ∣ n then 2 else min_fac_aux (n + 2) 3
\end{leancode}

Destacar que para el uso de estas funciones en Lean me ha sido necesario
la importación de una nueva teoría, en este caso correspondiente a los
factoriales,
\href{https://github.com/leanprover-community/mathlib/blob/master/src/data/nat/factorial.lean}{data.nat.factorial}.

Además, se puede observar que al definir la función mínimo factor primo
de un número natural se ha hecho uso de una función auxiliar llamada
\textbf{min\_fac\_aux} y que se define de la siguiente manera:

\begin{definicion}
  Sea \(n\) y \(k\) números naturales. Definimos la función
  \textbf{min\_fac\_aux} con la siguiente casuística:
  \begin{itemize}
  \item  Si se verifica que \(n\) es menor que el producto de \(k\) por sí
    mismo; entonces la función \textbf{min\_fac\_aux} devuelve el propio
    número \(n\).

  \item En el caso de que \(k\) divida a \(n\); entonces
    \textbf{min\_fac\_aux} devuelve el número \(k\).

  \item En cualquier otro caso, la función \textbf{min\_fac\_aux} aplicada
    sobre los números \(n\) y \(k\) es igual a aplicar esta misma función
    sobre los números \(n\) y \(k+2\).
  \end{itemize}
\end{definicion}

Tras todo lo detallado, se tiene que al haber introducido la definición de los
números \(m\) y \(p\) anteriormente descritos en Lean; el estado actual pasa a
ser el siguiente:
\begin{leancode}
n: ℕ
m: ℕ := n.factorial + 1
p: ℕ := m.min_fac
⊢ ∃ (p : ℕ) (H : p ≥ n), prime p
\end{leancode}

En el estado anterior se observa que, por defecto, Lean usa la notación
de puntos, pero se puede cambiar la opción añadiendo
\begin{leancode}
set_option pp.structure_projections false
\end{leancode}
con lo que el estado anterior lo escribe con la notación habitual
\begin{leancode}
n : ℕ,
m : ℕ := factorial n + 1,
p : ℕ := min_fac m
⊢ ∃ (p : ℕ) (H : p ≥ n), prime p
\end{leancode}

El siguiente paso que queremos realizar en la formalización es decirle a
Lean que el número \(p\) que estamos buscando es exactamente el que
hemos definido justo antes. Esto lo hacemos mediante la táctica
\tactica{use}{use}, la cual, como su propio nombre indica, le dice a
Lean que use el número que ordenemos como testigo de la fórmula
existencial. Veamos como sigue la formalización:

\begin{leancode}
theorem infinitos_numeros_primos : ∀ n, ∃ p ≥ n, prime p:=
begin
  intro n,

  let m := factorial n + 1,
  let p := min_fac m,

  use p,
  sorry,
end
\end{leancode}

Estamos dando un paso muy importante en el que el estado de la formalización
pasa a ser:
\begin{leancode}
n : ℕ
m : ℕ := factorial n + 1
p : ℕ := min_fac m
⊢ p ≥ n ∧ prime p
\end{leancode}

Se puede observar que el objetivo a demostrar se ha convertido en una
conjunción: por un lado hay que demostrar que \(p\) es mayor o igual que
\(n\) y, por otro lado, que el número \(p\) que hemos definido es primo. Ahora
nos interesa separar el problema a demostrar en dos subproblemas, esto es
muy sencillo con el uso de la táctica \tactica{split}{split}, la cual nos
divide la conjunción. De esta manera la formalización seguiría como:
\begin{leancode}
theorem infinitos_numeros_primos : ∀ n, ∃ p ≥ n, prime p:=
begin
  intro n,

  let m := factorial n + 1,
  let p := min_fac m,

  use p,
  split,
  { sorry, },
  { sorry, },
end
\end{leancode}

Se puede observar que en la formalización se ha pasado de tener un
\tactica{sorry}{sorry} a tener dos; esto se tiene como consecuencia de
haber dividido el problema en dos partes. Además, la manera más elegante
y cómoda de escribirlo es que como se hecho, diferenciando la prueba de
cada problema entre llaves. De esta manera, se tiene que, tras la
aplicación de la táctica \tactica{split}{split}, el estado pasa a ser:
\begin{leancode}
n : ℕ
m : ℕ := factorial n + 1
p : ℕ := min_fac m
⊢ p ≥ n

n : ℕ
m : ℕ := factorial n + 1
p : ℕ := min_fac m
⊢ prime p
\end{leancode}

De esta manera, se pueden observar los dos objetivos a demostrar que faltarían
para concluir la prueba. Ahora bien, para demostrar el primero de los objetivos,
es necesario usar el segundo resultado de los que hay que demostrar. Es por eso,
que lo que proponemos es demostrar antes que el número \(p\) es primo (que es
el resultado que necesitamos para demostrar el primer objetivo).

Para ello introduciremos una hipótesis de la siguiente manera:

\begin{leancode}
theorem infinitos_numeros_primos : ∀ n, ∃ p ≥ n, prime p:=
begin
  intro n,

  let m:= factorial n + 1,
  let p:= min_fac m,
  have hp: prime p := sorry,

  use p,
  split,
  {sorry,},
  {sorry,},
end
\end{leancode}

Denotemos la hipótesis de que \(p\) sea primo de la siguiente manera:
\begin{equation}\tag{hp}\label{hpintro}
  \text{primo}(p)
\end{equation}

Se puede observar que al haber introducido la hipótesis (\ref{hpintro}) tras
la definición del propio número \(p\), dicha hipótesis puede ser utilizada
a partir de ahí.

Para introducir la hipótesis (\ref{hpintro}) se ha hecho uso de la táctica
\tactica{have}{have}. En este resultado también hemos añadido la \tactica{sorry}
{sorry} porque la prueba se realizará luego. En el estado del problema el único
cambio que ha habido respecto al último presentado es que se ha incluido la
hipótesis (\ref{hpintro}) en los dos subproblemas:
\begin{leancode}
n : ℕ
m : ℕ := factorial n + 1
p : ℕ := min_fac m
hp : prime p
⊢ p ≥ n

n : ℕ
m : ℕ := factorial n + 1
p : ℕ := min_fac m
hp : prime p
⊢ prime p
\end{leancode}

Recordemos que el número \(p\) se ha definido como el mínimo factor primo de
\(m\) que sea distinto de uno. Por tanto, se tiene que la prueba de que \(p\)
es primo será directa y la dejaremos para el final.

Nos centraremos en la prueba de que el número \(p\) es mayor o igual que
\(n\). En general, muchas de las pruebas de desigualdades en
matemáticas se hacen por reducción al absurdo. Es decir, supongo que no
es cierto dicha desigualdad y llego a una contradicción en algún
momento. Procederemos así para la prueba que a nosotros nos interesa; en
Lean esto se formularía con la táctica
\tactica{by_contra / by_contradiction}{by\_contradiction}:
\begin{leancode}
theorem infinitos_numeros_primos : ∀ n, ∃ p ≥ n, prime p:=
begin
  intro n,

  let m := factorial n + 1,
  let p := min_fac m,
  have hp : prime p := sorry,

  use p,
  split,
  { by_contradiction,
    sorry, },
  { sorry, },
end
\end{leancode}

De esta manera también se actualiza el estado y pasamos a tener una
nueva hipótesis y tenemos que demostrar falso, así es como Lean nos
indica que tenemos que llegar a una contradicción. Esto es:
\begin{leancode}
n : ℕ
m : ℕ := factorial n + 1
p : ℕ := min_fac n
hp : prime p
h : ¬p ≥ n
⊢ false
\end{leancode}

Para poder concluir la prueba de este primer problema, se van a introducir
las tres siguientes hechos:
\begin{align}
  &p | (n!+1),\tag{h1}\label{h1intro}\\
  &p | n!,    \tag{h2}\label{h2intro}\\
  &p | 1.     \tag{h3}\label{h3intro}
\end{align}

Introduzcámolas ahora en Lean:
\begin{leancode}
theorem infinitos_numeros_primos : ∀ n, ∃ p ≥ n, prime p:=
begin
  intro n,

  let m := factorial n + 1,
  let p := min_fac m,
  have hp : prime p := sorry,

  use p,
  split,
  { by_contradiction,
    have h1 : p ∣ factorial n + 1 := sorry,
    have h2 : p ∣ factorial n := sorry,
    have h3 : p ∣ 1 := sorry,
    sorry, },
  { sorry, },
end
\end{leancode}

En el caso del estado de este subproblema simplemente se han introducido las
hipótesis (\ref{h1intro}), (\ref{h2intro}) y (\ref{h3intro}), el objetivo a
probar no cambia:
\begin{leancode}
n : ℕ
m : ℕ := factorial n + 1
p : ℕ := min_fac m
hp : prime p
h : ¬p ≥ n
h1 : p ∣ factorial n + 1
h2 : p ∣ factorial n
h3 : p ∣ 1
⊢ false
\end{leancode}

A través de las hipótesis añadidas demostrar que se llega a una
contradicción no es nada complicado.

Entonces, se ha llegado a lo que es la estructura final de la prueba. Se
puede observar que tenemos varios resultados que tenemos pendientes con
la táctica \tactica{sorry}{sorry}, pero los resultados que tenemos son
muy sencillos de probar y la siguiente parte de la sección veremos cómo
hacerlo con una herramienta muy útil de Lean.

\subsection{Finalización de la prueba}

Una vez ya se ha llegado a la estructura final de la prueba con el uso
en una o varias ocasiones de la táctica \tactica{sorry}{sorry} es muy
sencillo llevar a cabo la finalización de la demostración, siempre y
cuando estemos seguros de que los resultados que nos falten por
demostrar sean ciertos.

Cuando nos encontramos ante resultados casi inmediatos de probar, Lean
posee una táctica muy útil que nos indica qué lema o teorema hay que
usar. Esta táctica es \tactica{library_search}{library\_search}. Es por
eso que lo primero que intentamos es utilizar dicha táctica en todos los
resultados que nos faltan por probar. De esta forma, de los seis
resultados que nos faltaban por probar cuatro de ellos se prueban con
esta táctica:
\begin{leancode}
theorem infinitos_numeros_primos : ∀ n, ∃ p ≥ n, prime p:=
begin
  intro n,

  let m := factorial n + 1,
  let p := min_fac m,
  have hp : prime p := sorry,

  use p,
  split,
  { by_contradiction,
    have h1 : p ∣ factorial n + 1 := by library_search,
    have h2 : p ∣ factorial n := by sorry,
    have h3 : p ∣ 1 := by library_search,
    library_search, },
  { library_search, },
end
\end{leancode}

Para ir simplificando los problemas que nos quedan, a continuación lo
que se puede hacer es escribir la demostración que la táctica
\tactica{library_search} {library\_search} ha propuesto en los problemas
que sí funciona. De esta forma, la formalización en Lean nos queda:
\begin{leancode}
theorem infinitos_numeros_primos : ∀ n, ∃ p ≥ n, prime p:=
begin
  intro n,

  let m := factorial n + 1,
  let p := min_fac m,
  have hp : prime p := sorry,

  use p,
  split,
  { by_contradiction,
    have h1 : p ∣ factorial n + 1 := min_fac_dvd m,
    have h2 : p ∣ factorial n := by sorry,
    have h3 : p ∣ 1 := (nat.dvd_add_right h2).mp h1,
    exact prime.not_dvd_one hp h3,},
  { exact hp, },
end
\end{leancode}

Antes de proseguir con la demostración de los dos resultados que nos faltan
por demostrar (\(p\) primo y (\ref{h2intro})), veamos las pruebas que nos han
demostrado los cuatro resultados anteriores.

En primer lugar, para el caso de la hipótesis (\ref{h1intro}) se ha hecho uso
del siguiente teorema:
\begin{teorema}
  Sea \(n\) cualquier número natural, entonces se tiene que el mínimo factor
  primo de \(n\) divide al propio \(n\).
\end{teorema}
Y su formalización en Lean es la siguiente:
\begin{leancode}
theorem min_fac_dvd (n : ℕ) : min_fac n ∣ n
\end{leancode}

En particular, como el número \(m\) está definido como el factorial de la suma
de \(n\) y \(1\) y \(p\) es el mínimo factor primo de \(m\), se tiene de manera
inmediata que \(p\) divide al factorial de \(n+1\).

En segundo lugar, demostrar que (\ref{h3intro}) es bastante inmediato a partir
de las hipótesis (\ref{h1intro}) y (\ref{h2intro}). Por la hipótesis
(\ref{h1intro}) se tiene que \(p\) divide a la suma del factorial de \(n\) y
\(1\); mientras que por la hipótesis (\ref{h2intro}) se sabe que \(p\) divide
al factorial de \(n\). Entonces aplicando el teorema que se plantea a
continuación se tiene de manera inmediata.
\begin{teorema}
  Sean \(k\), \(m\) y \(n\) números naturales tales que \(k\) divide a
  \(m\). Entonces se tiene que \(k\) divide a la suma de \(m\) y \(n\)
  si y solamente si \(k\) divide a \(n\).  Simbólicamente:
  \begin{equation*}
    ∀(k,m,n∈ℕ), k|m ⟶ (k|m+n⟺k|n).
  \end{equation*}
\end{teorema}

Cuya formalización en Lean es la siguiente:
\begin{leancode}
theorem dvd_add_right {k m n : ℕ} (h : k ∣ m) : k ∣ m + n ↔ k ∣ n
\end{leancode}

El tercer resultado que ha demostrado la táctica
\tactica{library_search} {library\_search} es la contradicción a la que
había llegar en el primer problema de la división. Una vez se tiene la
hipótesis (\ref{h3intro}) que nos dice que el número \(p\) divide a
\(1\) y sabiendo que \(p\) es un número primo que se tiene por
hipótesis; la contradicción es inmediate por la propia definición de ser
primo. Para llegar a dicha contradicción se hace uso de que un número
primo no puede dividir al número \(1\), formalmente este teorema se
enunciaría como sigue:
\begin{teorema}
  Sea \(p\) un número natural verificando que es un número primo. Entonces
  se tiene que \(p\) no divide a \(1\).
\end{teorema}
Cuya correspondiente formalización en Lean sería:
\begin{leancode}
theorem prime.not_dvd_one {p : ℕ} (pp : prime p) : ¬ p ∣ 1
\end{leancode}

Por último, el cuarto resultado es demostrar que \(p\) es primo lo cual
se tiene de manera inmediata a través de la hipótesis (\ref{hpintro}) que nos
falta por demostrar.  Para ello, Lean nos propone usar la táctica
\tactica{exact}{exact}.

A continuación, se van a demostrar las dos hipótesis que nos faltan para
poder concluir la demostración; es decir, (\ref{h2intro}) y (\ref{hpintro}).
A priori, estos dos resultados no son tan triviales como los anteriores pero
veremos cómo resolverlos. Una forma muy útil y cómoda de trabajar cuando la
táctica \tactica{library_search}{library\_search} no ha funcionado es escribirlo
como un lema independiente. Separemos las dos pruebas que queremos realizar en
dos lemas diferentes:
\begin{itemize}
\item \textbf{Demostración y formalización del hecho h2}

  El hecho (\ref{h2intro}) bajo las hipótesis que tenemos puede ser
  enunciado de la siguiente manera:
  \begin{lema}[h2]
    Sean \(m\), \(n\) y \(p\) tres números naturales tales que \(m\) es
    igual a la suma del factorial de \(n\) con \(1\) y \(p\) es igual al
    mínimo factor primo de \(m\). Además, estos números verifican las
    siguientes tres hipótesis:
    \begin{align}
      &\text{primo}(p),\label{hpintro}\tag{hp}\\
      &¬p ≥ n,         \tag{h}\label{hintro}\\
      &p | (n!+1).     \tag{h1}\label{h11intro}
    \end{align}
    Entonces se tiene que \(p\) divide al factorial de \(n\).
  \end{lema}

  Para formalizar la demostración de la hipótesis (\ref{h2intro}) hemos creado
  un lema auxiliar con las hipótesis correspondiente. Esta formalización es:
  \begin{leancode}
lemma h2
  (n : ℕ)
  (m = factorial n + 1)
  (p = min_fac m)
  (hp : prime p)
  (h : ¬p ≥ n)
  (h1 : p ∣ factorial n + 1)
  : p ∣ factorial n :=
  begin
    sorry,
end
  \end{leancode}

  Se puede observar que hemos vuelto a usar la estructura
  \texttt{begin}- \texttt{end} que usamos para el teorema
  inicialmente. Esto se debe a que cuando nos encontramos ante
  resultados que de primeras no somos capaces de demostrar en una línea
  es más sencillo comenzar la formalización de esta manera.

  Ya hemos comprobado que la táctica
  \tactica{library_search}{library\_search} no es capaz de solucionarnos
  este lema; sin embargo, existe otra táctica que nos da sugerencias
  sobre las posibles soluciones. Esta táctica es
  \tactica{suggest}{suggest}. En general, cuando se usa esta táctica,
  Lean propone multitud de soluciones. Es por eso que debemos
  valorarlas y probar a elegir una. En el caso que a nosotros nos
  concierne la primera propuesta nos sirve de gran utilidad. Veámosla:
\begin{leancode}
lemma h2
  (n : ℕ)
  (m = factorial n + 1)
  (p = min_fac m)
  (hp : prime p)
  (h : ¬p ≥ n)
  (h1 : p ∣ factorial n + 1)
  : p ∣ factorial n:=
begin
  refine (prime.dvd_factorial hp).mpr _,
  sorry,
end
\end{leancode}

  La táctica \tactica{suggest}{suggest} nos ha propuesto el uso del siguiente
  lema:
  \begin{lema}
    Sean \(n\) y \(p\) dos números naturales tales que \(p\) es
    primo. Entonces se tiene que \(p\) divide al factorial de \(n\) si y
    solamante si \(p\) es menor o igual que \(n\)
  \end{lema}

  Cuya formalización en Lean es:
  \begin{leancode}
  lemma prime.dvd_factorial : ∀ {n p : ℕ} (hp : prime p), p ∣ n! ↔ p ≤ n
  \end{leancode}

  Además, en lugar de usar la táctica \tactica{exact}{exact}, se ha hecho uso
  de la táctica \tactica{refine}{refine} que es equivalente a la anterior pero
  se diferencian en que la última podemos dejar que Lean interprete sobre
  quién se aplica.

  Tras el uso de este lema, se tiene que el estado del problema se convierte en
  el siguiente:
  \begin{leancode}
  n m : ℕ,
  H : m = factorial n + 1,
  p : ℕ,
  H : p = min_fac m,
  hp : prime p,
  h : ¬p ≥ n,
  h1 : p ∣ factorial n + 1
  ⊢ p ≤ n
  \end{leancode}

  Una vez se ha conseguido a dar un paso en la demostración, se propone
  de nuevo el uso de la táctica
  \tactica{library_search}{library\_search} y es capaz de concluir la
  prueba de esta hipótesis:
  \begin{leancode}
lemma h2
  (n : ℕ)
  (m = factorial n + 1)
  (p = min_fac m)
  (hp : prime p)
  (h : ¬p ≥ n)
  (h1 : p ∣ factorial n + 1)
  : p ∣ factorial n:=
begin
  refine (prime.dvd_factorial hp).mpr _,
  library_search,
end
  \end{leancode}

  En este caso, para culminar la prueba de la hipótesis (\ref{h2intro}) ha
  hecho la siguiente propuesta:
  \begin{leancode}
lemma h2
  (n : ℕ)
  (m = factorial n + 1)
  (p = min_fac m)
  (hp : prime p)
  (h : ¬p ≥ n)
  (h1 : p ∣ factorial n + 1)
  : p ∣ factorial n:=
begin
  refine (prime.dvd_factorial hp).mpr _,
  exact le_of_not_ge h,
end
  \end{leancode}

  Simplemente ha hecho uso del lema que se presenta a continuación:
  \begin{lema}
    Sean \(a\) y \(b\) dos números cualesquiera tales que \(a\) no es
    mayor o igual que \(b\). Entonces se tiene que \(a\) es menor o
    igual que \(b\).
  \end{lema}

  Cuya formalización en Lean es:
  \begin{leancode}
  lemma le_of_not_ge {a b : α} : ¬ a ≥ b → a ≤ b
  \end{leancode}

\item \textbf{Demostración y formalización del hecho hp}

  A continuación, se demostrará el hecho (\ref{hpintro}) bajo las hipótesis
  del problema. Este nuevo lema podría ser enunciado de la siguiente manera:
  \begin{lema}
    Sean \(n\), \(m\) y \(p\) números naturales tales que verifican las dos
    siguiente condiciones:
    \begin{align}
      &m=n!+1\tag{h1p}\label{h1pintro}\\
      &p=\text{mínimo factor primo de }m\text{ que sea distinto de }1.\tag{h2p}
        \label{h2pintro}
    \end{align}
    Entonces se tiene que el número \(p\) así definido es primo.
  \end{lema}

  Y su correspondiente formalización en Lean sería:
  \begin{leancode}
lemma hp
  (n m p : ℕ)
  (h1 : m = factorial n + 1)
  (h2 : p = min_fac m)
  : prime p :=
begin
  sorry,
end
  \end{leancode}

  A continuación, procedamos a la demostración de ese resultado:

  \begin{demostracion}

    En primer lugar, para que realmente el lema que hemos enunciado se
    corresponda con el que se tiene en el teorema principal hay que reescribir
    \(p\) como nos dice la hipótesis (\ref{h2pintro}). En Lean esto se haría
    de la siguiente forma:
    \begin{leancode}
lemma hp
  (n m p : ℕ)
  (h1 : m = factorial n + 1)
  (h2 : p = min_fac m)
  : prime p :=
begin
  rw h2p,
  sorry,
end
    \end{leancode}
    En este caso para introducir la hipótesis (\ref{h2pintro}) en el objetivo a
    probar se ha hecho uso de la táctica \tactica{rw / rewrite}{rw}. De esta
    forma, el objetivo a demostrar se convierte en:
    \begin{leancode}
nmp: ℕ
h1p: m = factorial n + 1
h2p: p = min_fac m
⊢ prime (min_fac m)
    \end{leancode}

    Al intentar aplicar la táctica \tactica{library_search}{library\_search},
    seguimos sin tener una propuesta de solución. Es por eso que intentamos con
    la táctica \tactica{suggest}{suggest} y se obtiene varias prupuestas, de la
    que se ha elegido la siguiente:
    \begin{leancode}
lemma hp
  (n m p : ℕ)
  (h1p : m = factorial n + 1)
  (h2p : p = min_fac m)
  : prime p :=
begin
  rw h2p,
  refine min_fac_prime _,
  sorry,
end
    \end{leancode}

    Lo que la táctica \tactica{suggest}{suggest} nos ha propuesto es hacer uso
    del teorema que se va a presentar a continuación junto con la táctica
    \tactica{refine}{refine}.

    \begin{teorema}
      Sea \(n\) un número natural distinto de \(1\). Entonces se tiene que el
      mínimo factor primo de \(n\) es primo.
    \end{teorema}

    La formalización en Lean de este teorema sería:
    \begin{leancode}
theorem min_fac_prime {n : ℕ} (n1 : n ≠ 1) : prime (min_fac n)
    \end{leancode}

    Tras la aplicación de este teorema el objetivo a demostrar pasa a ser el
    siguiente:
    \begin{leancode}
nmp: ℕ
h1p: m = factorial n + 1
h2p: p = min_fac m
⊢ m ≠ 1
    \end{leancode}

    De esta manera, si observamos las hipótesis que tenemos, el objetivo a probar
    es bastante sencillo. En primer lugar, se va a introducir la hipótesis de
    que el factorial del número \(n\) es mayor que cero, esto es:
    \begin{equation}
      0 < n!
    \end{equation}

    Este resultado se tiene de manera inmediata con el teorema siguiente:
    \begin{teorema}
      Sea \(n\) un número cualquiera, entonces se tiene que el factorial de dicho
      número es mayor estrictamente que cero.
    \end{teorema}

    Y en Lean nos quedaría:
    \begin{leancode}
theorem factorial_pos : ∀ n, 0 < n!
    \end{leancode}

    La introducción en Lean de esta hipótesis se haría a través de la taćtica
    \tactica{have}{have} y se plantearía como sigue:
    \begin{leancode}
lemma hp
  (n m p : ℕ)
  (h1p : m = factorial n + 1)
  (h2p : p = min_fac m)
  : prime p :=
begin
  rw h2p,
  refine min_fac_prime _,
  have : 0 < factorial n := factorial_pos n,
  sorry,
end
    \end{leancode}

    De esta forma, el objetivo a demostrar no cambia, simplemente se añade la
    hipótesis definida al conjunto de las hipótesis.

    A continuación, si observamos las hipótesis que tenemos, se puede comprobar
    que haciendo uso de la desigualdad planteada y la definición del número
    \(m\) se puede concluir la formalización. Como se trata de un paso en el que
    hay que trabajar con desigualdades tiene sentido probar con la táctica
    \tactica{linarith}{linarith}. De esta manera, se concluiría la prueba:

    \begin{leancode}
lemma hp
  (n m p : ℕ)
  (h1p : m = factorial n + 1)
  (h2p : p = min_fac m)
  : prime p :=
begin
  rw h2p,
  refine min_fac_prime _,
  have : 0 < factorial n := factorial_pos n,
  linarith,
end
    \end{leancode}
    Ya se tendría demostrada entonces la hipótesis (\ref{hpintro}).
  \end{demostracion}
\end{itemize}

Al haber demostrado y formalizado las hipótesis (\ref{h2intro}) y
(\ref{hpintro}) ya se tendría completada la prueba de que existen infitos
números primos naturales. Bastaría con incluir las demostraciones realizadas
en la formalización principal:
\begin{leancode}
theorem infinitos_numeros_primos:
∀ n, ∃ p ≥ n, prime p:=
begin
  intro n,

  let m := factorial n + 1,
  let p := min_fac m,
  have hp : prime p ,
  { refine min_fac_prime _,
    have : 0 < factorial n := factorial_pos n,
    linarith,},

  use p,
  split,
  { by_contradiction,
    have h1 : p ∣ factorial n + 1 := min_fac_dvd m,
    have h2 : p ∣ factorial n,
    { refine (prime.dvd_factorial hp).mpr _,
      exact le_of_not_ge h,},
    have h3 : p ∣ 1 := (nat.dvd_add_right h2).mp h1,
    exact prime.not_dvd_one hp h3,},
  { exact hp, },
  end
\end{leancode}

Para finalizar de manera completa la formalización en Lean de este resultado
faltaría un último paso. Este paso consiste en tratar de reducir las
demostraciones de algunos hechos; en concreto, los hechos (\ref{hpintro}) y
(\ref{h2intro}) son los que podrían ser simplificados. Para ello, se hace
uso de la táctica \tactica{show_term}{show\_term} de la siguiente manera:
\begin{leancode}
theorem infinitos_numeros_primos:
  ∀ n, ∃ p ≥ n, prime p:=
begin
  intro n,

  let m := factorial n + 1,
  let p := min_fac m,
  have hp : prime p ,
  show_term{ refine min_fac_prime _,
    have : 0 < factorial n := factorial_pos n,
    linarith,},

  use p,
  split,
  { by_contradiction,
    have h1 : p ∣ factorial n + 1 := min_fac_dvd m,
    have h2 : p ∣ factorial n,
    show_term{ refine (prime.dvd_factorial hp).mpr _,
      exact le_of_not_ge h,},
    have h3 : p ∣ 1 := (nat.dvd_add_right h2).mp h1,
    exact prime.not_dvd_one hp h3,},
  { exact hp, },
end
\end{leancode}

En el caso del hecho (\ref{hpintro}) la propuesta que nos hace Lean no es óptima
y por eso mismo la desestimamos; mientras la que hace para el hecho
(\ref{h2intro}) sí lo es y la introducimos en nuestra formalización. Entonces,
se tiene que la formalización del teorema final sería:
\begin{leancode}
theorem infinitos_numeros_primos:
  ∀ n, ∃ p ≥ n, prime p:=
begin
  intro n,

  let m := factorial n + 1,
  let p := min_fac m,
  have hp : prime p ,
  { refine min_fac_prime _,
    have : 0 < factorial n := factorial_pos n,
    linarith,},

  use p,
  split,
  { by_contradiction,
    have h1 : p ∣ factorial n + 1 := min_fac_dvd m,
    have h2 : p ∣ factorial n,
    exact iff.mpr (prime.dvd_factorial hp) (le_of_not_ge h),
    have h3 : p ∣ 1 := (nat.dvd_add_right h2).mp h1,
    exact prime.not_dvd_one hp h3,},
  { exact hp, },
end
\end{leancode}
