\chapter{Prueba de teoremas con Lean}
En este primer capítulo del trabajo se hace una introducción al uso de Lean
y mathlib de manera que sea posible la correcta comprensión del resto del
trabajo. Estas herramientas han sido usadas con el objetivo de probar diversos
resultados matemáticos; es por eso que, previo al estudio detallado de la
estructura de una prueba y cómo realizarla, comnzaremos introduciendo los
términos de Lean y mathlib.

\section{Introducción a Lean y mathlib}


\section{Planteamiento de las demostraciones con Lean}
A continuación, se va a explicar de manera detallada el procedimiento mediante
el cual se han planteado las pruebas de todos los resultados que se plantean
en el trabajo. Con el objetivo de que sea más fácil de seguir la explicación,
se planterá con un ejemplo muy visual que ya propuso Scott Morrison en el
vídeo \ref{video}.

