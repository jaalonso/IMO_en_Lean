
\section{IMO 1977 Q6}

En esta sección se va a detallar la solución al problema Q6 que se
propuso en el año 1977. Comenzaremos presentando el enunciado de dicho
problema y su correspondiente solución en lenguaje
natural. Posteriormente se presentará la formalización en Lean de dicho
problema y el análisis de ésta.

Destacar que la formalización original del problema se ha obtenido de
\cite{TC} y fue propuesta por Tian Chen.

\noindent
\framebox{\parbox{\textwidth}{
    \textbf{Problema 1 (1977--Q6)}. Consideremos la función
    \(f:ℕ⁺ → ℕ⁺\) satisfaciendo que
    \[f(f(n))<f(n+1)\]
    para cualquier número \(n\). Probar que para todo número
    natural positivo \(n\) se verifica que
  \[f(n)=n.\]}}

Para llevar a cabo la formalizaciṕon en Lean del problema que
se ha planteado, se va a importar la teoría
\href{https://github.com/leanprover-community/mathlib/blob/
  master/src/data/pnat/basic.lean}{data.pnat.basic} sobre el conjunto
formado por los números naturales que son positivos.

En Lean esto lo expresamod como sigue:
\begin{leancode}
import data.pnat.basic
\end{leancode}

Para llevar a cabo la resolución del problema que se nos plantea,
se va introducir un teorema auxiliar, el cual consistirá en
demostrar el mismo resultado que se nos plantea pero para una
función que está definida en todo el conjunto de los números
naturales. Enunciemos dicho teorema:

\begin{teorema}[Extension]\label{extension}
  Sea \(f:ℕ → ℕ\) una función satisfaciendo que
  \begin{equation}
    f(f(n))<f(n+1)
  \end{equation}
  para cualquier número natural \(n\). Entonces se tiene que
  para todo número natural \(n\) se verifica que
  \begin{equation}
    f(n)=n.
  \end{equation}
\end{teorema}
\begin{demostracion}
  Suponiendo que se tiene una función definida entre el conjunto
  de los números naturales, es decir, \(f: ℕ → ℕ\), verificando
  que
  \begin{equation}\label{h1}\tag{h1}
     f(f(n))<f(n+1);
  \end{equation}
  se tiene que demostrar que para cualquier número natural se cumple
  que \(f(n)\) es igual a \(n\). Simbólicamente:
  \begin{equation}
    ∀ n, f(n)=n.
  \end{equation}

  Previo a la demostración en sí de este resultado, se va a probar
  un resultado auxiliar que nos será necesario para la conclusión
  del primero. Destacar que este teorema auxiliar se va a demostrar
  bajo las condiciones que se encuentra el problema.

  Hay que demostrar que dados dos números naturales cualesquiera
  \(k,n\) tales que \(k\) es menor o igual que \(n\). Entonces
  se tiene que \(k\) es menor o igual que \(f(n)\). Simbólicamente:
  \begin{equation}\label{h2}\tag{h2}
    ∀ k, n ∈ ℕ, k≤ n ⟶ k ≤ f(n).
  \end{equation}

  Para demostrar (\ref{h2}), aplicaremos el método de inducción
  en el número natural \(k\):

  Probar por inducción en \(k\) el resultado consiste en demostrar
  que para el caso en el que \(k\) es nulo y que suponiendo que se
  verifica para \(k\), entonces se verifica para \(k\) más uno.
  Distingamos entonces los casos:
  
  \begin{itemize}
    \item \textbf{Caso \(k=0\).}

      En este caso, hay que demostrar que
      \begin{equation}\label{dem}
        0≤f(n),
      \end{equation}
      bajo las hipótesis (\ref{h1}) y que para todo número natural
      \(n\) se verifica que
      \begin{equation}
        0≤ n. 
      \end{equation}

      Para demostrar (\ref{dem}) basta con aplicar que por
      definición todo número natural es mayor igual que cero.
      Entonces, como consecuencia de que la función \(f\) está
      definida entre el conjunto de los números naturales, se tiene
      que \(f(n)\) siempre va a ser mayor o igual que cero para
      cualquier \(n\) natural.
      
    \item \textbf{Caso \(k+1\).}
    
      En este caso, asumiendo como cierta la hipótesis de inducción,
      es decir, suponiendo que se verifica que
      \begin{equation}\label{hind}\tag{h\(\_\)ind}
        ∀ n ∈ ℕ, k≤ n ⟶ k ≤ f(n),
      \end{equation}
      hay que demostrar que para todo número natural \(n\) tal que
      sea mayor o igual que \(k+1\), entonces se verifica que
      \begin{equation}\label{hind2}
        k+1 ≤ f(n).
      \end{equation}

      Ahora bien, demostrar (\ref{hind2}) es equivalente a demostrar
      que \(k\) sea menor entricto que \(f(n)\), esto es:
      \begin{equation}
        k<f(n).
      \end{equation}

      A continuación, se va a deducir una serie de hipótesis que se
      tienen de manera casi directa y que nos ayudarán a la
      resolución del problema principal.
    \end{itemize}
    Hola
\end{demostracion}

La formalización en Lean de este teorema sería la siguiente:
\begin{leancode}
theorem Extension
  (f : ℕ → ℕ)
  (h1 : ∀ n, f (f n) < f (n + 1))
  : ∀ n, f n = n :=
begin
  have h2: ∀ (k n : ℕ), k ≤ n → k ≤ f n,
  { intro k,
     induction k with k h_ind,
     { intros n hn,
       exact nat.zero_le (f n), },
     { intros n hk,
       apply nat.succ_le_of_lt,
       rw nat.succ_eq_add_one at hk,
       have hk1: k ≤ n-1 := nat.le_sub_right_of_add_le hk,
       have hk2: k ≤ f (n-1):= h_ind (n-1) hk1,
       have hk3: k ≤ f(f(n-1)) := h_ind (f(n-1)) hk2,
       have h11: f (f (n-1)) < f(n-1+1):= h1 (n-1),
       rw nat.sub_add_cancel at h11,
       { calc k ≤ f(f(n-1)) : hk3
            ... < f(n)      : h11,},
       have hk0: 1 ≤ k+1 := nat.succ_le_succ (nat.zero_le k),
       exact (le_trans hk0 hk), }},
  have hf: ∀ n, n ≤ f n,
    { intro n,
      apply h2 n n,
      exact le_rfl, },
  have mon: ∀ n, f n < f(n+1),
    { intro n,
      exact lt_of_le_of_lt (hf (f n)) (h1 n), },
  have f_mon: strict_mono f := strict_mono.nat mon,
  intro n,
  apply nat.eq_of_le_of_lt_succ (hf n),
  exact (f_mon.lt_iff_lt.mp (h1 n)),
end
\end{leancode}

Una vez ya se ha demostrado el teorema \ref{extension}, se va a
introducir el teorema con el cual el problema estaría resuleto.
El teorema que introduciremos a continuación, no es más que un caso3
particular del anterior.

<<<<<<< HEAD
\begin{teorema}[imo1977\(\_ \)q62]
  Sea \(f:ℕ⁺ → ℕ⁺\) una función satisfaciendo que
  \begin{equation}
    f(f(n))<f(n+1)
  \end{equation}
  para cualquier número natural positivo \(n\). Entonces se tiene
  que para todo número natural positivo \(n\) se verifica que
  \begin{equation}
    f(n)=n.
  \end{equation}
\end{teorema}
\begin{demostracion}
hola
\end{demostracion}

Finalmente, la formalización en Lean de este teorema nos quedaría:
\begin{leancode}
theorem imo1977_q62 (f : ℕ+ → ℕ+) (h : ∀ n, f (f n) < f (n + 1)) :
  ∀ n, f n = n :=
begin
  intro n,
  simpa using Extension (λ m, if 0 < m then f m.to_pnat' else 0) _ n,
  intro x,
  cases x,
  {simp},
  {simpa using h _},
end
\end{leancode}

=======
>>>>>>> 3d1f684c8187cde8e50e0b8b66086ff7a9052140
%%% Local Variables:
%%% mode: latex
%%% TeX-master: "IMO_en_Lean"
%%% End:
