\section{IMO 1977 Q6}

En esta sección se va a detallar la solución al problema Q6 que se
propuso en el año 1977. Comenzaremos presentando el enunciado de dicho
problema y su correspondiente solución en lenguaje
natural. Posteriormente se presentará la formalización en Lean de dicho
problema y el análisis de la misma.

Destacar que la formalización original del problema se ha obtenido de
\cite{TC} y fue propuesta por Tian Chen.

\noindent
\framebox{\parbox{\textwidth}{
    \textbf{Problema 3 (1977--Q6)}. Consideremos la función
    \(f:ℕ⁺ → ℕ⁺\) satisfaciendo que
    \[f(f(n)) < f(n+1)\]
    para cualquier número \(n\). Probar que para todo número
    natural positivo \(n\) se verifica que
    \[f(n) = n.\]}}

\begin{nota}
  En el enunciado del problema anterior se observa que la función \(f\)
  está definida del conjunto \(ℕ⁺\) en sí mismo; dicho conjunto se
  define como el conjunto de los números naturales positivos.
\end{nota}

Para llevar a cabo la formalización en Lean del problema que
se ha planteado, se va a importar la teoría
\href{https://github.com/leanprover-community/mathlib/blob/
  master/src/data/pnat/basic.lean}{data.pnat.basic} sobre el conjunto
formado por los números naturales que son positivos.

En Lean esto lo expresamos como sigue:
\begin{leancode}
import data.pnat.basic
\end{leancode}

Para llevar a cabo la resolución del problema que se nos plantea,
se va introducir un teorema auxiliar, el cual consistirá en
demostrar el mismo resultado que se nos plantea pero para una
función que está definida en todo el conjunto de los números
naturales. Enunciemos dicho teorema:

\begin{teorema}[Extension]\label{extension}
  Sea \(f:ℕ → ℕ\) una función satisfaciendo que
  \begin{equation}
    f(f(n)) < f(n+1)
  \end{equation}
  para cualquier número natural \(n\). Entonces se tiene que
  para todo número natural \(n\) se verifica que
  \begin{equation}
    f(n) = n.
  \end{equation}
\end{teorema}

\begin{demostracion}
  Suponiendo que se tiene una función \(f\) definida entre el conjunto
  de los números naturales; es decir, \(f: ℕ → ℕ\), verificando
  que
  \begin{equation}\label{h77_1}\tag{h1}
     f(f(n)) < f(n+1);
  \end{equation}
  se tiene que demostrar que para cualquier número natural se cumple
  que \(f(n)\) es igual a \(n\). Simbólicamente:
  \begin{equation}
    ∀ n ∈ ℕ, f(n) = n.
  \end{equation}

  Previo a la demostración en sí de este resultado, se van a probar una
  serie de resultados auxiliares que nos serán necesarios para la
  conclusión del primero. Destacar que estos resultados auxiliares que
  se van a demostrar se encuentran bajo las mismas condiciones que el
  problema principal.

  \noindent
  \textbf{\underline{Hipótesis h2}}

  Comenzaremos demostrando que dados dos números naturales cualesquiera
  \(k,n\) tales que \(k\) es menor o igual que \(n\).  Entonces se tiene
  que \(k\) es menor o igual que \(f(n)\).  Simbólicamente:
  \begin{equation}\label{h77_2}\tag{h2}
    ∀ k, n ∈ ℕ, k ≤ n ⟶ k ≤ f(n).
  \end{equation}

  Para demostrar (\ref{h77_2}), aplicaremos el método de inducción en el
  número natural \(k\):

  Probar por inducción en \(k\) el resultado consiste en demostrarlo
  para el caso en el que \(k\) es nulo y que suponiendo que se
  verifica para \(k\), entonces se verifica para \(k\) más uno.
  Distingamos entonces los casos:

  \begin{itemize}
    \item \textbf{Caso \(k=0\).}

      En este caso, hay que demostrar que
      \begin{equation}\label{dem}
        ∀n, 0 ≤ f(n),
      \end{equation}
      bajo las hipótesis (\ref{h77_1}) y que para todo número natural
      \(n\) se verifica que
      \begin{equation}
        0 ≤ n.
      \end{equation}

      Para demostrar (\ref{dem}) basta con aplicar que por definición
      todo número natural es mayor o igual que cero. Entonces, como
      consecuencia de que la función \(f\) está definida del conjunto de
      los números naturales en sí mismo, se tiene que \(f(n)\) siempre
      va a ser mayor o igual que cero para cualquier \(n\) natural.

    \item \textbf{Caso \(k+1\).}

      En este caso, asumiendo como cierta la hipótesis de inducción,
      es decir, suponiendo que se verifica que
      \begin{equation}\label{hind}\tag{h\(\_\)ind}
        ∀ n ∈ ℕ, k ≤ n ⟶ k ≤ f(n),
      \end{equation}
      hay que demostrar que para todo número natural \(n\) que
      sea mayor o igual que \(k+1\), se verifica que
      \begin{equation}\label{hind2}
        k+1 ≤ f(n).
      \end{equation}

      Una vez ya se ha introducido la hipótesis de inducción, se va
      a denotar la hipótesis de que para cualquier número natural
      \(n\) sea mayor igual que \(k+1\) como:
      \begin{equation}\label{hk}\tag{hk}
        k+1 ≤ n.
      \end{equation}

      Ahora bien, demostrar (\ref{hind2}) es equivalente a demostrar
      que \(k\) sea menor estricto que \(f(n)\), esto es:
      \begin{equation}\label{hind3}
        k < f(n).
      \end{equation}

      A continuación, se van a deducir una serie de hipótesis que se
      tienen de manera casi directa.
      \begin{itemize}
      \item A partir de la hipótesis (\ref{hk}), se tiene de manera
        directa que pasando restando el uno de una parte de la
        desigualdad a otra, se tiene que
        \begin{equation}\label{hk1}\tag{hk1}
          k ≤ n-1.
        \end{equation}

      \item Haciendo uso ahora de la hipótesis de inducción,
        (\ref{hind}), y de la hipótesis anteriormente deducida,
        (\ref{hk1}), se puede deducir que
        \begin{equation}\label{h77_k2}\tag{hk2}
          k ≤ f(n-1).
        \end{equation}

      \item De manera totalmente análoga en como se ha deducido
        la hipótesis (\ref{h77_k2}), se puede deducir que
        \begin{equation}\label{hk3}\tag{hk3}
          k ≤ f(f(n-1)),
        \end{equation}
        sin más que aplicar la hipótesis de inducción, (\ref{hind}),
        y el resultado anteriormente deducido, (\ref{h77_k2}).

      \item La última hipótesis que introduciremos en este apartado,
        será haciendo uso de la hipótesis principal del teorema,
        es decir, (\ref{h77_1}). Se particularizará para el caso en el
        que se considera \(n-1\), de esta forma se tendría:
        \begin{equation}\label{exph11}
          f(f(n-1)) < f(n-1+1).
        \end{equation}

        Ahora bien, la expresión (\ref{exph11}) se puede reescribir
        como
        \begin{equation}\label{h77_11}\tag{h11}
          f(f(n-1)) < f(n).
        \end{equation}
        Esta reescritura se ha llevado a cabo mediante el uso de
        la cancelación en la suma de los números naturales, esto es:

        \begin{proposicion}\label{sumanat}
          Sean \(m\) y \(n\) dos números pertenecientes al conjunto
          de los números naturales tales que \(m\) es menor o igual
          que \(n\). Entonces se tiene que si al número \(n\) le
          restamos y sumamos el número \(m\) el resultado que se
          obtiene es \(n\). Simbólicamente:
          \begin{equation}
            ∀ n,m ∈ ℕ, m ≤ n ⟶ n-m+m = n.
          \end{equation}
        \end{proposicion}

        La formalización en Lean de esta proposición es la que se
        presente a continuación:
        \begin{leancode}
          theorem sub_add_cancel {n m : ℕ} (h : m ≤ n) : n - m + m = n
        \end{leancode}
      \end{itemize}

      De esta forma, al haber hecho uso de la proposición \ref{sumanat},
      el objetivo a demostrar que antes era (\ref{hind3}), ahora se le
      añade otro nuevo objetivo y es demostrar que además se cumple que
      el número natural \(n\) es mayor o igual que \(1\), esto es:
      \begin{equation}\label{n1}
        n ≥ 1.
      \end{equation}

      Por claridad en la notación, estos dos objetivos mencionados
      se van a dividir en dos problemas diferentes:
      \begin{enumerate}
      \item \textbf{Problema 1}

        Demostrar (\ref{hind3}) es bastante sencillo con ayuda de
        las hipótesis deducidas anteoriormente. Veámoslo:
        \begin{equation}
          k\stackrel{(*)}{≤}f(f(n-1))\stackrel{(**)}{<}f(n),
        \end{equation}
        donde en \((*)\) se ha hecho uso de la hipótesis (\ref{hk3})
        y en \((**)\) se ha usado la hipótesis (\ref{h77_11}).

      \item \textbf{Problema 2}

        Ahora se tiene que demostrar (\ref{n1}) también bajo todas
        las hipótesis que se han deducido posteriormente. Aunque
        realmente para probar este resultado no son necesarias,
        sólo se necesitan los dos siguientes resultados generales
        de los números naturales:

        \begin{lema}\label{lemaNatCero}
          Sea \(n\) un número natural cualquiera, entonces dicho
          número \(n\) es mayor o igual que cero.
        \end{lema}

        Cuya formalización en Lean es:
        \begin{leancode}
          lemma zero_le : ∀ (n : ℕ), 0 ≤ n
        \end{leancode}

        \begin{lema}\label{lemaNatSuc}
          Sean \(n\) y \(m\) dos números naturales tales que \(n\)
          es menor o igual que \(m\). Entonces se tiene que el
          número natural sucesor de \(n\) (es decir, \(n+1\)) es
          menor o igual que el sucesor de \(m\). Simbólicamente:
          \begin{equation}
            ∀n, m ∈ ℕ, n ≤ m ⟶ n+1 ≤ m+1.
          \end{equation}
        \end{lema}

        Y la formalización en Lean de este lema es la siguiente:
        \begin{leancode}
          lemma succ_le_succ {n m : ℕ} : n ≤ m → succ n ≤ succ m
        \end{leancode}

        De esta forma se tendría que si consideramos un número
        natural \(k\), haciendo uso del lema \ref{lemaNatCero}, se
        tiene que
        \begin{equation}\label{k0}
          k ≥ 0.
        \end{equation}

        Si además, ahora se hace uso del lema \ref{lemaNatSuc}
        considerando los números naturales \(k\) y \(0\) y la
        hipótesis (\ref{k0}), se puede concluir que
        \begin{equation}\label{hk0}\tag{hk0}
          1 ≤ k+1
        \end{equation}

        Ahora bien, para finalizar la prueba de este problema basta
        con aplicar la transitividad de las relaciones menor o igual que
        y menor que sobre las hipótesis (\ref{hk}) y (\ref{hk0}).

      \end{enumerate}

    Una vez ya se han demostrado los objetivos (\ref{hind3}) y
    (\ref{n1}), se tiene que ya se ha probado el caso \(k+1\) del
    método de inducción.
    \end{itemize}

  De esta forma, tras haber terminado la prueba de la hipótesis
  (\ref{h77_2}) por el método de inducción, se procederá a la
  continuación de la prueba.

  \noindent
  \textbf{\underline{Hipótesis hf}}

  Proseguiremos con la demostración de que para cualquier número
  natural \(n\) se verifica que dicho número \(n\) es menor o igual
  que \(f(n)\). Simbólicamente:
  \begin{equation}\label{hf}\tag{hf}
    ∀n∈ ℕ, n ≤ f(n).
  \end{equation}

  Fijando un número natural \(n\) cualesquiera, se tiene que le
  podemos aplicar la hipótesis (\ref{h77_2}) considerando que los dos
  números naturales que aparecen en el enunciado de esa hipótesis
  son el mismo \(n\).

  Como consecuencia de que sabemos que cualquier número es menor
  o igual que sí mismo, se tiene que la condición de (\ref{h77_2}) se
  verifica y por tanto se concluye que
  \begin{equation}
    n ≤ f(n),
  \end{equation}
  que era lo que queríamos demostrar.

  \noindent
  \textbf{\underline{Hipótesis mon}}

  El enunciado de la esta hipótesis que queremos demostrar que se
  verifica en nuestro problema es que para cualquier número natural
  \(n\) se verifica que \(f(n)\) es menor estrictamente que \(f(n+1)\).
  Simbólicamente:
  \begin{equation}\label{mon}\tag{mon}
    ∀n ∈ ℕ, f(n) < f(n+1).
  \end{equation}

  Para demostrar que (\ref{mon}) es cierto bajo las hipótesis de nuestro
  problema, comenzamos fijando un número natural \(n\).

  Ahora bien, por un lado se tiene que aplicando la hipótesis
  (\ref{h77_1}) a dicho número natural \(n\) resulta que
  \begin{equation}\label{mon1}
    f(f(n)) < f(n+1).
  \end{equation}

  Por otro lado, se sabe que la imagen de aplicar la función \(f\) a un
  número natural \(n\) es un número natural; es decir, \(f(n)\)
  pertenece al conjunto de los número naturales. Entonces, si aplicamos
  la hipótesis (\ref{hf}) a \(f(n\)) se tiene que
  \begin{equation}\label{mon2}
    f(n) ≤ f(f(n)).
  \end{equation}

  De manera que utilizando la transitividad de las relaciones menor o
  igual que y menor que en los resultados obtenidos en (\ref{mon1}) y
  (\ref{mon2}) ya se tendría el resultado deseado, (\ref{mon}).

  \noindent
  \textbf{\underline{Hipótesis f\(\_\)mon}}

  Este va a ser el último resultado que vamos a demostrar previo a la
  conclusión en sí del teorema y consiste en demostrar que la función
  \(f\) que se está considerando es estrictamente monótona.

  \begin{definicion}\label{def:mono}
    Se dice que una función \(f\) es estrictamente monótona si para
    cualesquiera \(a\) y \(b\) tales que \(a\) es menor estrictamente
    que \(b\), se tiene que \(f(a)\) es también menor estrictamente que
    \(f(b)\). Simbólicamente:
    \begin{equation}
      ∀a,b, a < b ⟶ f(a) < f(b).
    \end{equation}
  \end{definicion}

  Esta definición en Lean tiene la siguiente formalización:
  \begin{leancode}
    def strict_mono [has_lt α] [has_lt β] (f : α → β) : Prop :=
    ∀ ⦃a b⦄, a < b → f a < f b
  \end{leancode}

  En el caso de nuestra función, que se encuentra definida en el
  conjunto de los números naturales, existe un lema el cual nos da las
  condiciones para que la función sea estrictamente monótona.

  \begin{lema}\label{lem:mono}
    Sea \(f\) una función que tiene como conjunto de salida el conjunto
    de los números naturales y sea el conjunto de llegada un conjunto
    cualquiera que sea preorden, es decir, \(f:ℕ → β\) donde se ha
    denotado \(β\) como el preorden. Se tiene que si para cualquier
    número natural \(n\) se verifica que \(f(n)\) es menor estrictamente
    que \(f(n+1)\), entonces \(f\) es una función estrictamente
    monótona.
  \end{lema}

  Y su correspondiente formalización en Lean sería:
  \begin{leancode}
    lemma nat {β} [preorder β] {f : ℕ → β} (h : ∀n, f n < f (n+1)) :
    strict_mono f
  \end{leancode}

  De esta forma, como nuestra función \(f\) verifica la hipótesis
  (\ref{hf}), al hacer uso del lema \ref{lem:mono} se tendría ya que la
  función de nuestro problema es estrictamente monótona.

  Una vez ya se han introducido todos los resultados auxiliares que eran
  necesarios, recordemos que el objetivo a demostrar era que dada la
  función \(f\) definida entre el conjunto de los números naturales tal
  que se verificaba la hipótesis (\ref{h77_1}). Había que demostrar que
  para cualquier número natural \(n\) se verifica que \(f(n)\) es igual
  a \(n\). Simbólicamente era:
  \begin{equation}\label{objetivo}
    ∀n, f(n) = n.
  \end{equation}

  Para demostrar (\ref{objetivo}), fijamos un número natural \(n\),
  entonces se tiene que probar que se verifica la igualdad
  \begin{equation}\label{objetivo2}
    f(n) = n.
  \end{equation}

  Haciendo uso de la hipótesis (\ref{hf}) para el número natural \(n\)
  que estamos considerando se tiene entonces que
  \begin{equation}
    n ≤ f(n).
  \end{equation}

  Por tanto, haciendo uso de esto, el objetivo (\ref{objetivo2}) pasa a
  convertirse en
  \begin{equation}\label{objetivo3}
    f(n) < n+1.
  \end{equation}

  Para concluir la demostración basta con probar (\ref{objetivo3}); esto
  se puede hacer de manera directa con la hipótesis principal del
  problema, (\ref{h77_1}), y la monotonía de la función \(f\), hipótesis
  (\ref{def:mono})
\end{demostracion}

La formalización en Lean de este teorema sería la siguiente:
\begin{leancode}
theorem Extension
  (f : ℕ → ℕ)
  (h1 : ∀ n, f (f n) < f (n + 1))
  : ∀ n, f n = n :=
begin
  have h2: ∀ (k n : ℕ), k ≤ n → k ≤ f n,
  { intro k,
     induction k with k h_ind,
     { intros n hn,
       exact nat.zero_le (f n), },
     { intros n hk,
       apply nat.succ_le_of_lt,
       rw nat.succ_eq_add_one at hk,
       have hk1: k ≤ n-1 := nat.le_sub_right_of_add_le hk,
       have hk2: k ≤ f (n-1):= h_ind (n-1) hk1,
       have hk3: k ≤ f(f(n-1)) := h_ind (f(n-1)) hk2,
       have h11: f (f (n-1)) < f(n-1+1):= h1 (n-1),
       rw nat.sub_add_cancel at h11,
       { calc k ≤ f(f(n-1)) : hk3
            ... < f(n)      : h11,},
       have hk0: 1 ≤ k+1 := nat.succ_le_succ (nat.zero_le k),
       exact (le_trans hk0 hk), }},
  have hf: ∀ n, n ≤ f n,
    { intro n,
      apply h2 n n,
      exact le_rfl, },
  have mon: ∀ n, f n < f(n+1),
    { intro n,
      exact lt_of_le_of_lt (hf (f n)) (h1 n), },
  have f_mon: strict_mono f := strict_mono.nat mon,
  intro n,
  apply nat.eq_of_le_of_lt_succ (hf n),
  exact (f_mon.lt_iff_lt.mp (h1 n)),
end
\end{leancode}
\index{Extension}

En la formalización de este teorema Lean han sido necesarios el uso
de los siguiente lemas y teoremas:
\begin{itemize}
\item \mint{lean}|nat.zero_le : ∀ (n: ℕ), 0≤n|
  \indLema{zero\_le}
\item \mint{lean}|succ_le_of_lt {a b : ℕ} (h : a < b) : succ a ≤ b|
  \indLema{succ\_le\_of\_lt}
\item \mint{lean}|succ_eq_add_one (n : ℕ) : succ n = n + 1|
  \indLema{succ\_eq\_add\_one}
\item \mint{lean}|le_sub_right_of_add_le (h : m + k ≤ n) : m ≤ n - k|
  \indLema{le\_sub\_right\_of\_add\_le}
\item \mint{lean}|sub_add_cancel {n m : ℕ} (h : m ≤ n) : n - m + m = n|
  \indLema{sub\_add\_cancel}
\item \mint{lean}|succ_le_succ {n m : ℕ} : n ≤ m → succ n ≤ succ m |
  \indLema{succ\_le\_succ}
\item \mint{lean}|le_trans : ∀ {a b c : α}, a ≤ b → b ≤ c → a ≤ c|
  \indLema{le\_trans}
\item \mint{lean}|le_rfl [preorder α] {x : α} : x ≤ x |
  \indLema{le\_rfl}
\item \mint{lean}|lt_of_le_of_lt : ∀ {a b c : α}, a ≤ b → b < c → a < c |
  \indLema{lt\_of\_le\_of\_lt}
\item \mint{lean}|nat {β} [preorder β] {f : ℕ → β} (h : ∀n, f n < f (n+1)) :  strict_mono f |
  \indLema{strict\_mono.nat}
\item \mint{lean}|eq_of_le_of_lt_succ {n m : ℕ} (h₁ : n ≤ m) (h₂ : m < n + 1) : m = n |
  \indLema{eq\_of\_le\_of\_lt\_succ}
\end{itemize}
Además,se han usado las tácticas
\tactica{apply}{apply}.
\tactica{exact}{exact},
\tactica{have}{have},
\tactica{induction}{induction},
\tactica{intro}{intro} y
\tactica{rw / rewrite}{rw}.

Una vez ya se ha demostrado el teorema \ref{extension}, se va a
introducir el teorema con el cual el problema estaría resuelto. El
teorema que se va a presentar a continuación no es más que un caso
particular del teorema \ref{extension}, ya probado.

La diferencia entre estos dos teoremas es que mientras que en
\ref{extension} se consideraba que la función \(f\) estaba definida
del conjunto de los números naturales en sí mismo; en el siguiente
teorema, dicha función estará definida del conjunto de los números
naturales positivos en sí mismo. Veámoslo:

\begin{teorema}[imo1977\(\_ \)q6]
  Sea \(f:ℕ⁺ → ℕ⁺\) una función satisfaciendo que
  \begin{equation}
    f(f(n)) < f(n+1)
  \end{equation}
  para cualquier número natural positivo \(n\). Entonces se tiene
  que para todo número natural positivo \(n\) se verifica que
  \begin{equation}\label{objf}
    f(n) = n.
  \end{equation}
\end{teorema}

\begin{demostracion}
  Considerando la función \(f\) definida del conjunto de los números
  naturales positivos en sí mismo verificando que
  \begin{equation}\label{h}\tag{h}
    ∀ n ∈ ℕ⁺, f(f(n)) < f(n+1).
  \end{equation}
  Entonces, se quiere probar que para cualquier número natural positivo
  \(n\) se cumple (\ref{objf}).

  En primer lugar, fijemos un número natural positivo arbitrario
  \(n\). A continuación, se hace uso del teorema \ref{extension}, el
  cual está definido sobre el conjunto de los números
  naturales. Aplicamos este teorema sobre una función que está definida
  sobre el conjunto de los número naturales, de manera que si
  consideramos un número \(m\) natural se distinguen dos casos:
  \begin{enumerate}
  \item Que el número natural \(m\) sea positivo. En esta situación,
    la función se define como \(f(m)\), donde \(f\) es la definida en el
    enunciado de este teorema.
  \item Que el número natural \(m\) sea nulo. En este caso la aplicación
    definida nos lleva éste número al cero.
  \end{enumerate}

  Si denotamos esta función por \(f_1\), simbólicamente se tendría:
  \begin{equation}\label{f1}
    f_1(m)=\left\lbrace\begin{array}{ll}
                         f(m), & \text{si } m>0, \\
                         0, & \text{si } m=0.
                       \end{array}\right.
  \end{equation}

  Ahora bien, para poder aplicar el teorema \ref{extension} y de esta
  forma poder concluir la demostración, faltaría probar que se verifica
  la hipótesis del problema. Esto es:
  \begin{equation}
    ∀x∈ ℕ, f_1(f_1(x)) < f_1(x+1),
  \end{equation}

  Para ello, comenzaremos fijando un número natural \(x\) arbitrario. De
  esta forma, se tiene que probar que
  \begin{equation}\label{probar}
    f_1(f_1(x)) <f_1(x+1),
  \end{equation}

  Como consecuencia de como se ha definido la función \(f_1\),
  (\ref{f1}), para demostrar (\ref{probar}) vamos a dividir la prueba en
  dos casos distintos:
  \begin{itemize}
  \item \textbf{Caso \(m=0\)}

    En este caso la expresión a demostrar (\ref{probar}), se convierte
    en
    \begin{equation}\label{probar2}
      f_1(f_1(0)) < f_1(0+1).
    \end{equation}

    Sin más que aplicar la definición de la función, (\ref{f1}), se
    tiene que demostrar (\ref{probar2}) se convierte en
    \begin{equation}
      0 < f(1).
    \end{equation}

    Este resultado se tiene de manera inmediata pues el conjunto imagen
    de la función \(f\) es el conjunto de los números naturales
    positivos.

  \item \textbf{Caso \(m>0\)}

    En este caso, se tiene que por la definición de \(f_1\), probar
    (\ref{probar}) se convierte en
    \begin{equation}\label{probar3}
    f(f(x)) < f(x+1).
    \end{equation}

    Pero este resultado es inmediato haciendo uso de la hipótesis
    (\ref{h}) y de esta forma se tendría también demostrado este caso.
  \end{itemize}

  Concluyendo, tras la demostración de estos dos casos, ya se tendría
  la prueba del teorema que buscábamos.
\end{demostracion}

Finalmente, la formalización en Lean de este teorema sería:
\begin{leancode}
theorem imo1977_q6
  (f : ℕ+ → ℕ+)
  (h : ∀ n, f (f n) < f (n + 1))
  : ∀ n, f n = n :=
begin
  intro n,
  simpa using Extension (λ m, if 0 < m then f m.to_pnat' else 0) _ n,
  intro x,
  cases x,
  { simp },
  { simpa using h _},
end
\end{leancode}
\index{imo1977\_q6}

En la formalización en Lean de este teorema sólo ha sido necesario el
uso de la siguiente aplicación:
\begin{leancode}
def to_pnat' (n : ℕ) : ℕ+ := succ_pnat (pred n)
\end{leancode}
Además,se han usado las tácticas
\tactica{cases}{cases},
\tactica{intro}{intro},
\tactica{simpa}{simpa} y
\tactica{simp}{simp}.

%%% Local Variables:
%%% mode: latex
%%% TeX-master: "IMO_en_Lean"
%%% End:
