\chapter{Tácticas en Lean}\label{apentacti}

En este apéndice se van a explicar de manera detallada las diferentes
tácticas usadas durante el desarrollo del trabajo. Con el objetivo de
clarificar las ideas, se plantearán diferentes ejemplos para ver cómo
actuan las diversas tácticas.

Muchos de los ejemplos que he usado para la clarificación de las
diferentes tácticas han sido obtenidos del tutorial que estudié para
comenzar a trabajar con Lean, en concreto \cite{tutor}.

\section{Táctica sorry}

La primera táctica que vamos a estudiar es la táctica
\dtactica{sorry}{sorry}.  En particular, esta táctica es muy útil y tiene
una función muy sencilla: es capaz de aceptar cualquier resultado. Ahora
bien, a nosotros no nos interesa usar esta táctica para la formalización
como tal de las pruebas pero sí como herramienta de ayuda que nos
permita formalizar resultados auxiliares o hechos dentro de una prueba.

Realmente cualquier ejemplo, lema o teorema sería válido para ver cómo
funciona esta táctica. Veamos el siguiente:
\begin{leancode}
example (h₁ : a ∣ b) (h₂ : b ∣ c) : a ∣ c :=
begin
  sorry,
end
\end{leancode}

Asimismo, veamos un ejemplo en el que la táctica \dtactica{sorry}{sorry} sea
utilizada con la función de probar un hecho dentro de la prueba. Es más,
en este ejemplo aparecen las dos funciones mencionadas de la táctica:
\begin{leancode}
example (a b : ℝ) : a = a*b → a = 0 ∨ b = 1 :=
begin
  have H : a*(1 - b) = 0, by sorry,
  sorry,
end
\end{leancode}

\section{Táctica rewrite}

A continuación, se presenta la táctica \dtactica{rw / rewrite}{rw}. Esta táctica
es muy usada y su forma de funcionar es muy intuitiva: consiste en reemplazar
la ecuación o el si y solamente si que se encuentre después de la táctica
\dtactica{rw / rewrite}{rw} sobre el objetivo a demotrar.

Si a continuación de la táctica \dtactica{rw / rewrite}{rw} se encuentra una
flecha hacia la izquierda (\(←\)) se tiene que la sustitución es aplicada
al revés.

En el ejemplo que se presenta a continuación se pueden observar los dos usos
de la táctica \dtactica{rw / rewrite}{rw} que hemos mencionado:

\begin{leancode}
example (a b : ℝ) : (a + b)*(a - b) = a^2 - b^2 :=
begin
  rw mul_sub (a+b) a b,
  rw add_mul a b b,
  rw pow_two a,
  rw pow_two b,
  rw mul_comm a b,
  rw ← sub_sub ((a+b)*a) (b*a) (b*b),
  rw add_mul a b a,
  rw ← add_sub,
  rw sub_self,
  rw add_zero (a*a),
end
\end{leancode}

Asimismo, otra función de la táctica \dtactica{rw / rewrite}{rw} de la cual se
ha hecho uso durante el desarrollo del trabajo es aplicar la sustitución en
alguna hipótesis y no en el objetivo a demostrar. Esto se hace haciendo uso
del predicado \texttt{at}; veámoslo en Lean:
\begin{leancode}
example (a b c d : ℝ) (hyp : c = d*a + b) (hyp' : b = a*d) : c = 2*a*d :=
begin
  rw hyp' at hyp,
  rw mul_comm d a at hyp,
  rw ← two_mul (a*d) at hyp,
  rw ← mul_assoc 2 a d at hyp,
  exact hyp,
end
\end{leancode}

En este ejemplo se pueden observar las tres funciones descritas de la táctica
\dtactica{rw / rewrite}{rw}.

\section{Táctica have}

La siguiente táctica a estudiar será la táctica
\dtactica{have}{have}. Esta táctica es usada cuando se quieren introducir
nuevos lemas al problema que tendrán que ser demostrados luego.

En este caso el mismo ejemplo que uno de los que usamos para estudiar la táctica
\dtactica{sorry}{sorry}. Este ejemplo era:

\begin{leancode}
example (a b : ℝ) : a = a*b → a = 0 ∨ b = 1 :=
begin
  intro hyp,
  have H : a*(1 - b) = 0, by sorry,
  sorry,
end
\end{leancode}

\section{Táctica exact}

La táctica \dtactica{exact}{exact} ha sido una de las más utilizadas en
el trabajo y cuya función es muy simple: nos introduce una prueba
directa del objetivo a demostrar.

Para ver un ejemplo del uso de esta táctica, también se va a hacer uso de uno
ya propuesto anteriormente y es uno de los que se estudió al ver la táctica
\dtactica{rw / rewrite}{rw}. El ejemplo era el siguiente:

\begin{leancode}
example (a b c d : ℝ) (hyp : c = d*a + b) (hyp' : b = a*d) : c = 2*a*d :=
begin
  rw hyp' at hyp,
  rw mul_comm d a at hyp,
  rw ← two_mul (a*d) at hyp,
  rw ← mul_assoc 2 a d at hyp,
  exact hyp,
end
\end{leancode}

\section{Táctica intro}

La siguiente táctica que se va a introducir es \dtactica{intro / intros}{intro},
esta táctica puede ser utilizada de diversas maneras. Una manera muy común de
usarla es cuando tenemos que demostrar una implicación, entonces se supone
como cierta la primera parte de la implicación y esto se hace a través de la
táctica \dtactica{intro / intros}{intro}.

En el siguiente ejemplo se ve muy claro:

\begin{leancode}
example (a b : ℝ): 0 ≤ b → a ≤ a + b :=
begin
  intro hb,
  exact le_add_of_nonneg_right hb,
end
\end{leancode}

En el siguiente ejemplo que se plantea, se hace uso intros, en este caso como
se quiere demostrar que \(Q\) implica no \(P\), al decirle intros q p, le
estamos diciendo que introduzca las hipótesis de que se verifica \(Q\) y
también \(P\). De esta manera, el objetivo a demostrar pasa a ser false.
\begin{leancode}
example (P Q : Prop) (h₁ : P ∨ Q) (h₂ : ¬ (P ∧ Q)) : Q → ¬P :=
begin
  intros q p,
  exact h₂ ⟨p,q⟩ ,
end
\end{leancode}

\section{Táctica apply}

La táctica \dtactica{apply}{apply} es la siguiente que vamos a introducir. La
función de esta táctica trata de unificar el objetivo a demostrar con la
conclusión de un resultado auxiliar (el que se especifique justo después de
aplicar la táctica). De manera que si unifica, los objetivos a demostrar pasan
a ser las diversas premisas que tuviese el resultado usado.

Veámos un ejemplo en el que se utiliza esta táctica:
\begin{leancode}
def non_decreasing (f : ℝ → ℝ) := ∀ x₁ x₂, x₁ ≤ x₂ → f x₁ ≤ f x₂
def non_increasing (f : ℝ → ℝ) := ∀ x₁ x₂, x₁ ≤ x₂ → f x₁ ≥ f x₂

example (f g : ℝ → ℝ) (hf : non_decreasing f) (hg : non_increasing g) :
non_increasing (g ∘ f) :=
begin
  intros x1 x2 h,
  apply hg,
  apply hf,
  exact h,
end
\end{leancode}

\section{Táctica linarith}

La siguiente táctica ha sido otra muy utilizada en el trabajo y es
\dtactica{linarith}{linarith}. Esta táctica es capaz de probar multitud de
igualdades y desigualdades de manera directa; realemente, es capaz de probar
casi cualquier problema lineal. En el ejemplo que se plantea a continuación
se puede ver de manera muy clara:

\begin{leancode}
example (a b : ℝ) (ha : 0 ≤ a) (hb : 0 ≤ b) : 0 ≤ a + b :=
begin
  linarith,
end
\end{leancode}

Se tienen dos hipótesis: por un lado que \(x\) es menor o igual que \(y\) y por
otro que \(y\) es menor o igual que \(x\). Entonces, con la táctica
\dtactica{linarith}{linarith} se prueba directamente que \(x\) es igual a \(y\).

\comentario{En el ejemplo no aparece nada sobre la x y la y.}

\section{Táctica nlinarith}

La táctica \dtactica{nlinarith}{nlinarith}, que es la que vamos a estudiar a
continuaón, es muy parecida a la táctica \dtactica{linarith}{linarith} que
acabamos de describir. Realmente la táctica \dtactica{nlinarith}{nlinarith} no
es más que una extensión de la táctica \dtactica{linarith}{linarith} que puede
resolver problemas no lineales.

Para ver un ejemplo sobre esta táctica, se ha usado uno que se corresponde con
un lema auxiliar del problema Q6 de 2001 de las Olimpiadas Internacionales de
Mátemáticas. El ejemplo es el que se plantea a continuación:

\begin{leancode}
example (a b c d : ℤ)
  (hba : b < a)
  (hcb : c < b)
  (hdc : d < c)
  (h : a*c + b*d = (a + b - c + d) * (-a + b + c + d))
  : a*c + b*d < a*b + c*d:=
begin
  nlinarith,
end
\end{leancode}

\section{Táctica assume}

La siguiente táctica que se estudia es la táctica \dtactica{assume}{assume}. Como
su propio nombre indica nos sirve para asumir o fijar una variable o incluso una
hipótesis.

Cuando nos encontramos ante un resultado en el que hay que demostrar un para
todo, esta táctica es muy útil. Veamos el siguiente ejemplo:

\begin{leancode}
variables{n : ℕ}
example : ∀ n, 3*n=n*3 :=
begin
  assume n,
  sorry,
end
\end{leancode}

Se puede observar que el objetivo a demostrar pasa a ser sólo y excluivamente
probar la igualdad puesto que hemos fijado el número \(n\) perteneciente al
conjunto de los números naturales.

\section{Táctica by contradiction}

La táctica \dtactica{by_contra / by_contradiction}{by\_contradiction} es muy
usada en las pruebas matemáticas. Esta táctica consiste en suponer que el
objetivo a demostrar no se verifica y que se acabe llegando a una contradicción.

Un ejemplo en el que el uso de esta táctica se ver muy claro es el siguiente:

\begin{leancode}
example (P Q : Prop) (h : ¬ Q → ¬ P) : P → Q :=
begin
  intro hP,
  by_contradiction hnQ,
  exact h hnQ hP,
end
\end{leancode}

En la primera línea, se introduce como hipótesis hP que se tiene la proposición
\(P\) y luego se denota como hipótesis hnQ que no se tiene \(Q\). En la última
línea es cuando ya se llega a la contradicción.

\section{Táctica let}

La táctica \dtactica{let}{let} que ha sido usada en alguna formalización que
se ha realizado en Lean consiste en introducir una nueva hipótesis (a la que
se puede nombrar si así se desea).

En el ejemplo que se tiene a continuación, se puede ver que se ha introducido
una nueva hipótesis a través de la táctica \dtactica{let}{let}:

\begin{leancode}
variables (u v w : ℕ → ℝ) (l l' : ℝ)
notation `|`x`|` := abs x
def seq_limit (u : ℕ → ℝ) (l : ℝ) : Prop :=
∀ ε > 0, ∃ N, ∀ n ≥ N, |u n - l| ≤ ε

def tendsto_infinity (u : ℕ → ℝ) := ∀ A, ∃ N, ∀ n ≥ N, u n ≥ A

example {u : ℕ → ℝ} : tendsto_infinity u → ∀ l, ¬ seq_limit u l :=
begin
  intro h,
  intro l,
  intro lim,
  cases lim 1 (by linarith) with N1 hN1,
  cases h (l+2) with N2 hN2,
  let N3 := max N1 N2,
  sorry,
end
\end{leancode}

\section{Táctica use}

La táctica \dtactica{use}{use} irá acompañado de una expresión. Esta
táctica será utilizada cuando nos encontremos que hay que probar que
existe un elemento tal que verifica una condición; entonces, cuando le
queremos decir que considere uno en concreto se lo decimos a Lean
a través de la táctica \dtactica{use}{use}.

En el siguiente ejemplo se puede ver de manera inmediata:

\begin{leancode}
example : ∃ n : ℕ, 8 = 2*n :=
begin
  use 4,
  refl,
end
\end{leancode}

\section{Táctica induction}

La táctica \dtactica{induction}{induction} es uno de los mecanismos más famosos
para demostrar resultados en Matemáticas. Simplemente consiste en demostrar
un resultado por inducción, es decir, si hacemos inducción el número \(b\),
se tiene que demostrar el resultado para \(b\) igual a \(0\) y luego, suponiendo
que se verifica para \(k=b\) hay que demostrar el mismo resultado para \(k+1\).

En el ejemplo que se plantea a continuación, se ve cómo se usa esta táctica:
al decir induction b with k h, se está diciendo que hacemos la inducción sobre
el número \(b\) denotando los casos por \(k=0\) y \(k+1\):

\begin{leancode}
example (t a b : ℕ) : t * (a + b) = t * a + t * b :=
begin
  induction b with k h,
  {sorry,},
  {sorry,},
end
\end{leancode}

\section{Táctica cases}

La táctica \dtactica{cases}{cases} es también una táctica muy usada en las
formalizaciones de Lean. Existen diversas formas de usar esta táctica, no
obstante, nos centraremos en las dos que se han usado en el trabajo y que
consisten en:
\begin{enumerate}
\item Cuando se tiene una hipótesis que es una conjunción, se usa la
  táctica \dtactica{cases}{cases} para separar sus componentes. Es decir,
  para la regla de eliminación de la conjunción.

\item Cuando se tiene una hipótesis que afirma que existe un número tal que
  verifica una determinada condición, se usa la táctica \dtactica{cases}{cases}
  para por un lado tener el número en cuestión y por otro la condición que él
  verifica. Es decir, para la regla de eliminación del existencial.

\item Cuando se tiene una hipótesis que es una disyunción, se usa la táctica
  \dtactica{cases}{cases} para dividir el problema en dos: probar el objetivo
  suponiendo la primera parte de la disyunción y probarlo suponiendo que
  se verifica la segunda parte. Es decir, para la regla de eliminación
  de la disyunción.
\end{enumerate}

En el siguiente ejemplo que se presenta, se puede ver cómo funciona la táctica
\dtactica{cases}{cases} cuando se usa para la regla de eliminación de la
conjunción:

\begin{leancode}
example {a b : ℝ} : (0 ≤ a ∧ 0 ≤ b) → 0 ≤ a + b :=
begin
  intros hyp,
  cases hyp with ha hb,
  exact add_nonneg ha hb,
end
\end{leancode}

En el segundo ejemplo, se puede observar la táctica \dtactica{cases}{cases}
usada para la regla de eliminación del existencial:
\begin{leancode}
example (n : ℕ) (h : ∃ k : ℕ, n = k + 1) : n > 0 :=
begin
  cases h with k₀ hk₀,
  rw hk₀,
  exact nat.succ_pos k₀,
end
\end{leancode}

En este último ejemplo, se presenta la táctica \dtactica{cases}{cases} para la
regla de eliminación de la disyunción:

\begin{leancode}
example (P Q : Prop): P ∨ Q → Q ∨ P :=
begin
  intro hpq,
  cases hpq with hp hq,
  {sorry,},
  {sorry,},
end
\end{leancode}

\section{Táctica split}

El caso de la táctica \dtactica{split}{split} es bastante intuitivo: esta táctica
divide cuando hay que demostrar un si y solamente si en las dos implicaciones
como objetivos separados o también separa en dos objetivos cuando hay que
demostrar una conjunción. Es decir, incluye las reglas de introducción
del bicondicional y de la conjunción.

En este primer ejemplo, se puede observar como separa el si y solamente si en
dos objetivos separados:

\begin{leancode}
example (P Q R : Prop) : (P ∧ Q → R) ↔ (P → (Q → R)) :=
begin
  split,
  { intros h1 h2 h3,
    exact h1 ⟨ h2,h3⟩ },
  { intros h1 h2,
    cases h2 with p q,
    exact h1 p q}
end
\end{leancode}

Mientras que un ejemplo muy simple en el que la táctica \dtactica{split}{split}
es usada para dividir una conjunción es el siguiente:
\begin{leancode}
example (P Q : Prop) (hp: P) (hq : Q) : P ∧ Q :=
begin
  split,
  { exact hp,},
  { exact hq,},
end
\end{leancode}

\section{Táctica from}

La táctica que se va a presentar es la conocida como \dtactica{from}{from}, esta
táctica es muy similar a la táctica \dtactica{exact}{exact}; pero a diferencia
de esta última, la táctica \dtactica{from}{from} se puede usar con otras tácticas
como \dtactica{have}.

Veamos un ejemplo del uso de la táctica \dtactica{from}{from} para probar una
hipótesis que ha sido introducida con la táctica \dtactica{have}{have}:

\begin{leancode}
def up_bounds (A : set ℝ) := { x : ℝ | ∀ a ∈ A, a ≤ x}
def is_max (a : ℝ) (A : set ℝ) := a ∈ A ∧ a ∈ up_bounds A
infix ` is_a_max_of `:55 := is_max

example (A : set ℝ) (x y : ℝ) (hx : x is_a_max_of A) (hy : y is_a_max_of A) :
x = y :=
begin
  have : x ≤ y, from hy.2 x hx.1,
  have : y ≤ x, from hx.2 y hy.1,
  linarith,
end
\end{leancode}

\section{Tácticas left y right}

La táctica \dtactica{left / right}{left} se usa cuando se quiere quedar con la
primera parte del objetivo a demostrar que está formada por dos constructores.
De manera totalmente análoga, la táctica \dtactica{left / right}{right} se queda
con la segunda parte del objetivo cuando está formada por dos
constructores. Esta táctica incluyes las reglas de introducción de la
disyunción.

Continuando la formalización de uno de los ejemplos que se ha presentado en la
explicación de la táctica \dtactica{cases}{cases} (en concreto el usado para
ilustrar la regla de eliminación de la disyunción), se puede ver muy bien el
uso de las tácticas \dtactica{left / right}{left} y \dtactica{left / right}{right}:
\begin{leancode}
example (P Q : Prop): P ∨ Q → Q ∨ P :=
begin
  intro hpq,
  cases hpq with hp hq,
  { right,
    exact hp,},
  { left,
    exact hq,},
end
\end{leancode}

\section{Táctica library search}

La táctica \dtactica{library_search}{library\_search} es una táctica que
para llevar a cabo formalizaciones de pruebas en Lean es muy útil. Esta
táctica se suele utilizar cuando estamos cerca de concluir la
demostración y no sabemos qué lema auxiliar utilizar o si existe alguno
o no, \dtactica{library_search} {library\_search} nos da esta
información. En el caso de que haya un lema mediante el cual se pueda
probar el resultado o una prueba medio directa, esta táctica nos la
proporciona; mientras que en el caso de que no sea capaz de encontrarlo
pues no nos proporciona ninguna respuesta.

Por ejemplo, en uno de los ejemplos propuestos analizar la táctica
\dtactica{intro / intros}{intro}, se puede ver muy bien el uso de esta
táctica:

\begin{leancode}
example (a b : ℝ): 0 ≤ a → b ≤ a + b :=
begin
  library_search,
end
\end{leancode}

Si hacemos caso de la propuesta de formalización que nos propone nos quedaría:

\begin{leancode}
example (a b : ℝ): 0 ≤ a → b ≤ a + b :=
begin
  exact le_add_of_nonneg_left,
end
\end{leancode}

Se puede observar que es una formalización diferente a la propuesta en
el ejemplo de la táctica \dtactica{intro / intros}{intro} e incluso me
atrevería a decir que más elegante.

\section{Táctica norm num}

La táctica \dtactica{norm_num}{norm\_num} es capaz de probar de manera
directa multitud de igualdades y desigualdades.

Por ejemplo, la siguiente igualdad la prueba en un solo paso:

\begin{leancode}
example : (2 : ℝ) + 2 = 4 :=
begin
  norm_num,
end
\end{leancode}

En el ejemplo que se presena ahora, la táctica \dtactica{norm_num}{norm\_num} es
capaz de probar la siguiente desigualdad:
\begin{leancode}
example : (73 : ℝ) < 789/2 :=
begin
  norm_num,
end
\end{leancode}

\section{Táctica refine}

La táctica \dtactica{refine}{refine} funciona de manera totalmente
análoga a como lo hace la táctica \dtactica{exact}{exact}; no obstante,
en esta nueva táctica se puede escribir \(\_\) como huecos que pueden
ser rellenados. Tras el uso de esta táctica, se pasará a tener tantos
objetivos como huecos se hayan puesto.

Veamos un ejemplo de esto:

\begin{leancode}
example (p : ℕ) (h: p.prime) : 1 ∣ p :=
begin
  refine is_unit.dvd _,
  exact is_unit_one,
end
\end{leancode}

\section{Táctica ring}

La táctica \dtactica{ring}{ring} en Lean es una táctica muy útil puesto que es
capaz de resolver las ecuaciones cuando se está trabajando con anillos
(semi) conmutativos.

Por ejemplo, cuando nos encontramos ante situaciones \textit{triviales}, se
pueden demostrar directamente mediante el uso de esta táctica. El siguiente
ejemplo es muy representativo:

\begin{leancode}
example (a b : ℝ) : (a + b) + a = 2*a + b :=
begin
  by ring,
end
\end{leancode}

\section{Táctica simp}

La táctica \dtactica{simp}{simp} lo que hace es simplificar el objetivo a probar
haciendo uso de lemas ya definidos en Lean. En el ejemplo planteado se ve muy
claro:
\begin{leancode}
example (a b : ℝ) (h: b=0): a + b = a :=
begin
  simp,
  exact h,
end
\end{leancode}

Mediante el uso de la táctica \dtactica{simp}{simp} se ha simplificado el
objetivo a demostrar mediante el uso del lema auxiliar de cancelación. De esta
manera, el objetivo a demostrar pasa a ser que el número \(b\) es cero.

\section{Táctica simpa}

La táctica \dtactica{simpa}{simpa} es muy parecida a la táctica \dtactica{simp}
{simp}; no obstante, la que se introduce ahora se trata de una táctica de
conclusión de los resultados.

El mismo ejemplo que se ha propuesto para la táctica \dtactica{simp}{simp} es
válido para la táctica \dtactica{simpa}{simpa} y es más, con esta última se
concluye el resultado de manera directa. Veámoslo:

\begin{leancode}
example (a b : ℝ) (h : b=0): a + b = a :=
begin
  simpa,
end
\end{leancode}

\section{Táctica suggest}

En general, cuando la táctica \dtactica{library_search}{library\_search} no es
capaz de solucionarlos el problema y no sabemos cómo continuar con la
formalización en Lean del resultado, se suele usar la táctica \dtactica{suggest}
{suggest}. Cuando se usa esta táctica en la formalización de un resultado, se
nos proporciona una lista de posibles vías de continuar la formalización. En
general, la mayoría de las posibilidades propuestas hacen uso de la táctica
\dtactica{refine}{refine} ya descrita.

Veamos el uso de esta táctica con un ejemplo:

\begin{leancode}
example (a b : ℝ) : a + b ≤ a + b +1 :=
begin
  suggest,
  sorry,
end
\end{leancode}

Destacar que la táctica \dtactica{suggest}{suggest} no es capaz de probar el
resultado, de ahí que se añada la táctica \dtactica{sorry}{sorry}. Algunas de
las propuestas que se nos hace en el ejemplo propuesto son las siguientes:

\begin{leancode}
Try this: exact (a + b).le_succ
Try this: exact nat.le.intro rfl
Try this: refine eq.ge _
Try this: refine eq.le _
Try this: refine ge.le _
Try this: refine not_lt.mp _
Try this: refine le_of_eq _
Try this: refine ge_of_eq _
Try this: refine le_of_lt _
Try this: refine ge_iff_le.mp _
Try this: refine ge_iff_le.mpr _
Try this: refine le_add_left _
Try this: refine sup_eq_left.mp _
...
\end{leancode}

Es más, usando cualquiera de las dos primeras propuestas se tendría demostrado
el problema. Aquí planteamos las dos:

\begin{leancode}
example (a b : ℕ) : a + b ≤ a + b +1 :=
begin
  exact (a + b).le_succ,
end

example (a b : ℕ) : a + b ≤ a + b +1 :=
begin
  exact nat.le.intro rfl,
end
\end{leancode}


\section{Táctica specialize}

La táctica \dtactica{specialize}{specialize} actúa sobre una hipótesis
determinada y su función consiste en concretar los términos que en la
hipótesis son generales o universales por los que el usuario
imponga. Esta táctica se corresponde con la regla de eliminación del
cuantificador universal.

Para ver mejor cómo funciona, planteemos un ejemplo:
\begin{leancode}
lemma unique_max
  (A : set ℝ)
  (x y : ℝ)
  (hx : x is_a_max_of A)
  (hy : y is_a_max_of A) :
  x = y :=
begin
  cases hx with x_in x_up,
  cases hy with y_in y_up,
  specialize x_up y,
  specialize x_up y_in,
  specialize y_up x x_in,
  linarith,
end
\end{leancode}

Se puede observar que en el primer uso de la táctica
\dtactica{specialize}{specialize}, se ha particularizado la hipótesis x\(\_\)up
para el número real \(y\) considerado.

De manera totalmente análoga se tiene en los dos siguientes usos de la táctica
\dtactica{specialize}{specialize}.

\section{Táctica push neg}

La táctica \dtactica{push_neg}{push\_neg} puede ser aplicada sobre una
hipótesis o sobre el propio objetivo a demostrar. Básicamente, lo que
hace esta táctica es que cuando se tiene la negación de un cierto hecho,
lo convierte en un hecho interiorizando la negación pero con el mismo
significado y manteniendo el nombre de las variables.

\begin{leancode}
example {x y : ℝ} : (∀ ε > 0, y ≤ x + ε) →  y ≤ x :=
begin
  intro h,
  by_contradiction H,
  push_neg at H,
  sorry,
end
\end{leancode}

Se puede observar que la hipótesis H es la negación de que \(y\) sea menor o
igual que \(x\). Tras la aplicación de la táctica \dtactica{push_neg}{push\_neg},
esta hipótesis se convierte en considerar que \(x\) es menor estrictamente que
\(y\).

\section{Táctica congr'}

La última táctica que se va a detallar en este apéndice es la táctica
\dtactica{congr'}{congr'}, la cual sólo ha sido utilizada una vez en el
desarrollo del trabajo. La táctica \dtactica{congr'}{congr'} es una
táctica muy parecida a \dtactica{congr}{congr} pero menos agresiva que
esta última.

Estas tácticas intentan demostrar el objetivo cuando se trata de una
igualdad devolviendo unos determinados subojetivos a probar. La táctica
\dtactica{congr'}{congr'} hace uso de un argumento opcional que es el de
la recursividad de las funciones.

Veamos un ejemplo de la táctica \dtactica{congr'}{congr'}:

\begin{leancode}
variables (f g: ℝ → ℝ)
example {x y : ℝ} : f (g (x + y)) = f (g (y + x)):=
begin
  congr',
  {sorry,},
  {sorry,},
end
\end{leancode}

Se puede observar que tras aplicar la táctica \dtactica{congr'}{congr'} el
objetivo a probar pasa a dividirse en dos: por un lado, probar que el número
\(x\) es igual a \(y\) y por otro que \(y\) es igual a \(x\).

\section{Otras tácticas}

En este apéndice se han detallado todas las tácticas que se han utilizado en
el desarrollo del trabajo; no obstante, existen muchas más tácticas que pueden
ser utilizadas en las formalizaciones en Lean. En \cite{tactic} es un buen
lugar para ver y estudiar las tácticas que se pueden usar en Lean.
