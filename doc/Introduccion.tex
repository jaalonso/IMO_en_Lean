\chapter*{Introducción}
\addcontentsline{toc}{chapter}{Introducción}

Las \href{https://www.imo-official.org/}{Olimpiadas Internacionales de
  Matemáticas (IMO)}\footnote{\url{https://www.imo-official.org}} son
probablemente la competición de ejercicios de inteligencia con mayor
número de celebraciones en el mundo y es por eso que se encuentra entre
los últimos grandes objetivos de la Inteligencia Artificial.

En el proyecto
\href{https://imo-grand-challenge.github.io/}{IMO Grand Challenge}
se plantea construir un sistema de Inteligencia Artificial que sea capaz
de ganar la medalla de oro en esta competición.

Con el objetivo de eliminar la ambigüedad que existe a la hora de comprobar si
las diversas soluciones propuestas a los ejercicios son correctas o no y qué
puntación hay que concederle a cada una, se plantea la variante de las IMO
que se conoce como formal--to--formal (F2F), \cite{challenge}. Básicamente, esta
variante consiste en que recibe el enunciado y la solución al problema, ambos
formalizados en Lean y comprueba si la solución proporcionada es correcta o no
según las reglas estipuladas.

Lean es un demostrador interactivo, mediante el cual se pueden comprobar
si los resultados o demostraciones que se proponen a un problema son
correctas. Esta herramienta es muy útil pues hay situaciones en las que
a los humanos se nos puede olvidar comentar un aspecto trivial de la
demostración pero el cual es indispensable para la correcta
compresión. Asimismo, hay situaciones en las que por motivos de
extensión o complejidad se hace casi imposible poder realizar la
comprobación correspondiente. En la actualidad, los demostradores
interactivos son un tema muy a la orden del día y que en los últimos
tiempos se ha convertido en una herramienta muy importante, como se
comenta en \cite{ART2}.

De esta manera, se tiene que el reto que se quiere conseguir consiste en
crear un sistema inteligente que sea capaz de obtener la puntuación en
F2F necesaria para alcanzar la medalla de oro en la competición, en el
caso de competir contra humanos.

Como consecuencia de todo esto, actualmente se están desarrollando
multitud de trabajos y proyectos que se basan en la formalización de
diversos problemas (no sólo de las IMO, también de otras competiciones)
con ayuda de diferentes demostradores interactivos como Lean o
Isabelle/HOL.

El proyecto MiniF2F, \cite{mini}, ha comenzado a desarrollarse hace muy
poco tiempo y consiste en formalizar problemas correspondientes a
diferentes competiciones como las IMO pero también otras como American
Mathematics Competitions (AMC). Asimismo, se propone la formalización de
problemas de matemáticas de nivel escolar. En este proyecto las
formalizaciones que se proponen son a través de varias herramientas,
entre ellas Lean.

En el trabajo que aquí se propone, se estudia detalladamente diversas
formalizaciones en Lean de problemas matemáticos correspondientes a las
IMO. Para llevar a cabo este trabajo nos hemos basado en el artículo
\cite{ART} en el que estudiantes de la Universidad de Belgrado
estudiaron de manera teórica y formalizaron en el programa Isabelle/HOL
problemas de las IMO.

El trabajo se ha comenzando con una \textbf{introducción a Lean},
(\ref{cap1}), donde se ha descrito brevemente esta herramienta y también
su librería matemática \textbf{mathlib}. Asimismo, en este capítulo se
desarrolla la \textbf{formalización de una prueba} de manera muy
detallada: explicando y justificando cada paso de la formalización, para
que de esta manera se pueda comprender la estructura de las
formalizaciones.

A continuación, en el capítulo llamado \textbf{elementos de matemáticas
  en Lean}, (\ref{cap2}), se plantean varias formalizaciones sobre lemas
o teoremas que han sido estudiados en asignaturas de formación básica
correspondientes al Grado en Matemáticas de la Universidad de Sevilla.
En concreto, los resultados son de las asignaturas de Cálculo
Infinitesimal y Álgebra Básica. Estas formalizaciones fueron obtenidas
con ayuda del tutorial \cite{tutor} que realicé junto con
\href{https://www.ma.imperial.ac.uk/~buzzard/xena/natural_number_game/}{The
  Natural Number Game}\footnote{The Natural Number Game es un curso en
  el que se explica la teoría de los números naturales diseñada a modo
  de juego de manera que a medida que avanzas en él se introducen más
  técnicas de demostración que se deben ir utilizando.}.

El siguiente capítulo es el más extenso del trabajo y es en el que se
detallan varias \textbf{formalizaciones en Lean de problemas de las IMO}
que ya han sido formalizados con anterioridad (\ref{cap3}). Además de la
correspondiente formalización en Lean, en cada problema se plantea su
resolución en lenguaje natural para que el lector pueda entender y ver
la analogía entre ambas. En concreto se han estudiado los siguientes
problemas que se presentan a continuación:
\begin{itemize}
\item \textbf{IMO 1959 Q1}

  En este problema hay que demostrar que una cierta fracción es irreducible
  para cualquier número natural. La formalización propuesta en la que se ha
  basado la que se plantea aquí en el trabajo es \cite{KL}.

\item \textbf{IMO 1962 Q4}

  En este caso, el problema que se plantea es resolver una ecuación en la que
  aparecen funciones trigonométricas. La formalización en la que se ha basado
  la que se plantea en el trabajo es la pruepuesta en \cite{KLHM}.

\item \textbf{IMO 1977 Q6}

  Este problema consiste en probar que la función que se define en el enunciado
  en el conjunto de los números naturales bajo una ciertas condiciones, verifica
  una igualdad determinada. En este caso, la formalización que se ha usado
  para plantear la que aparece en el trabajo es \cite{TC}.


\item \textbf{IMO 2001 Q2}

  El último problema que se formaliza en esta sección del trabajo se corresponde
  con demostrar una desigualdad que se verefica para cualesquiera tres números
  reales. En este caso, la formalización en la que se ha basado la planteado
  en el trabajo es muy reciente (en concreto, ha sido realizada este año 2021)
  y se puede encontrar en \cite{TCC}.
\end{itemize}

\comentario{Cambiar la descripción de los 4 problemas anteriores por su
  enunciado en lenguaje natural y en Lean.}

Todos estos capítulos que se han desarrollado previamente, han permitido
adquirir los conocimientos y la destreza suficiente como para formalizar un
problema que no haya sido formalizado con anterioridad. Entonces, en el último
capítulo del trabajo se ha llevado a cabo la \textbf{
formalización en Lean de un nuevo problema que no se había formalizado antes},
(\ref{cap4}). En concreto, el problema que se ha formalizado en este capítulo es
el Q6 correspondiente al año 2001. Este problema consiste en que dado cuatros
números enteros verificando una serie de hipótesis, hay que probar que un
número que se define a través de esos cuatro números enteros es primo.

\comentario{Cambiar la descripción de los 4 problemas anteriores por su
  enunciado en lenguaje natural y en Lean.}

Finalmente, en el apéndice \ref{apentacti} se han explicado todas
las tácticas de Lean que se han usado en el trabajo.

\comentario{Dejar los siguientes párrafos para la presentación del TFM.}

% Finalmente, cabe mencionar que la realización de este trabajo ha supuesto el
% aprendizaje del programa Lean y el uso de su librería mathlib. Asimismo,
% comentar también que en el apéndice \ref{apentacti} se han explicado todas
% las tácticas de Lean que se han usado en el trabajo.
%
% Además, el objetivo pincipal del trabajo ha sido alcanzado: obtener la destreza
% suficiente como para ser capaz de formalizar un problema que no se había
% formalizado con anterioridad.
%
% Por último, comentar que es un trabajo que tiene muchas líneas futuras de
% continuación pues aún quedan multitud de problemas de las IMO que no han sido
% formalizados. Es más, existen otras muchas competiciones o problemas que
% también se pueden formalizar mediante demostradores interactivos.

%%% Local Variables:
%%% mode: latex
%%% TeX-master: "IMO_en_Lean"
%%% End:
