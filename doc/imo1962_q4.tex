\section{IMO 1962 Q4}

En esta sección se va a detallar la solución al problema Q4
correspondiente al año 1962. Realizaremos la demostración en lenguaje
natural del problema y a su vez se presentará la correspondiente
formalización en Lean.

La formalización que aquí se presenta ha sido inspirada en
las que se proponen en \cite{KLHM} realizadas por Kevin Lacker y
Healther Macbeth.

\noindent
\framebox{\parbox{\textwidth}{
  \textbf{Problema 2 (1962--Q4)}. Resolver la ecuación
  \[\cos²(x)+\cos²(2x)+\cos²(3x)=1\].}}

Para llevar a cabo la formalización en Lean de este problema se va a
importar la teoría
\href{https://github.com/leanprover-community/mathlib/blob/master/src/analysis/special_functions/trigonometric.lean}{analysis.special\_functions.trigonometric}
(sobre funciones trigonométicas) y habilitar el
\href{https://leanprover.github.io/reference/other_commands.html#namespaces}{espacio
de nombre} de los números reales (para simplificar la notación, por
ejemplo escribiendo \texttt{cos} en lugar de \texttt{real.cos}). Además,
se indica con \texttt{noncomputable theory} que se van a usar funciones
no computables (es decir, funciones para las que Lean no generará código
evaluable).

Todo lo anterior se expresa en Lean por
\begin{leancode}
import analysis.special_functions.trigonometric

open real
open_locale real
noncomputable theory
\end{leancode}

Previo a la resolución como tal del problema, se van a introducir
una serie de definiciones que nos servirán para llevar una notación
más compacta:

\begin{definicion}\label{problema}
  De aquí en adelante usaremos la etiqueta \textbf{problema}
  para referirnos a la expresión que queremos resolver,
  es decir,
  \begin{equation}\label{expresionprob}
    \cos²(x)+\cos²(2x)+\cos²(3x)=1.
  \end{equation}
\end{definicion}

De manera que la formalización en Lean de la definición
(\ref{problema}) es:
\begin{leancode}
def problema (x : ℝ) : Prop :=
cos x ^ 2 + cos (2 * x) ^ 2 + cos (3 * x) ^ 2 = 1
\end{leancode}
\index{problema}

\begin{definicion}\label{funaux}
  De aquí en adelante usaremos la etiqueta \textbf{funAuxiliar} para
  referirnos a la siguiente expresión:
  \begin{equation*}
    \cos(x)·\left(\cos²(x)-\frac{1}{2}\right)·\cos(3x).
  \end{equation*}
\end{definicion}

La formalización en Lean en este caso sería:
\begin{leancode}
def funauxiliar (x : ℝ) : ℝ :=
cos x * (cos x ^ 2 - 1/2) * cos (3 * x)
\end{leancode}
\index{funauxiliar}

\subsection{Equivalencia de problemas}

Una vez introducido estas dos definiciones, se va a proceder a demostrar
que resolver problema es equivalente a resolver la expresión obtenida al
igualar a cero la funAuxiliar.

Esta demostración se detallará a continuación, pero previo a ella se
necesita de la demostración de un lema auxiliar:

\begin{lema}[Igualdad]\label{igualdadlema}
  Para cualquier \(x\) perteneciente al conjunto de los números reales,
  se verifica la siguiente igualdad:
  \begin{equation}\label{lemaigualdad}
    \frac{\cos²(x)+\cos²(2x)+\cos²(3x)-1}{4}=\funauxiliar(x).
  \end{equation}
\end{lema}

\begin{demostracion}
  En primer lugar, según la definición vista en \ref{funaux}, se
  tendría que la expresión (\ref{lemaigualdad}) se convierte en:
  \begin{equation}\label{igualdad}
    \frac{\cos²(x)+\cos²(2x)+\cos²(3x)-1}{4}=
    \cos(x)·\left(\cos²(x)-\frac{1}{2}\right)·\cos(3x).
  \end{equation}

  Ahora bien, para llevar a cabo la demostración comenzaremos
  introduciendo las conocidas definiciones del coseno del ángulo doble y
  del coseno del ángulo triple, que son:
  \begin{align}
    & \cos(2x)=2\cos²(x)-1\label{cos2}\\
    & \cos(3x)=4\cos³(x)-3\cos(x)\label{cos3},
  \end{align}
  donde estas dos propiedades en Lean se formalizan con los dos
  siguientes lemas:
  \begin{itemize}
\item \mint{lean}|cos_two_mul : cos (2 * x) = 2 * cos x ^ 2 - 1|
  \indLema{cos\_two\_mul}
\item \mint{lean}|cos_three_mul : cos (3 * x) = 4 * cos x ^ 3 - 3 * cos x|
  \indLema{cos\_three\_mul}
\end{itemize}

A continuación, se van a desarrollar los dos términos de la igualdad
(\ref{igualdad}). Comencemos por el primer término de los dos:
\begin{align}\label{term1}
  \frac{\cos²(x)+\cos²(2x)+\cos²(3x)-1}{4}&=\frac{\cos²(x)+(2\cos²(x)-1)²}{4}\\
  &+\frac{(4\cos³(x)-3\cos(x))²-1}{4},
\end{align}
donde simplemente se han introducido las definiciones (\ref{cos2})
y (\ref{cos3}).

Desarrollando los cuadrados que aparecen en (\ref{term1}) y
simplificando los términos, se acaba obteniendo que:
\begin{equation}\label{term11}
  \frac{\cos²(x)+\cos²(2x)+\cos²(3x)-1}{4}=
  \cos(x)\left(\frac{3}{2}\cos(x)+4\cos⁵(x)-5\cos³(x)\right)
\end{equation}

Por otro lado, desarrollando de manera totalmente análoga el
segundo término de la igualdad (\ref{igualdad}), se tiene que:
\begin{align}
  \cos(x)·\left(\cos²(x)-\frac{1}{2}\right)·\cos(3x)
                                         &\stackrel{(*)}{=}\cos(x)·
                                           \left(\cos²(x)-\frac{1}{2}\right)·
                                           (4\cos³(x)-3\cos(x)
                                           )\label{term2}\\
                                         &=\cos(x)·
                                           \left(4\cos⁵(x)-5\cos³(x)
                                           +\frac{3}{2}\cos(x)\right)
                                           \label{term21},
\end{align}
donde en (\ref{term2}) se ha hecho uso de las definiciones (\ref{cos2})
y (\ref{cos3}).

De manera que observando las expresiones (\ref{term11}) y
(\ref{term21}), se puede concluir que la igualdad planteada en
(\ref{igualdad}) es cierta.  De esta forma, ya se tendría el lema
demostrado.
\end{demostracion}

Veamos la formalización en Lean de este lema:
\begin{leancode}
lemma Igualdad {x : ℝ} :
  (cos x ^ 2 + cos (2 * x) ^ 2 + cos (3 * x) ^ 2 - 1) / 4 = funauxiliar x :=
begin
  rw funauxiliar,
  rw real.cos_two_mul,
  rw cos_three_mul,
  ring_nf,
end
\end{leancode}
\index{Igualdad}

En la prueba del lema se han usado los siguientes lemas:
\begin{itemize}
\item \mint{lean}|cos_three_mul : cos (3 * x) = 4 * cos x ^ 3 - 3 * cos x|
  \indLema{cos\_three\_mul}
\item \mint{lean}|cos_two_mul : cos (2 * x) = 2 * cos x ^ 2 - 1 |
  \indLema{cos\_two\_mul}
\end{itemize}
y las tácticas
\tactica{ring}{ring} y
\tactica{rw / rewrite}{rw}.

Una vez se ha introducido este lema, se puede proceder a la demostración
del lema que ya se adelantó y que consiste en probar que resolver la
expresión \textbf{problema}, (\ref{expresionprob}), es equivalente a
resolver la expresión obtenida de igualar a cero \textbf{funauxiliar},
(\ref{funaux}). Veámoslo:

\begin{lema}[Equivalencia]\label{lemaequivalenciaprob}
  Resolver la expresión (\ref{expresionprob}) es equivalente a resolver
  la expresión
  \begin{equation}\label{probaux}
    \funauxiliar(x) = 0.
  \end{equation}
\end{lema}

\begin{demostracion}
  Esta demostración se trata de una doble implicación, por ello,
  separaremos las implicaciones:

  \noindent
  \framebox{\longrightarrow}
  En primer lugar, comenzamos considerando el problema planteado
  por la siguiente expresión:
  \begin{equation}\label{h1b}\tag{h1}
    \cos²(x) + \cos²(2x) + \cos²(3x) = 1.
  \end{equation}

  Queremos demostrar que este problema planteado es equivalente a
  considerar el problema planteado en (\ref{probaux}).

  Aplicando el Lema \ref{igualdadlema} sobre (\ref{probaux}), se tiene
  que el objetivo a demostrar pasa a ser que el problema (\ref{h1b}) es
  equivalente al problema:
  \begin{equation}\label{h11}
    \frac{\cos²(x) + \cos²(2x) + \cos²(3x) - 1}{4} = 0.
  \end{equation}

  Observando esta expresión, donde tenemos una fracción igualada a cero,
  se concluye que el numerador ha de ser nulo. Entonces, se tiene que
  demostrar (\ref{h11}) es equivalente a demostrar que
  \begin{equation}
    \cos²(x) + \cos²(2x) + \cos²(3x) - 1 = 0.
  \end{equation}

  De esta forma ya se ha teminado esta implicación pues es exactamente
  la hipótesis (\ref{h1b}), basta con pasar el término unidad al lado
  derecho de la igualdad.

  \noindent
  \framebox{\longleftarrow} A continuación, supongamos que se tiene
  el problema
  \begin{equation}\label{h2b}\tag{h2}
    \funauxiliar(x) = 0.
  \end{equation}
  Queremos demostrar que resolver este problema es equivalente a
  resolver el problema planteado en (\ref{expresionprob}).

  En primer lugar, aplicamos el Lema \ref{igualdadlema} sobre la
  hipótesis (\ref{h2b}). De manera que se obtiene que la hipótesis
  se transforma en la siguiente:
  \begin{equation}
    \frac{\cos²(x) + \cos²(2x) + \cos²(3x) - 1}{4} = 0.
  \end{equation}

  Como se tiene una división igualada a cero, el numerador ha de ser
  cero. De esta forma, se verifica que:
  \begin{equation}
    \cos²(x) + \cos²(2x) + \cos²(3x) - 1 =0
  \end{equation}

  Para concluir esta implicación basta con pasar el término uno
  hacia el otro lado de la igualdad. De esta forma se obtiene que
  \begin{equation}
    \cos²(x) + \cos²(2x) + \cos²(3x) = 1,
  \end{equation}
  que era el resultado que deseábamos.
\end{demostracion}

La formalización en Lean de este Lema es la siguiente:
\begin{leancode}
lemma Equivalencia
  {x : ℝ}
  : problema x ↔ funauxiliar x = 0 :=
begin
  split,
  { intro h1,
    rw problema at h1,
    rw ← Igualdad,
    rw div_eq_zero_iff,
    norm_num,
    rw sub_eq_zero,
    exact h1, },
  { intro h2,
    rw problema,
    rw ← Igualdad at h2,
    rw div_eq_zero_iff at h2,
    norm_num at h2,
    rw sub_eq_zero at h2,
    exact h2, },
end
\end{leancode}
\index{Equivalencia}

En la prueba del lema se han usado los siguientes lemas:
\begin{itemize}
\item \mint{lean}|div_eq_zero_iff : a / b = 0 ↔ a = 0 ∨ b = 0|
  \indLema{div\_eq\_zero\_iff}
\item \mint{lean}|sub_eq_zero : a - b = 0 ↔ a = b |
  \indLema{sub\_eq\_zero}
\end{itemize}
y las tácticas
\tactica{exact}{exact},
\tactica{norm_num}{norm\_num} y
\tactica{rw / rewrite}{rw}.

En el caso de la prueba de nuestro lema, esta táctica ha sido usada para
demostrar que cuatro es distinto de cero. Esta táctica prueba de manera
directa este tipo de igualdades y desigualdades.

\subsection{Resolución del problema equivalente}\label{secequiv}

Una vez ya se ha demostrado que el problema (\ref{probaux}) es
equivalente al problema que inicialmente queríamos demostrar, se va a
proceder a encontrar los ceros de (\ref{probaux}). Para ello, se
introducirán tres lemas auxiliares. El primero de ellos consistirá en
ver para en qué dos casos se verifica el problema (\ref{probaux});
mientras que en los dos últimos lemas se detallará la forma de la
solución para los dos casos que se obtendrán en el primer lema.

\begin{lema}[CasosSolucion]\label{lemaCasosSolucion}
  El problema (\ref{probaux}) se verifica
  si y solamente si o bien el coseno al cuadrado de \(x\) es igual
  a un medio, o bien, el coseno del triple de \(x\) es nulo.
  Simbólicamente:
  \begin{equation}\label{problema1}
    \funauxiliar(x) = 0 ⟺ \cos²(x) = \frac{1}{2} ∨ \cos(3x) = 0.
  \end{equation}
\end{lema}

\begin{demostracion}
  En primer lugar, previo a la demostración en sí del problema
  (\ref{problema1}), se va a proceder a la reescritura del mismo.
  Para ello, lo que hacemos es escribir la definición de
  funAuxiliar vista en la definición \ref{funaux}, es decir,
  \begin{equation}\label{fun1}
    \funauxiliar(x) = \cos(x)·\left(\cos²(x)-\frac{1}{2}\right)·\cos(3x).
  \end{equation}

  Aplicando ahora la propiedad asociativa de la multiplicaión,
  se tiene que (\ref{fun1}) es equivalente a
  \begin{equation}\label{fun2}
    \cos(x)·\left(\left(\cos²(x)-\frac{1}{2}\right)·\cos(3x)\right).
  \end{equation}

  De esta forma como en el problema inicial se tiene que funAuxiliar
  está igualada a cero; se sabe que cuando un producto está igualado
  a cero uno de los dos términos ha de ser nulo. En el caso que a
  nosotros nos incumbe, esto se reescribe de la siguiente manera:
  \begin{align}
    \cos(x)·\left(\left(\cos²(x)-\frac{1}{2}\right)·\cos(3x)\right)=0
    &⟺ \cos(x) = 0 ∨ \left(\cos²(x)-\frac{1}{2}\right)·\cos(3x)=0 \\
    &⟺ \cos(x) = 0 ∨ \cos²(x)=\frac{1}{2} ∨ \cos(3x)=0,
  \end{align}
  donde hemos aplicado dos veces el resultado de que el producto
  de dos términos esté igualado a cero.

  De esta forma, tenemos que el problema inicial, (\ref{problema}),
  es equivalente al que se plantea a continuación:
  \begin{equation}\label{problemaEquiv2}
    \cos(x)=0 \lor \cos²(x)=\frac{1}{2} \lor \cos(3x)=0
    ⟺ \cos²(x)=\frac{1}{2} \lor \cos(3x)=0.
  \end{equation}

  Ahora ya sí procederemos a la demostración de
  (\ref{problemaEquiv2}); como se trata de un si y solamente si
  lo haremos por doble implicación.

  \noindent
  \framebox{\longrightarrow} Suponemos que se verifica que
  \begin{equation}\label{Casosh1}\tag{h1}
    \cos(x)=0 \lor \cos²(x)=\frac{1}{2} \lor \cos(3x)=0.
  \end{equation}

  Entonces tenemos que demostrar que bajo la suposición (\ref{h1b}),
  se verifica que
  \begin{equation}\label{CasosConc1}
     \cos²(x)=\frac{1}{2} \lor \cos(3x)=0.
  \end{equation}

  La hipótesis (\ref{h1b}) significa que alguna de las disyunciones
  ha de ser cierta. Por ello, considerar dicha hipótesis es
  equivalente a considerar dos posibles casos, que serían los
  siguientes:
  \begin{align}
    &\cos(x)=0 \label{Casosh11}\tag{h11}\\
    &\cos²(x)=\frac{1}{2}\lor \cos(3x)=0. \label{Casosh12}\tag{h12}
  \end{align}

  Esto supone que la demostración pasa a ser dividida en
  dos subproblemas: el primero de
  ellos consiste en demostrar (\ref{CasosConc1}) suponiendo
  cierta la hipótesis (\ref{Casosh11}) y el segundo de ellos
  es también demostrar (\ref{CasosConc1}) pero ahora bajo la
  hipótesis (\ref{Casosh12}).

  \begin{itemize}
  \item \textbf{Subproblema 1}

    Con el objetivo de demostrar (\ref{CasosConc1}) suponiendo
    cierta la hipótesis (\ref{Casosh11}), basta con demostrar una
    de las dos disyunciones. En este caso se demostrará la segunda
    de ellas, es decir, se demostrará que el coseno del triple de
    \(x\) es nulo.

    Para demostrar esto, basta con hacer uso de la definición del
    ángulo triple. De esta forma se tiene que:
    \begin{equation}
      \cos(3x)=4\cos³(x)-3\cos(x)\stackrel{(*)}{=}0,
    \end{equation}
    donde en \((*)\) se ha hecho uso de la hipótesis
    (\ref{Casosh11}). Y ya se tendría el resultado.

  \item \textbf{Subproblema 2}

    En este caso suponemos cierta la hipótesis (\ref{Casosh12})
    y tenemos que demostrar (\ref{CasosConc1}), lo cual se tiene
    de manera directa pues son la misma afirmación.
  \end{itemize}

  \noindent
  \framebox{\longleftarrow} A continuación se tiene que demostrar
  la implicación contraria. Esto es, suponiendo cierta la
  hipótesis
  \begin{equation}\label{Casosh2}\tag{h2}
    \cos²(x)=\frac{1}{2} \lor \cos(3x)=0,
  \end{equation}
  hay que demostrar que es equivalente a
  \begin{equation}\label{CasosConc2}
    \cos(x)=0 \lor \cos²(x)=\frac{1}{2}\lor \cos(3x)=0.
  \end{equation}

  En este caso, la implicación es directa puesto que basta con
  quedarse con las dos disyunciones de la derecha en
  (\ref{CasosConc2}) que se corresponden con la propia hipótesis
  (\ref{Casosh2}).
\end{demostracion}

A continuación, se plantea la formalización en Lean de este lema:
\begin{leancode}
lemma CasosSolucion
  {x : ℝ}
  : funauxiliar x = 0 ↔ cos x ^ 2 = 1/2 ∨ cos (3 * x) = 0 :=
begin
  rw funauxiliar,
  rw mul_assoc,
  rw mul_eq_zero,
  rw mul_eq_zero,
  rw sub_eq_zero,
  split,
  { intro h1,
    cases h1 with h11 h12,
    right,
    rw cos_three_mul,
    rw h11,
    ring,
    exact h12,},
  { intro h2,
    right,
    exact h2,},
end
\end{leancode}
\index{CasosSolucion}

En la prueba del lema se han usado los siguientes lemas:
\begin{itemize}
\item \mint{lean}|cos_three_mul : cos (3 * x) = 4 * cos x ^ 3 - 3 * cos x|
  \indLema{cos\_three\_mul}
\item \mint{lean}|mul_assoc : (a * b) * c = a * (b * c) |
  \indLema{mul\_assoc}
\item \mint{lean}|mul_eq_zero : a * b = 0 ↔ a = 0 ∨ b = 0|
  \indLema{mul\_eq\_zero}
\item \mint{lean}|sub_eq_zero : a - b = 0 ↔ a = b|
  \indLema{sub\_eq\_zero}
\end{itemize}
y las tácticas
\tactica{exact}{exact},
\tactica{left / right}{left},
\tactica{left / right}{right},
\tactica{ring}{ring},
\tactica{rw / rewrite}{rw} y
\tactica{split}{split}.

\begin{lema}[SolucionCosenoCuadrado]\label{lemaCosenoCuadrado}
  La expresión coseno al cuadrado de \(x\) es igual a un medio si y
  solamente si existe un \(k\) perteneciente al conjunto de los números
  enteros tal que \(x\) es igual al producto de \(2k\) más uno por \(π\)
  cuarto. Simbólicamente:
  \begin{equation}\label{problema2}
    \cos²(x) = \frac{1}{2} ⟺ ∃ k ∈ ℤ, x = (2k+1)⋅\frac{π}{4}.
  \end{equation}
\end{lema}

\begin{demostracion}
  Se puede observar que se trata de una demostración de doble
  implicación. No obstante, previo a la demostración de las
  correspondientes implicaciones, reescribiremos el problema
  (\ref{problema2}). Para ello, comenzaremos usando la siguiente
  relación
  \begin{equation}
    \cos²(x) = \frac{1}{2} + \frac{\cos(2x)}{2}.
  \end{equation}

  De esta forma, el problema (\ref{problema2}), se puede reescribir como
  sigue:
  \begin{equation}
    \frac{1}{2} + \frac{\cos(2x)}{2} = \frac{1}{2} ⟺
    ∃ k ∈ ℤ, x = (2k + 1) ⋅ \frac{π}{4}.
  \end{equation}

  Se puede observar que en la primera parte de la equivalencia, la
  igualdad tiene el mismo término a derecha que a izquierda, de manera
  que se puede eliminar. Obtenemos que
  \begin{equation}\label{problema21}
    \frac{\cos(2x)}{2} = 0 ⟺ ∃ k ∈ ℤ, x = (2k+1) ⋅ \frac{π}{4}.
  \end{equation}

  Por último, en (\ref{problema21}), se tiene una fracción igualada a 0,
  esto significa que el numerador de dicha fracción ha de ser nulo. Por
  tanto, el problema que inicialmente planteamos, (\ref{problema2}),
  quedaría reescrito de la siguiente manera:
  \begin{equation}\label{problema22}
    \cos(2x) = 0  ⟺ ∃ k ∈ ℤ, x = (2k+1) ⋅ \frac{π}{4}.
  \end{equation}

  Procedamos ahora a la demostración por doble implicación del
  problema (\ref{problema22}).

  \noindent
  \framebox{\longrightarrow} Supongamos que se verifica que
  \begin{equation}\label{Dobleh2}\tag{h1}
    \cos(2x) = 0
  \end{equation}

  Entonces, se quiere demostrar que
  \begin{equation}\label{ConcDoble}
    ∃ k ∈ ℤ, x = (2k+1) ⋅ \frac{π}{4}.
  \end{equation}

  Para ello, comenzaremos usando la caracterización de la solución del
  coseno de un ángulo igualado a cero. Esto es:

  \begin{proposicion}\label{alpha}
    Sea \(θ\) un número perteneciente al conjunto de los
    números naturales. Entonces se tiene que se verifica la
    expresión coseno de \(θ\) igual a cero si y solamente si
    existe un número \(k\) perteneciente al conjunto de los
    números enteros tal que \(θ\) es igual al producdo del
    doble de \(k\) más uno por \(π\) medio. Simbólicamente:
  \begin{equation}\label{alpha2}
    \cos(θ) = 0 ⟺ ∃ k ∈ ℤ,  θ = \frac{(2k+1)π}{2}.
  \end{equation}
  \end{proposicion}

  La formalización en Lean de esta proposición es la planteada
  a continuación:
  \begin{leancode}
    theorem cos_eq_zero_iff
      {θ : ℝ}
      : cos θ = 0 ↔ ∃ k : ℤ, θ = (2 * k + 1) * π / 2
  \end{leancode}

  Entonces, usando la propiedad \ref{alpha} sobre nuestra hipótesis
  (\ref{Dobleh2}), se tiene que
  \begin{equation}\label{Doble2h11}
    ∃k ∈ ℤ, 2x = \frac{(2k+1)π}{2}.
  \end{equation}

  A partir de la hipótesis (\ref{Doble2h11}), se sabe que existe un
  número entero verificando dicha hipótesis. Sin pérdida de generalidad,
  se puede considerar que dicho número entero es un determinado número
  que denotaremos por \(k_1\) y de esta forma la hipótesis
  (\ref{Doble2h11}) se convierte en:
  \begin{equation}\label{Doblehk1}\tag{hk1}
     2x = \frac{(2k_1+1)π}{2}.
  \end{equation}

  Por tanto, para demostrar (\ref{ConcDoble}) basta con usar el mismo
  número entero, \(k_1\), y la hipótesis (\ref{Doblehk1}). Pasando el
  dos que se encuentra multiplicando en la parte izquierda de la
  igualdad ya se tendría lo que queremos demostrar.

  De manera que ya hemos terminado la primera implicación de la
  demostración.

   \noindent
   \framebox{\longleftarrow} Supongamos ahora que existe \(k\)
   perteneciente al conjunto de los números enteros, a partir del cual
   \(x\) se puede escribir como el producto de el doble de \(k \) más
   uno, por \( π\) medio. Simbólicamente,
   \begin{equation}\label{CosDobleh2}\tag{h2}
      ∃ k ∈ ℤ, x = \frac{(2k+1)π}{4}.
    \end{equation}

    Sin pérdida de generalidad, la hipótesis (\ref{CosDobleh2}) se puede
    reescribir considerarando que el número entero que la verifica es un
    determinado número entero que denotaremos por \(k_2\).  Entonces la
    hiótesis (\ref{CosDobleh2}) pasa a ser:
    \begin{equation}\label{hk2}\tag{hk2}
       x = \frac{(2k_2+1)π}{4}.
    \end{equation}

    Entonces se quiere demostrar que se verifica la siguiente expresión:
    \begin{equation}\label{ConcDoble3}
      \cos(2x) = 0.
    \end{equation}

    Análogamente a como hicimos en la implicación anterior,
    sustituiremos la expresión del coseno igualada a cero por la
    descrita en la Proposición \ref{alpha}. Entonces, si se lo aplicamos
    a (\ref{ConcDoble3}), se obtiene que el objetivo a probar pasa a ser
    que exista un número entero \(k\) tal que el ángulo \(2x\) se pueda
    escribir como se describió en (\ref{alpha2}). Simbólicamente:
    \begin{equation}
      ∃ k∈ ℤ, 2x = \frac{(2k+1)π}{2}.
    \end{equation}

    Finalmante, se puede concluir la demostración haciendo uso del
    número entero \(k_2\) y de la hipótesis (\ref{hk2}).
\end{demostracion}

La formalización en Lean del lema anteriormente desarrollado es
la que se presenta a continuación:
\begin{leancode}
lemma SolucionCosenoCuadrado
  {x : ℝ}
  : cos x ^ 2 = 1/2 ↔ ∃ k : ℤ, x = (2 * k + 1) * π / 4 :=
begin
  rw cos_sq,
  rw add_right_eq_self,
  rw div_eq_zero_iff,
  norm_num,
  split,
  { intro h1,
    rw cos_eq_zero_iff at h1,
    cases h1 with k1 hk1,
    use k1,
    linarith, },
  { intro h2,
    cases h2 with k2 hk2,
    rw cos_eq_zero_iff,
    use k2,
    linarith,},
end
\end{leancode}
\index{SolucionCosenoCuadrado}

Se puede observar que en la demostración en Lean aparece el número
entero \(k\) representado por \textit{↑k}. Esto es debido a lo que en
Lean se conoce como coerción (o inmersión), en nuestro caso, entre el
conjunto de los números reales y el conjunto de los números
enteros. Realmente, esto no es más que todo número entero se puede ver
como un número real en el caso de que sea necesario.

Seguidamente, veamos los lemas auxiliares que se han utilizado
para realizar la formalización de este lema son los siguiente:
\begin{itemize}
\item \mint{lean}|cos_sq : cos x ^ 2 = 1 / 2 + cos (2 * x) / 2 |
  \indLema{cos\_sq}
\item \mint{lean}|add_right_eq_self : a + b = a ↔ a = 0|
  \indLema{add\_right\_eq\_self}
\item \mint{lean}|div_eq_zero_iff : a / b = 0 ↔ a = 0 ∨ b = 0|
  \indLema{div\_eq\_zero\_iff}
\end{itemize}
Se han usado las tácticas
\tactica{cases}{cases},
\tactica{exact}{exact},
\tactica{intro / intros}{intro},
\tactica{left / right}{right},
\tactica{ring}{ring},
\tactica{rw / rewrite}{rw} y
\tactica{split}{split}.

\begin{lema}[SolucionCosenoTriple]\label{lemaCosenoTriple}
  La expresión coseno del triple de \(x\) es igual a cero si y solamente
  si existe un \(k\) perteneciente al conjunto de los números enteros
  tal que \(x\) es igual al producto de \(2k\) más uno por \(π\)
  sexto. Simbólicamente:
  \begin{equation}\label{problemaTriple}
    \cos(3x) = 0 ⟺ ∃ k ∈ ℤ, x = \frac{(2k+1)π}{6}.
  \end{equation}
\end{lema}

\begin{demostracion}
  Antes de comenzar con la demostración en sí del lema, vamos a
  reescribir el problema (\ref{problemaTriple}) haciendo uso de la
  Proposición \ref{alpha}. Entonces, nos quedaría el siguiente problema:
  \begin{equation}
    ∃ k ∈ ℤ, 3x = \frac{(2k+1)π}{2} ⟺ ∃ k ∈ ℤ, x = \frac{(2k+1)π}{6}.
  \end{equation}

  Ahora sí, se procederá a la demostración de las correspondientes
  implicaciones por separado.

  \noindent
  \framebox{\longrightarrow}
  Supongamos que se verifica la siguiente hipótesis:
  \begin{equation}\label{Tripleh1}\tag{h1}
    ∃ k ∈ ℤ, 3x = \frac{(2k+1)π}{2}.
  \end{equation}

  Se quiere demostrar, que entonces se verifica también la siguiente
  afirmación:
  \begin{equation}\label{ConTriple}
     ∃k ∈ ℤ, x = \frac{(2k+1)π}{6}.
   \end{equation}

   Al considerar la hipótesis (\ref{Tripleh1}) sabemos que existe al
   menos un número entero verificando esa expresión. Sin pérdida de
   generalidad podemos considerar que el número entero que verifica
   (\ref{Tripleh1}) es \(k_1\). De manera que la hipótesis pasa a ser:
   \begin{equation}\label{Triplehk1}\tag{hk1}
    3x = \frac{(2k_1+1)π}{2}.
  \end{equation}

  Con todo esto, la demostración de (\ref{ConTriple}) se tiene de manera
  directa mediante el uso del número entero \(k_1\) y la hipótesis
  (\ref{Triplehk1}).

   \noindent
   \framebox{\longleftarrow}
   Supongamos que se verifica ahora la siguiente hipótesis:
   \begin{equation}\label{Tripleh2}\tag{h2}
     ∃ k ∈ ℤ, x = \frac{(2k+1)π}{6}.
  \end{equation}

  Se quiere demostrar que entonces, se verifica la siguiente afirmación:
  \begin{equation}\label{ConTriple2}
    ∃ k ∈ ℤ, 3x = \frac{(2k+1)π}{2}.
  \end{equation}

  De manera totalmente análoga a como se ha procedido en la implicación
  anterior; sin pérdida de generalidad, en la hipótesis
  (\ref{Tripleh2}), como sabemos que existe dicho número entero, se
  puede denotar como \(k_2\) y la hipótesis pasa a ser:
  \begin{equation}\label{Triplehk2}\tag{hk2}
     x = \frac{(2k_2+1)π}{6}.
   \end{equation}

   Entonces, ya se puede concluir la demostración, pues basta con usar
   el mismo número entero, \(k_2\), en (\ref{ConTriple2}) y usar la
   hipótesis (\ref{Triplehk2}).
\end{demostracion}

A continuación, se presenta la correspondiente formalización en Lean de
la demostración del lema que acabamos de detallar:
\begin{leancode}
lemma SolucionCosenoTriple
  {x : ℝ}
  : cos (3 * x) = 0 ↔ ∃ k : ℤ, x = (2 * k + 1) * π / 6 :=
begin
  rw cos_eq_zero_iff,
  split,
  { intro h1,
    cases h1 with k1 hk1,
    use k1,
    linarith,},
  { intro h2,
    cases h2 with k2 hk2,
    use k2,
    linarith,},
end
\end{leancode}
\index{SolucionCosenoTriple}

En la formalización de este lema en Lean sólo ha sido necesario el uso
de un teorema y es el siguiente:
\begin{itemize}
\item \mint{lean}|cos_eq_zero_iff : cos θ = 0 ↔ ∃ k : ℤ, θ = (2 * k + 1) * π / 2|
  \indLema{cos\_eq\_zero}
\end{itemize}
Además,se han usado las tácticas
\tactica{cases}{cases},
\tactica{intro / intros}{intro},
\tactica{linarith}{linarith},
\tactica{rw / rewrite}{rw},
\tactica{split}{split} y
\tactica{use}{use}.

\subsection{Conclusión}

Finalmente, una vez se ha demostrado tanto la equivalencia entre el
problema original que queríamos resolver y el problema auxiliar, como
las soluciones del problema auxiliar; se puede concluir el ejercicio.

En primer lugar, por lo que hemos visto en la subsección
\ref{secequiv}, se sabe que el problema equivalente
(y por tanto del problema original) tiene solución en dos
siguientes casos:
\begin{align}
  &∃ k ∈ ℤ, x = \frac{(2k+1)π}{4}  \label{condicion1}\\
  &∃ k ∈ ℤ, x = \frac{(2k+1)π}{6}. \label{condicion2}
\end{align}

Esto nos permite definir el conjunto de soluciones del problema como el
cojunto de los números reales, \(x\), tales que o bien verifiquen
(\ref{condicion1}), o bien (\ref{condicion2}). De manera formal, podemos
introducir el siguiente conjunto correspondiente a las soluciones del
problema:
\begin{equation}\label{ConjuntoSolucion}
  \Solucion =
  \left\{x ∈ ℝ \, | \, ∃ k ∈ ℤ, x = \frac{(2k+1)π}{4} ∨ x = \frac{(2k+1)π}{6}\right\}.
\end{equation}

La formalización en Lean de esta definición es la que se detalla
a continuación:
\begin{leancode}
def Solucion : set ℝ :=
{x : ℝ | ∃ k : ℤ, x = (2 * k + 1) * π / 4 ∨ x = (2 * k + 1) * π / 6}
\end{leancode}
\index{Solucion}

Por tanto, para concluir el problema basta con introducir un teorema que
hará uso de los lemas auxiliares que se han demostrado a lo largo de la
sección.

\begin{teorema}
  Sea \(x\) perteneciente al conjunto de los números reales, entonces
  \(x\) es solución la ecuación
  \begin{equation}\label{TeoremaCon}
    \cos²(x) + \cos²(2x) + \cos²(3x) = 1,
  \end{equation}
  si y solamente si \(x\) pertenece al conjunto descrito en
  (\ref{ConjuntoSolucion}).
\end{teorema}

\begin{demostracion}
  En primer lugar, a través del Lema \ref{lemaequivalenciaprob} se tiene
  que resolver el problema (\ref{TeoremaCon}) es equivalente a resolver
  la expresión:
  \begin{equation}\label{TeoremaPro}
    \funauxiliar(x) = 0 ⟺ x ∈ \Solucion.
  \end{equation}
  y, por \ref{lemaCasosSolucion}, a
  \begin{equation}\label{Objetivo2}
   \cos²(x) = \frac{1}{2} ∨ \cos(3x) =0 ⟺ x ∈ \Solucion.
  \end{equation}

  Entonces, sin más que hacer uso de los dos lemas auxiliares
  \ref{lemaCosenoCuadrado} y \ref{lemaCosenoTriple} que nos dan las
  condiciones necesarias y suficientes para que el coseno al cuadrado de
  \(x\) sea un medio (en el caso del primer lema) y para que el coseno
  del triple de \(x\) sea nulo (en el caso del segundo). Se tiene que
  (\ref{Objetivo2}) pasa a ser:
  \begin{equation}
    \left(∃ k ∈ ℤ, x = \frac{(2k+1)π}{4}\right) ∨
    \left(∃ k ∈ ℤ, x = \frac{(2k+1)π}{6}\right)
    ⟺ x∈ \text{Solucion}.
  \end{equation}

  A continuación, para concluir la demostración del teorema, basta con
  usar la propiedad distributiva del existencial sobre la disyunción,
  cuya formalización en Lean es
  \begin{leancode}
    theorem exists_or_distrib :
    (∃ x, p x ∨ q x) ↔ (∃ x, p x) ∨ (∃ x, q x)
  \end{leancode}

  Observando la definición del conjunto (\ref{ConjuntoSolucion}) y
  haciendo uso de la propiedad anterior se conluye el ejercicio.
\end{demostracion}

Finalmente, la formalización en Lean de este lema es:
\begin{leancode}
theorem imo1962_q4
  {x : ℝ}
  : problema x ↔ x ∈ Solucion :=
begin
  rw Equivalencia,
  rw CasosSolucion,
  rw SolucionCosenoTriple,
  rw SolucionCosenoCuadrado,
  rw Solucion,
  exact exists_or_distrib.symm,
end
\end{leancode}
\index{imo1962\_q4}

Se han usado las tácticas
\tactica{rw / rewrite}{rw} y
\tactica{exact}{exact}.
