\section{IMO 1962 Q4}

En esta sección se va a detallar la solución al problema Q4
correspondiente al año 1962. Realizaremos la demostración
en lenguaje natural del problema y a su vez se presentará la
correspondiente formalización en Lean.

La formalización que aquí se presenta ha sido inspirada en
las que se proponen en ???? realizadas por Kevin Lacker y
Healther Macbeth.



\noindent
\framebox{\parbox{\textwidth}{
  \textbf{Problema 2 (1962--Q4)}. Resolver la ecuación
  \[\cos²(x)+\cos²(2x)+\cos²(3x)=1\].}}

\textbf{Solución:}

Previo a la resolución como tal del problema, se van a
introducir una serie de definiciones que nos servirán
para llevar una notación más compacta:

\begin{definicion}\label{problema}
  De aquí en adelante usaremos la etiqueta \textbf{problema}
  para referirnos a la expresión que queremos resolver,
  es decir,
  \begin{equation}\label{expprob}
    \cos²(x)+\cos²(2x)+\cos²(3x)=1.
  \end{equation}
\end{definicion}

La formalización en Lean de esta definición es:
\begin{leancode}
 def problema (x : ℝ) : Prop :=
 cos x ^ 2 + cos (2 * x) ^ 2 + cos (3 * x) ^ 2 = 1
\end{leancode}

\begin{definicion}\label{funaux}
  De aquí en adelante usaremos la etiqueta
  \textbf{funauxiliar} para referirnos a la siguiente
  expresión:
  \begin{equation*}
    \cos(x)·(\cos²(x)-\frac{1}{2})·\cos(3x).
  \end{equation*}
\end{definicion}

La formalización en Lean en sete caso sería:
\begin{leancode}
def funcion_auxiliar (x : ℝ) : ℝ :=
cos x * (cos x ^ 2 - 1/2) * cos (3 * x)
\end{leancode}

\subsection{Equivalencia de problemas}
Una vez introducido estas dos definiciones, se va a proceder
a demostrar que resolver la ecuación\( \_ \)problema es
equivalente a resolver la ecuación obtenida al igualar a
cero la funauxiliar.

Esta demostración se detallará a continuación, pero previo a
ella se necesita de la demostración de un lema previo:

\begin{lema}[Igualdad]\label{igualdadlema}
  Para cualquier \(x\) perteneciente al conjunto de los
  números reales, se verifica la siguiente igualdad:
  \begin{equation}\label{lemaigualdad}
    \frac{\cos²(x)+\cos²(2x)+\cos²(3x)-1}{4}=\text{funauxiliar}(x).
  \end{equation}
\end{lema}
\begin{demostracion}
  En primer lugar, según la definición vista en \ref{funaux}, se
  tendría que la expresión (\ref{lemaigualdad}) se convierte en:
  \begin{equation}\label{igualdad}
    \frac{\cos²(x)+\cos²(2x)+\cos²(3x)-1}{4}=
    \cos(x)·(\cos²(x)-\frac{1}{2})·\cos(3x).
  \end{equation}
  Ahora bien, para llevar a cabo la demostración comenzaremos
  introduciendo las conocidas definiciones del coseno del
  ángulo doble y del coseno del ángulo triple, que son:
  \begin{align}
    & \cos(2x)=2\cos²(x)-\frac{1}{2}\label{cos2}\\
    & \cos(3x)=4\cos³(x)-3\cos(x)\label{cos3},
  \end{align}
  donde estas dos propiedades en Lean se formalizan con los
  dos siguientes lemas:
  \begin{itemize}
\item \mint{lean}|cos_two_mul : cos (2 * x) = 2 * cos x ^ 2 - 1|
  \index{\url{cos_two_mul}}
\item \mint{lean}|cos_three_mul : cos (3 * x) = 4 * cos x ^ 3 - 3 * cos x|
  \index{\url{cos_three_mul}}
\end{itemize}

A continuación, se van a desarrollar los dos términos de la
igualdad (\ref{igualdad}). Comencemos por el primer término
de los dos:
\begin{align}\label{term1}
  \frac{\cos²(x)+\cos²(2x)+\cos²(3x)-1}{4}&=\frac{\cos²(x)+(2\cos²(x)-1)²}{4}\\
  &+\frac{(4\cos³(x)-3\cos(x))²-1}{4},
\end{align}
donde simplemente se han introducido las definiciones (\ref{cos2})
y (\ref{cos3}).

Desarrollando los cuadrados que aparecen en (\ref{term1}) y
simplificando los términos se acaba teniendo que:
\begin{equation}\label{term11}
  \frac{\cos²(x)+\cos²(2x)+\cos²(3x)-1}{4}=
  \cos(x)(\frac{3}{2}\cos(x)+4\cos⁵(x)-5\cos³(x))
\end{equation}

Por otro lado, desarrollando de manera totalmente análoga el
segundo término de la igualdad (\ref{igualdad}), se tiene que:
\begin{align}
  \cos(x)·(\cos²(x)-\frac{1}{2})·\cos(3x)&\stackrel{(*)}{=}\cos(x)·
                                           (\cos²(x)-\frac{1}{2})·
                                           (4\cos³(x)-3\cos(x)
                                           )\label{term2}\\
                                         &=\cos(x)·
                                           (4\cos⁵(x)-5\cos³(x)
                                           +\frac{3}{2}\cos(x))
                                           \label{term21},
\end{align}
donde en (\ref{term2}) se ha hecho uso de las definiciones (\ref{cos2})
y (\ref{cos3}).

De manera que observando las expresiones (\ref{term11}) y (\ref{term21}),
se puede concluir que la igualdad planteada en (\ref{igualdad}) es cierta.
De esta forma, ya se tendría el lema demostrado.
\end{demostracion}

Veamos la formalización en Lean de este lema:
\begin{leancode}
lemma Igualdad {x : ℝ} :
(cos x ^ 2 + cos (2 * x) ^ 2 + cos (3 * x) ^ 2 - 1) / 4 = funauxiliar x :=
begin
  rw funauxiliar,
  rw real.cos_two_mul,
  rw cos_three_mul,
  ring,
end
\end{leancode}

Una vez se ha introducido este lema, se puede proceder a la demostración
del lema que ya se adelantó y que consiste en probar que resolver la
expresión \textbf{problema}, (\ref{problema}), es equivalente a resolver
la expresión obtenidad de igualar a cero \textbf{funauxiliar},
(\ref{funaux}). Veámoslo:

\begin{lema}[Equivalencia]
  Resolver la expresión (\ref{expprob}) es equivalente a resolver
  la expresión
  \begin{equation}\label{probaux}
    \text{funauxiliar}(x)=0.
  \end{equation}
\end{lema}
\begin{demostracion}
  Esta demostración se trata de una doble implicación, por ello,
  separaremos las implicaciones:

  \noindent
  \framebox{\longrightarrow}
  En primer lugar, comenzamos considerando el problema planteado
  por la siguiente expresión:
  \begin{equation}\label{h1}\tag{h1}
    \cos²(x)+\cos²(2x)+\cos²(3x)=1.
  \end{equation}

  Queremos demostrar que este problema planteado es equivalente a
  considerar el problema planteado en (\ref{probaux}).

  Aplicando el Lema \ref{igualdadlema} sobre (\ref{probaux}), se
  tiene que el objetivo a demostrar pasa a ser que el problema
  (\ref{h1}) es equivalente al problema:
  \begin{equation}\label{h11}
    \frac{\cos²(x)+\cos²(2x)+\cos²(3x)-1}{4}=0.
  \end{equation}

  Observando esta expresión, donde tenemos una fracción igualada
  a cero, se concluye que o bien el numerador o bien el denominador
  ha de ser nulo. Entonces, se tiene que demostrar (\ref{h11}) es
  equivalente a demostrar que
  \begin{equation}
    (\cos²(x)+\cos²(2x)+\cos²(3x)-1=0) \lor 4=0.
  \end{equation}

  Como se trata de una disyunción, para concluir la prueba bastaría
  con demostrar alguna de las dos igualdades y por ello, elegimos
  la primera parte de la disyunción, esto es,
  \begin{equation}\label{h12}
    \cos²(x)+\cos²(2x)+\cos²(3x)-1=0.
  \end{equation}

  De esta forma ya se ha teminado esta implicación pues (\ref{h12})
  es exactamente la hipótesis (\ref{h1}), basta con pasar el término
  unidad al lado derecho de la igualdad.


  \noindent
  \framebox{\longleftarrow} A continuación, supongamos que se tiene
  el problema
  \begin{equation}\label{h2}\tag{h2}
    \text{funauxiliar}(x)=0.
  \end{equation}
  Queremos demostrar que plantear este problema es equivalente al
  problema planteado en (\ref{expprob}).

  En primer lugar, aplicamos el Lema \ref{igualdadlema} sobre la
  hipótesis (\ref{h2}). De manera que se obtiene que la hipótesis
  se transforma en la siguiente:
  \begin{equation}
    \frac{\cos²(x)+\cos²(2x)+\cos²(3x)-1}{4}=0.
  \end{equation}
  
  Cuando se tiene una división igualda a cero, existen dos
  posibilidades: la primera de ellas es que el numerador sea
  cero, mientras que la segunda es o que lo sea el denominador.
  De esta forma, se verifica que:
  \begin{equation}
    (\cos²(x)+\cos²(2x)+\cos²(3x)-1=0) \lor 4=0
  \end{equation}

  Ahora bien, como sabemos que en el caso que nos concierne el
  denominador no puede ser nulo puesto que cuatro es distinto
  de cero; se tiene que el problema (\ref{h2}) es equivalente a:
  \begin{equation}
    \cos²(x)+\cos²(2x)+\cos²(3x)-1=0.
  \end{equation}

  Para concluir esta implicación basta con pasar el término uno
  hacia el otro lado de la igualdad. De esta forma se concluye que:
  \begin{equation}
    \cos²(x)+\cos²(2x)+\cos²(3x)=1.
  \end{equation}
\end{demostracion}

La formalización en Lean de este Lema es la siguiente:
\begin{leancode}
lemma Equivalencia {x : ℝ} : problema x ↔ funauxiliar x = 0 :=
begin
  split,
  {intro h1,
  rw problema at h1,
  rw ← Igualdad,
  rw div_eq_zero_iff,
  left,
  rw sub_eq_zero,
  exact h1,},
  {intro h2,
  rw problema,
  rw ← Igualdad at h2,
  rw div_eq_zero_iff at h2,
  norm_num at h2,
  rw sub_eq_zero at h2,
  exact h2,},
end
\end{leancode}

En la prueba del lema se han usado los siguientes lemas:
\begin{itemize}
\item \mint{lean}|div_eq_zero_iff : a / b = 0 ↔ a = 0 ∨ b = 0|
  \index{\url{div_eq_zero_iff}}
\item \mint{lean}|sub_eq_zero : a - b = 0 ↔ a = b |
  \index{\url{sub_eq_zero}}
\end{itemize}

Además se ha usado también la siguiente táctica:
\begin{leancode}
meta def norm_num (hs : parse simp_arg_list) (l : parse location) :
tactic unit :=
do f ← get_step, tactic.norm_num f hs l
\end{leancode}

En el caso de la prueba de nuestro lema, esta táctica ha sido
usada para demostrar que cuatro es distinto de cero. Esta táctica
ese tipo de igualdades las hace de manera directa.

\subsection{Resolución del problema equivalente}
Una vez ya se ha demostrado que el problema (\ref{probaux}) es
equivalente al problema que inicialmente queríamos demostrar, se
va a proceder a encontrar los ceros de (\ref{probaux}). Para
ello, se introducirán tres lemas auxiliares. El primero de ellos
consistirá en ver para en qué dos casos se verifica el problema
(\ref{probaux}); mientras que en los dos últimos lemas se detallará
la forma de la solución para los dos casos que se obtendrán en el
primer lema.

\begin{lema}[CasosSolucion]El problema (\ref{probaux}) se verifica
  si y solamente si o bien el coseno al cuadrado de \(x\) es igual
  a un medio, o bien, el coseno del triple de \(x\) es nulo.
  Simbólicamente:
  \begin{equation}\label{problema}
    \text{funauxiliar}(x)=0 ⟺ \cos²(x)=\frac{1}{2} \lor
    \cos(3x)=0.
  \end{equation}
  \end{lema}
\begin{demostracion}
  En primer lugar, previo a la demostración en sí del problema
  (\ref{problema}), se va a proceder a la reescritura del musmo.
  Para ello, lo que hacemos es escribir la definición de
  funauxiliar vista en (\ref{funaux}), es decir,
  \begin{equation}\label{fun1}
    \text{funauxiliar}(x)=\cos(x)·(\cos²(x)-\frac{1}{2})·\cos(3x).
  \end{equation}

  Aplicando ahora la propiedad asociativa de la multiplicaión,
  se tiene que (\ref{fun1}) es equivalente a
  \begin{equation}\label{fun2}
    \cos(x)·((\cos²(x)-\frac{1}{2})·\cos(3x)).
  \end{equation}

  De esta forma como en el problema inicial se tiene que funaxiliar
  está igualda a cero; se sabe que cuando un producto está igualado
  a cero uno de los dos términos ha de ser nulo. En el caso que a
  nosotros nos incumbe:
  \begin{align}
    \cos(x)·((\cos²(x)-\frac{1}{2})·\cos(3x))=0 &⟺ \cos(x)=0
                                                  \lor (\cos²(x)-
                                                  \frac{1}{2})·
                                                  \cos(3x)=0 \\
    & ⟺ \cos(x)=0 \lor \cos²(x)=\frac{1}{2} \lor \cos(3x)=0,
  \end{align}
  donde hemos aplicado dos veces el resultado de que el producto
  de dos términos esté igualado a cero.

  De esta forma, tenemos que el problema inicial, (\ref{problema}),
  se reescribe de la siguiente forma:
  \begin{equation}\label{problema2}
    \cos(x)=0 \lor \cos²(x)=\frac{1}{2} \lor \cos(3x)=0
    ⟺ \cos²(x)=\frac{1}{2} \lor \cos(3x)=0.
  \end{equation}

  Ahora ya sí procederemos a la demostración de (\ref{problema2});
  como se trata de un si y solamente si lo haremos por doble
  implicación.

  \noindent
  \framebox{\longrightarrow} Suponemos que se verifica que
  \begin{equation}\label{Casosh1}\tag{h1}
    \cos(x)=0 \lor \cos²(x)=\frac{1}{2} \lor \cos(3x)=0.
  \end{equation}

  Entonces tenemos que demostrar que bajo la suposición (\ref{h1}),
  se verifica que
  \begin{equation}\label{CasosConc1}
     \cos²(x)=\frac{1}{2} \lor \cos(3x)=0.
  \end{equation}

  La hipótesis (\ref{h1}) significa que una de las dos disyunciones
  ha de ser cierta. Por ello, considerar dicha hipótesis es
  equivalente a considerar dos posibles casos, que serían los
  siguientes:
  \begin{align}
    &\cos(x)=0 \label{Casosh11}\tag{h11}\\
    &\cos²(x)=\frac{1}{2}\lor \cos(3x)=0. \label{Casosh12}\tag{h12}
  \end{align}

  Esto supone que la demostración pasa a ser dividida en
  dos subproblemas: el primero de
  ellos consiste en demostrar (\ref{CasosConc1}) suponiendo
  cierta la hipótesis (\ref{Casosh11}) y el segundo de ellos
  es también demostrar (\ref{CasosConc1}) pero ahora bajo la
  hipótesis (\ref{Casosh12}).

  \begin{itemize}
  \item \textbf{Subproblema 1}
    
    Con el objetivo de demostrar (\ref{CasosConc1}) suponiendo
    cierta la hipótesis (\ref{Casosh11}), basta con demostrar una
    de las dos disyunciones. En este caso se demostrará la segunda
    de ellas, es decir, se demsotrará que el coseno del ángulo
    \(3x\) es nulo.

    Para demostrar esto, basta con hacer uso de la definción del
    ángulo triple. De esta forma se tiene que:
    \begin{equation}
      \cos(3x)=4\cos³(x)-3\cos(x)\stackrel{(*)}{=}0,
    \end{equation}
    donde en \((*)\) se ha hecho uso de la hipótesis
    (\ref{Casosh11}). Y ya se tendría el resultado.
    
  \item \textbf{Subproblema 2}

    En este caso suponemos cierta la hipótesis (\ref{Casosh12})
    y tenemos que demostrar (\ref{CasosConc1}), lo cual se tiene
    de manera directa pues son la misma afirmación.
  \end{itemize}

  \noindent
  \framebox{\longleftarrow} A continuación se tiene que demostrar
  la implicación contraria. Esto es, suponiendo cierta la
  hipótesis
  \begin{equation}\label{Casosh2}\tag{h2}
    \cos²(x)=\frac{1}{2} \lor \cos(3x)=0,
  \end{equation}
  hay que demostrar que es equivalente a
  \begin{equation}\label{CasosConc2}
    \cos(x)=0 \lor \cos²(x)=\frac{1}{2}\lor \cos(3x)=0.
  \end{equation}

  En este caso, la implicación es directa puesto que basta con
  quedarse con las dos disyunciones de la derecha en
  (\ref{CasosConc2}).
\end{demostracion}

A continuación, se plantea la formalización en Lean de este lema:
\begin{leancode}
lemma CasosSolucion {x : ℝ} :
funauxiliar x = 0 ↔ cos x ^ 2 = 1/2 ∨ cos (3 * x) = 0 :=
begin
  rw funauxiliar,
  rw mul_assoc,
  rw mul_eq_zero,
  rw mul_eq_zero,
  rw sub_eq_zero,
  split,
  {intro h1,
  cases h1 with h11 h12,
  right,
  rw cos_three_mul,
  rw h11,
  ring,
  exact h12,},
  {intro h2,
  right,
  exact h2,},
end
\end{leancode}

En la prueba del lema se han usado los siguientes lemas:
\begin{itemize}
\item \mint{lean}|cos_three_mul : cos (3 * x) = 4 * cos x ^ 3 - 3 * cos x|
  \index{\url{cos_three_mul}}
\item \mint{lean}|mul_assoc : ∀ a b c : G, a * b * c = a * (b * c) |
  \index{\url{mul_assoc}}
\item \mint{lean}|mul_eq_zero : a * b = 0 ↔ a = 0 ∨ b = 0|
  \index{\url{mul_eq_zero}}
\item \mint{lean}|sub_eq_zero : a - b = 0 ↔ a = b|
  \index{\url{sub_eq_zero}}
\end{itemize}

\begin{lema}[SolucionCosenoCuadrado]La expresión coseno al
  cuadrado de \(x\) es igual a un medio si y solamente si
  existe un \(k\) perteneciente al conjunto de los números
  enteros tal que \(x\) es igual al producto de \(2k\) más
  uno por \(π\) cuarto. Simbólicamente:
  \begin{equation}\label{problema2}
    \cos²(x)=\frac{1}{2} ⟺ x=(2k+1)⋅\frac{π}{4}.
  \end{equation}
\end{lema}
\begin{demostracion}
  Se puede observar que se trata de una demostración de doble
  implicación. No obstante, previo a la demostración de las
  correspondientes implicaciones, reescribiremos el problema
  (\ref{problema2}). Para ello, comenzaremos usando la definición
  del cuadrado del coseno de un ángulo, es decir,
  \begin{equation}
    \cos²(x)=\frac{1}{2}+\cos(2x).
  \end{equation}

  De esta forma, el problema (\ref{problema2}), se puede reescribir
  como sigue:
  \begin{equation}
    \frac{1}{2}+\frac{\cos(2x)}{2}=\frac{1}{2} ⟺ x=(2k+1)⋅
    \frac{π}{4}.
  \end{equation}

  Se puede observar que en la primera parte de la equivalencia, la
  igualdad tiene el mismo término a derecha que a izquierda, de
  manera que se puede eliminar. Obtenemos que
  \begin{equation}\label{problema21}
    \frac{\cos(2x)}{2}=0 ⟺ x=(2k+1)⋅\frac{π}{4}.
  \end{equation}

  Por último, en (\ref{problema21}), se tiene una fracción igualada
  a 0, esto significa que o el numerador es cero o bien lo es el
  denominador. Por tanto, el problema que inicialmente planteamos,
  (\ref{problema2}), quedaría reescrito de la siguiente manera:
  \begin{equation}\label{problema22}
    \cos(2x)=0 \lor 2=0 ⟺ x=(2k+1)⋅\frac{π}{4}.
  \end{equation}

  Procedamos ahora a la demostración por doble implicación del
  problema (\ref{problema22}).

  \noindent
  \framebox{\longrightarrow} Supongamos que se verifica que
  \begin{equation}\label{Dobleh2}\tag{h1}
    \cos(2x)=0 \lor 2=0.
  \end{equation}

  Entonces, se quiere demostrar que la solución \(x\) es de la
  siguiente forma:
  \begin{equation}\label{ConcDoble}
    x=(2k+1)⋅\frac{π}{4}.
  \end{equation}
  Al igual que en la demostración del lema anterior, al tener una
  hipótesis formada por una disyunción, se tiene que dicha hipótesis
  puede ser separada en dos posibles casos:
  \begin{align}
    &\cos(2x)=0 \label{Dobleh11}\tag{h11}\\
    &2=0. \label{Dobleh12}\tag{h12}
  \end{align}

  De manera análoga, para demostrar (\ref{ConcDoble}) ahora se
  divide en dos subproblemas: el primero de ellos considerando
  la hipótesis (\ref{Dobleh11}) y en el segundo considerando
  (\ref{Dobleh12}).

  \begin{itemize}
  \item \textbf{Subproblema 1}

    Supongamos (\ref{Dobleh11}), se ha de demostrar
    (\ref{ConcDoble}). Para ello, comenzaremos usando que sabemos
    que cuando tenemos el coseno de un ángulo \( α \) igualado
    a cero, se verfica que, para todo \(k\) perteneciente a los
    números naturales, dicho ángulo es igual al producdo de el
    doble de \(k\) más uno por \(π \) medios. Simbólicamente:
    \begin{equation}\label{alpha}
      \cos(α)=0 ⟺ α=\frac{(2k+1)π}{2}, ∀ k ∈ ℤ.
    \end{equation}

    Entonces, usando la propiedad (\ref{alpha}) sobre nuestra
    hipótesis (\ref{Dobleh11}), se tiene que
    \begin{equation}\label{hk1}\tag{hk1}
      2x=\frac{(2k_1+1)π}{2}, ∀ k_1 ∈ ℤ.
    \end{equation}

    De esta forma, sin más que pasar dividiendo el dos que se
    encuentra multiplicando en la parte izquierda de la igualdad
    se tiene lo que queremos demostrar, (\ref{ConcDoble}).

  \item \textbf{Subproblema 2}

    En este caso, partimos de la hipótesis (\ref{Doble12}), lo cual
    bien sabemos que es una hipótesis falsa. Entonces, se tendría
    que partiendo de una hipótesis falsa cualquier afirmación es
    cierta, y en concreto la que a nosotros nos interesa también,
    (\ref{ConcDoble}).
  \end{itemize}

  De manera que ya hemos terminado la primera implicación de la
  demostración.

   \noindent
   \framebox{\longrightarrow} Supongamos ahora que existe  \(k_2\)
   perteneciente al conjunto de los números enteros, a partir del
   cual \(x\) se puede escribir como el producto de el doble de
   \(k_2 \) más uno, por \( π \) medio. Simbólicamente
   \begin{equation}\label{hk2}\tag{hk2}
      x=\frac{(2k_2+1)π}{4}, ∀ k_2 ∈ ℤ.
    \end{equation}

    Entonces queremos demostrar que se verifica la siguiente
    expresión:
    \begin{equation}\label{ConcDoble2}
      \cos(2x)=0 \lor 2=0.
    \end{equation}

    Es más, como se trata de una disyunción para concluir la
    prueba basta con demostrar una de las dos posibilidades.
    En nuestro caso, se demostrará que el coseno del ángulo doble
    de \(x\) es nulo. Esto es,
    \begin{equation}\label{ConcDoble3}
      \cos(2x)=0.
    \end{equation}

    Análogamente a como hicimos en la implicación anterior,
    sustituiremos la expresión del coseno igualada a cero por la
    descrita en (\ref{alpha}). Entonces, si se lo aplicamos a
    (\ref{ConcDoble3}), se obtiene que existe ????
    \end{demostracion}

Formalización en Lean del lema:
\begin{leancode}
lemma SolucionCosenoCuadrado {x : ℝ} : cos x ^ 2 = 1/2 ↔
∃ k : ℤ, x = (2 * ↑k + 1) * π / 4 :=
begin
  rw cos_square,
  rw add_right_eq_self,
  rw div_eq_zero_iff,
  split,
  {intro h1,
  cases h1 with h11 h12,
  rw cos_eq_zero_iff at h11,
  cases h11 with k1 hk1,
  use k1,
  linarith,
  norm_num at h12,},
  {intro h2, 
  cases h2 with k2 hk2,
  left,
  rw cos_eq_zero_iff,
  use k2,
  linarith,},
end
\end{leancode}


\begin{lema}[SolucionCosenoTriple]La expresión coseno del
  triple de \(x\) es igual a cero si y solamente si
  existe un \(k\) perteneciente al conjunto de los números
  enteros tal que \(x\) es igual al producto de \(2k\) más
  uno por \(π\) sexto. Simbólicamente:
  \begin{equation}
    \cos(3x)=0 ⟺ x=(2k+1)⋅\frac{π}{6}.
  \end{equation}
\end{lema}
\begin{demostracion}
  Demostracion del lema SolucionCosenoTriple en lenguaje
  natural.
\end{demostracion}

A continuación, se presenta la correspondiente formalización
en Lean del lema anterior:
\begin{leancode}
lemma SolucionCosenoTriple {x : ℝ} : cos (3 * x) = 0 ↔ 
∃ k : ℤ, x = (2 * ↑k + 1) * π / 6 :=
begin
  rw cos_eq_zero_iff,
  split,
  {intro h1,
  cases h1 with k1 hk1,
  use k1,
  linarith,},
  {intro h2,
  cases h2 with k2 hk2,
  use k2,
  linarith,},
end
\end{leancode}


\subsection{Conclusión}
Aquí se detallará la conclusión final.