\section{IMO 2001 Q6}

En esta nueva sección, se va a detallar la resolución del problema Q6
propuesto en el año 2001. Como en el resto de planteamientos anteoriores
que se han detallado en el capítulo anterior, se comenzará presentado la
demostración en lenguaje natural del problema y luego su correspondiente
formalización en Lean.

Para la realización de esta formalización, se ha hecho uso de la propuesta de
solución en lenguaje natural del problema en cuestión que aparece en ???.

\noindent
\framebox{\parbox{\textwidth}{
    \textbf{Problema 1 (2001--Q6)}. Sean \(a\), \(b\), \(c\) y \(d\) cuatro
    números enteros tales que \(a > b > c > d > 0\). Supongamos que
    \begin{equation*}
      ac+bd = (a+b-c+d)(-a+b+c+d).
    \end{equation*}
    Demostrar que \(ab+cd\) no es primo.
  }}

Para llevar a cabo la formalización en Lean de la resolución de este
problema, ha sido necesaria la importación de las siguientes teorías:
\begin{itemize}
\item \href{https://github.com/leanprover-community/mathlib/blob/master/src/data/int/basic.lean}{data.int.basic}, para poder trabajar con el conjunto
  de los números enteros.
\item \href{https://github.com/leanprover-community/mathlib/blob/master/src/algebra/associated.lean}{algebra.associated}, para trabajar con las
  definición de número primo.
\item \href{https://github.com/leanprover-community/mathlib/tree/master/src/tactic}{tactic}, para poder acceder a las diferentes tácticas.
\end{itemize}

La importación en Lean de todas estas teorías se formaliza como sigue:
\begin{leancode}
import data.int.basic
import algebra.associated
import tactic
\end{leancode}

Además, en este problema hay que definir las variables \(a\), \(b\), \(c\) y
\(d\) que pertenecen al conjunto de los números enteros. Las diferentes
desigualdades que verifican se enunciarán en los lemas en los que sean
necesarias dichas condiciones. En Lean esto se formaliza de la siguiente
manera:
\begin{leancode}
variables {a b c d : ℤ}
\end{leancode}

A continuación, se va a dividir en dos partes esta sección: en la primera
parte probaremos cinco resultados auxiliares que han sido necesarios para
la formalización del problema y en la segunda parte se formalizará el
teorema final que nos da el resultado deseado.

\subsection{Resultados auxiliares}
Como ya se ha dicho anteriormente, en esta sección del problema se van
a formalizar cinco resultados auxiliares. Veámos estos lemas con sus
correspondientes formalizaciones en Lean:

\begin{lema}[sumas\_equivalentes]\label{q601lemasuma}
  Sean \(a\), \(b\), \(c\) y \(d\) cuatro números enteros tales que 
    \begin{equation}\label{q601hipsum}\tag{h}
      ac+bd = (a+b-c+d)(-a+b+c+d).
    \end{equation}
    Entonces se tiene que se verifica que
    \begin{equation}\label{q601objsum}
      b²+bd+d²=a²-ac+c².
    \end{equation}
\end{lema}

\begin{demostracion}
  Para demostrar (\ref{q601objsum}) se comenzará desarrollando la segunda
  parte de la igualdad de la hipótesis (\ref{q601hipsum}), obteniendo el
  siguiente resultado:
  \begin{equation}\tag{h1}\label{q601h1sum}
    (a+b-c+d)(-a+b+c+d)=-a²+b²+ac-c²+bd+d²+ac+bd.
  \end{equation}

  A continuación, reescribimos el segundo término de la hipótesis
  (\ref{q601hipsum}) como se ha descrito en (\ref{q601h1sum}). De esta
  forma se llega a la siguiente igualdad:
  \begin{equation}\label{q601hip2sum}
    ac+bd=-a²+b²+ac-c²+bd+d²+ac+bd.
  \end{equation}

  La conclusión de la prueba ya se tiene de manera trivial. Sin más que
  simplificar la ecuación (\ref{q601hip2sum}) y redistribuyendo los términos
  se obtiene (\ref{q601objsum}).
\end{demostracion}

Este resultado se formaliza en Lean como a continuación se plantea:
\begin{leancode}
lemma sumas_equivalentes
  (h : a*c + b*d = (a + b - c + d) * (-a + b + c + d))
  : b^2 + b*d + d^2 = a^2 - a*c + c^2 :=
begin
  have h1: (a + b - c + d) * (-a + b + c + d) =
           -a^2  + b^2  + a*c - c^2 + b*d + d^2 + a*c + b*d,
    by ring,
  rw h1 at h,
  by nlinarith,
end
\end{leancode}
\index{sumas\_equivalentes}

Para levar a cabo la formalización de este lema no hemos hecho uso de
ningún lema auxiliar. No obstante, sí se han utilizado las tácticas de
\tactica{have}{have},
\tactica{rw / rewrite}{rw},
\tactica{ring}{ring} y 
\tactica{nlinarith}{nlinarith}.

Prosigamos con la demostración del problema, para ello el lema que se
presenta a continuación nos dará la igualdad de dos productos que será
de gran utilidad en la conclusión del problema.

\begin{lema}[productos\_equivalentes]\label{q601lemaprod}
    Sean \(a\), \(b\), \(c\) y \(d\) cuatro números enteros verificando que 
    \begin{equation}\label{q601hipprod}\tag{h}
      ac+bd = (a+b-c+d)(-a+b+c+d).
    \end{equation}
    Entonces se tiene que se verifica la siguiente igualdad
    \begin{equation}\label{q601objprod}
      (ac+bd)(b²+bd+d²)=(ab+cd)(ad+bc).
    \end{equation}
\end{lema}

\begin{demostracion}
  Para llevar a cabo la prueba de este resultado, se van a desarrollar los
  dos lados de la igualdad que se quiere demostrar, (\ref{q601objprod}),
  y se va a comprobar que son iguales.

  Comencemos desarrollando la parte izquierda de (\ref{q601objprod}):
  \begin{equation}\label{q601h1prod}\tag{h1}
    (ac+bd)(b²+bd+d²)=ac(b²+bd+b²)+bd(b²+bd+b²).
  \end{equation}

  Ahora bien, haciendo uso del lema que hemos demostrado antes, es decir,
  el lema \ref{q601lemasuma} sobre el segundo término que aparece sobre
  el que se puede aplicar, se tiene que
  \begin{equation}\label{q601h2prod}\tag{h2}
    ac(b²+bd+b²)+bd(b²+bd+b²)=ac(b²+bd+b²)+bd(a²-ac+c²).
  \end{equation}

  De esta forma, introduciendo el resultado obtenido en (\ref{q601h2prod})
  en (\ref{q601h1prod}), se llega a que:
  \begin{equation}\label{q601pr1}
     (ac+bd)(b²+bd+d²)=ac(b²+bd+b²)+bd(a²-ac+c²).
  \end{equation}

  Ahora bien, si seguimos desarrollando la segunda parte de la igualdad
  (\ref{q601pr1}), se llega a que:
  \begin{equation}\label{q601pr2}
     (ac+bd)(b²+bd+d²)= acb²+acd²+a²bd+bcd².
  \end{equation}

   Si desarrollamos el lado derecho de la ecuación que queremos demostrar,
   es decir, de (\ref{q601objprod}), se llega a:
   \begin{equation}\label{q601pr3}
     (ab+cd)(ad+bc)= acb²+acd²+a²bd+bcd².
   \end{equation}

   Se puede comprobar a través de las expresiones (\ref{q601pr2}) y
   (\ref{q601pr3}) que llegamos a lo mismo al desarrollar los dos términos
   de la expresión (\ref{q601objprod}). Por tanto, ya se tendría la
   prueba del lema.
\end{demostracion}

En este caso, la formalización en Lean es:
\begin{leancode}
lemma productos_equivalentes
  (h : a*c + b*d = (a + b - c + d) * (-a + b + c + d))
  : (a*c + b*d) * (b^2 + b*d + d^2) = (a*b + c*d) * (a*d + b*c) :=
begin
  have h1: (a*c + b*d) * (b^2 + b*d + d^2) =
           a*c * (b^2 + b*d + d^2) + b*d * (b^2 + b*d + d^2),
    by ring,
  have h2: a*c * (b^2 + b*d + d^2) + b*d * (b^2 + b*d + d^2) =
           a*c * (b^2 + b*d + d^2) + b*d * (a^2 - a*c + c^2),
    by rw sumas_equivalentes h,
  rw h2 at h1,
  by nlinarith,
end
\end{leancode}
\index{productos\_equivalentes}

A continuación, se van a probar dos lemas muy parecidos (pues ambos nos
afirman la veracidad de una desigualdad), que su vez son muy sencillos de
formalizar en Lean bajo las hipótesis del problema.

\begin{lema}[desigualdad\_auxiliar1]\label{lemades1}
  Sean \(a\), \(b\), \(c\) y \(d\) cuatro números enteros tales que
  \(a > b > c > d \). Además, supongamos que
  estos números enteros verifican que
    \begin{equation}
      ac+bd = (a+b-c+d)(-a+b+c+d).
    \end{equation}
    Entonces se tiene que \(ac+bd\) es menor que \(ab+cd\). Simbólicamente:
    \begin{equation}\label{q601objdes1}
      ac+bd<ab+cd
    \end{equation}
\end{lema}

\begin{demostracion}
    En primer lugar, denotemos las hipótesis del enunciado como sigue:
  \begin{align}
    &b<a,\tag{hba}\label{hba01des1}\\
    &c<b,\tag{hcb}\label{hcb01des1}\\
    &d<c.\tag{hdc}\label{hdc01des1}
  \end{align}

  Por otro lado, se tiene qye demostrar (\ref{q601objdes1}) es equivalente
  a demostrar la siguiente desigualdad:
  \begin{equation}\label{q601objdes12}
      0<ab+cd-(ac+bd).
  \end{equation}

  Sin embargo, desarrollando la resta de (\ref{q601objdes12}), se obtiene
  que el objetivo a probar se convierte en:
  \begin{equation}\label{q601objdes13}
      0<(a-d)(b-c).
  \end{equation}

  No obstante, este resultado es inmediato a partir de las desigualdades
  que tenemos de hipótesis, es decir, (\ref{hba01des1}), (\ref{hcb01des1})
  y (\ref{hdc01des1}).
\end{demostracion}

Formalizar este lema en Lean es bastante sencillo y directo ya que existe
una táctica que a través de las desigualdades que se tienen por hipótesis,
deduce el resultado. La formalización sería:
\begin{leancode}
lemma desigualdad_auxiliar1
  (hba : b < a)
  (hcb : c < b)
  (hdc : d < c)
  (h : a*c + b*d = (a + b - c + d) * (-a + b + c + d))
  : a*c + b*d < a*b + c*d:=
by nlinarith
\end{leancode}
\index{desigualdad\_auxiliar1}

En la demostración de este lema sólo ha sido necesario el uso de la táctica
\tactica{nlinarith}{nlinarith}.

\begin{lema}[desigualdad\_auxiliar2]\label{lemades2}
  Sean \(a\), \(b\), \(c\) y \(d\) cuatro números enteros tales que
  \(a > b\) y \( c > d \). Además, supongamos que
  estos números enteros verifican que
    \begin{equation}
      ac+bd = (a+b-c+d)(-a+b+c+d).
    \end{equation}
    Entonces se tiene que \(ad+bc\) es menor que \(ac+bd\). Simbólicamente:
    \begin{equation}
      ad+bc<ac+bd
    \end{equation}
\end{lema}

\begin{demostracion}
  Se detallará la prueba del lema.
\end{demostracion}

Análogamente al lema anterior, \ref{lemades1}, la formalización en Lean de
este lema sería inmediata:
\begin{leancode}
lemma desigualdad_auxiliar2
  (hba : b < a)
  (hdc : d < c)
  (h : a*c + b*d = (a + b - c + d)*(-a + b + c + d))
  : a*d + b*c < a*c + b*d:=
by nlinarith
\end{leancode}
\index{desigualdad\_auxiliar2}

\subsection{Conclusión del problema}
Conluiremos el problema en esta sección, para ello se hará uso de los lemas
auxiliares probados anteriormente. Para resolver el problema que nos
interesa, se enunciará el siguiente teorema:

\begin{teorema}[imo2001q6]
  Sean \(a\), \(b\), \(c\) y \(d\) cuatro
  números enteros tales que \(a > b > c > d > 0\). Además, supongamos que
  estos números enteros verifican que
    \begin{equation}
      ac+bd = (a+b-c+d)(-a+b+c+d).
    \end{equation}
    Entonces se tiene que \(ab+cd\) no es primo.
\end{teorema}
\begin{demostracion}
  Aquí se detallará la demostración del teorema.
\end{demostracion}

La formalización en Lean de este teorema sería:
\begin{leancode}
theorem imo2001q6
  (hd  : 0 < d)
  (hdc : d < c)
  (hcb : c < b)
  (hba : b < a)
  (h : a*c + b*d = (a + b - c + d)*(-a + b + c + d))
  : ¬ prime (a*b + c*d) :=
begin
  intro h1,
  have ha : 0 < a,
    by linarith,
  have hb : 0 < b,
    by linarith,
  have hc : 0 < c,
    by linarith,
  have h2 : (a*c + b*d) * (b^2 + b*d + d^2) =
            (a*b + c*d) * (a*d + b*c),
    from productos_equivalentes h,
  have h3 : a*c + b*d ∣ (a*b + c*d) * (a*d + b*c),
    from division h,
  have h4: (a*b + c*d ∣ a*c + b*d) ∨ (a*c + b*d  ∣ a*d + b*c),
    from left_dvd_or_dvd_right_of_dvd_prime_mul h1 h3,
  cases h4 with hj hk,
  { have hj1: a*b + c*d > a*c + b*d,
      from desigualdad_auxiliar1 hba hcb hdc h,
    have hpj: 0 < a*c + b*d,
      from add_pos (mul_pos ha hc) (mul_pos hb hd),
    have hj2: a*b + c*d ≤ a*c + b*d,
      from int.le_of_dvd hpj hj,
    have hj3: ¬ (a*b + c*d ≤ a*c + b*d),
      from not_le.mpr hj1,
    exact hj3 hj2, },
  { have hk1: a*c + b*d > a*d + b*c,
      from desigualdad_auxiliar2 hba hdc h,
    have hpk: 0 < a*d + b*c,
      from add_pos (mul_pos ha hd) (mul_pos hb hc),
    have hk2: a*c + b*d ≤ a*d + b*c,
      from int.le_of_dvd hpk hk,
    have hk3: ¬ (a*c+b*d ≤  a*d+b*c),
      from not_le.mpr hk1,
    exact hk3 hk2, },
end
\end{leancode}

%%% Local Variables:
%%% mode: latex
%%% TeX-master: "IMO_en_Lean"
%%% End:
