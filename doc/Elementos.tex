\chapter{Elementos de matemáticas en Lean}

En este capítulo se presentan algunas demostraciones de las que se
estudian en diversas asignaturas del Grado de Matemáticas formalizadas en Lean.

\section{Cálculo infinitesimal}
En esta primera sección vamos a ver diferentes resultados de análisis correspondientes a la asignatura de formación básica Cálculo Infinitesimal estudiada en el primer año del Grado.\\

Estudiaremos un total de seis resultados. Los tres primeros son relativos a la convergencia de sucesiones, mientras que los tres últimos son sobre la paridad de las funciones.

\subsection{Unicidad del límite}

\begin{definicion}\label{limite}
  Se dice que \(a\) es el límite de la sucesión \(u\) cuando \( \forall \epsilon >0, \exists N\) tal que \( \forall N \in \mathbb{N}\) se verifica que \(\forall n \geq N, \ |u_n-a| \leq \epsilon\).
\end{definicion}

La formalización de la definición anterior en Lean es
\begin{leancode}
def limite : (ℕ → ℝ) → ℝ → Prop :=
λ u c, ∀ ε > 0, ∃ N, ∀ n ≥ N, |u n - c| ≤ ε para todo
\end{leancode}
donde la notación para el valor absoluto se ha definido por
\begin{leancode}
notation `|`x`|` := abs x
\end{leancode}

\begin{teorema}
Cada sucesión tiene como máximo un límite.
\end{teorema}

\begin{demostracion}
  Comenzaremos viendo la demostración en lenguaje natural. Para ello, procederemos por Reducción al Absurdo: supondremos que la sucesión \( u\) posee dos límites distintos que denotaremos por \( l\) y \(l'\). Acabaremos demostrando que \(l=l'\). 
  
  En primer lugar, por la definición de límite anterior, \ref{limite}, se tiene que para el límite \(l \) podemos escribir que:
  \begin{equation}\label{lim1}
  \forall \epsilon >0, \exists N_1 \in \mathbb{N} \ \text{tal que }\forall n \geq N_1 |u_n-l| \leq \frac{\epsilon}{2}.
  \end{equation}
  Análogamente, para \(l'\) se tiene que:
  \begin{equation}\label{lim2}
  \forall \epsilon >0, \exists N_2 \in \mathbb{N} \ \text{tal que }\forall n \geq N_2 |u_n-l'| \leq \frac{\epsilon}{2}.
  \end{equation}
  
  A continuación, consideremos \( N_0 = \text{max} (N_1,N_2)\). De manera que considerando \(n \geq N_0\) y teniendo en cuenta lo visto anteriormente, se tiene que
  \begin{equation*}
  |l-l'|=|l-u_{N_0}+u_{N_0}-l'| \stackrel{(*)}{\leq} |u_{N_0}-l|+|u_{N_0}-l'| \stackrel{(**)}{\leq} \frac{\epsilon}{2}+\frac{\epsilon}{2}=\epsilon,
  \end{equation*}
  donde en \((*)\) se ha hecho uso de la desigualdad triangular y la definición de valor absoluto y en \((**)\) se ha usado las desigualdades (\ref{lim1}) y (\ref{lim2}).
  
  Por tanto se ha demostrado que \(\forall \epsilon >0\) se tiene que \( |l-l'| \leq \epsilon \). Finalmente, se tendría el resultado deseado haciendo uso del siguiente resultado:

  	Sean \(x,y \in \mathbb{R}\), si \(\forall \epsilon >0\) se verifica que \( |x-y| \leq \epsilon \), entonces \(x=y\).
\end{demostracion}

La formalización en Lean del teorema anterior es
\begin{leancode}
import data.real.basic
import algebra.group.pi
import tuto_lib

variables (u : ℕ → ℝ)
variables (a b x y : ℝ)

example
(ha : limite u a)
(hb : limite u b)
: a = b :=
begin
apply eq_of_abs_sub_le_all,
intros eps eps_pos,
cases ha (eps/2) (by linarith) with N1 hN1,
cases hb (eps/2) (by linarith) with N2 hN2,
let N0:=max N1 N2,
calc |a-b|= |(a-u N0+(u N0-b)| : by ring
...≤ |a-u N0| + |u N0-b| :by apply abs_add
...= |u N0-a| + |u N0-b| : by rw abs_sub 
... ≤ eps : by linarith [hN1 N0 (le_max_left N1 N2), 
hN2 N0 (le_max_right N1 N2)],
end
\end{leancode}

\subsection{Las sucesiones convergentes son sucesiones de Cauchy}
Previo al enunciado y demostración del Teorema, veamos en qué consiste una sucesión de Cauchy.
\begin{definicion}
	Se dice que una sucesión \(u\) es de Cauchy si \(\forall \epsilon >0, \ \exists N\in \mathbb{N} \) tal que \( \forall p, q \geq N\) se verifica que \(|u_p-u_q| \leq \epsilon \).
\end{definicion}
En Lean esta definición se formaliza como:
\begin{leancode}
def cauchy_sequence (u : ℕ → ℝ) := 
∀ ε > 0, ∃ N, ∀ p q, p ≥ N → q ≥ N → |u p - u q| ≤ ε
\end{leancode}
\begin{teorema}
	Toda sucesión convergente es una sucesión de Cauchy.
\end{teorema}
\begin{demostracion}
	Sea \(u\) una sucesión convergente, es decir, por la definición \ref{limite}, se tiene que si denotamos \(l\) como el límite de la sucesión en cuestión se verifica que
\begin{equation}\label{lim3}
\forall \epsilon >0, \exists N \in \mathbb{N} \ \text{tal que }\forall n \geq N |u_n-l| \leq \frac{\epsilon}{2}.
\end{equation}

A continuación, consideremos \(p,q \in \mathbb{N}\) tales que \(p,q \geq 0\). Entonces tenemos que se puede escribir:
\begin{equation*}
|u_p-u_q| = |u_p-l+l-u_q| \stackrel{(*)}{\leq} |u_p-l|+|u_q-l|\stackrel{(**)}{=} \frac{\epsilon}{2}+\frac{\epsilon}{2}=\epsilon,
\end{equation*}
donde \( (*) \) se ha hecho uso de la desigualdad triangular y de la propia definición del valor absoluto; mientras que en \((**) \)se ha usado lo visto en (\ref{lim3}) para los casos de \(p\) y \(q\).

De esta forma, se ha demostrado que \(\forall \epsilon \geq 0\) se verifica que \( |u_p-u_q| \leq 0\), es decir, que \(u\) es una sucesión de Cauchy.
\end{demostracion}

En Lean esto se formalice como sigue:
\begin{leancode}
import tuto_lib
variables {u : ℕ → ℝ} {a l : ℝ}
example : (∃ l, seq_limit u l) → cauchy_sequence u :=
begin
intros h eps eps_pos,
cases h with l hl,
rw seq_limit at hl,
cases hl (eps/2) (by linarith) with N hN,
use N,
intros p q hp hq,
calc |u p - u q| = |(u p - l)+(l - u q)| : by ring
... ≤ |u p - l| + |l - u q|: by apply abs_add
... = |u p - l| + |u q - l|: by rw abs_sub l (u q)
... ≤ eps/2 + eps/2 : by linarith [hN p hp, hN q hq]
... = eps : by ring,
end
\end{leancode}

\subsection{Suma de sucesiones convergentes}
\begin{teorema}
	Si \(u\) y \(v\) son sucesiones convergentes, entonces \( (u+v)\) es convergente y se verifica que \(\text{lim} (u+v)= \text{lim} u+ \text{lim} v\).
\end{teorema}
\begin{demostracion}
	Como consecuencia de que las sucesiones \(u\) y \(v\) son convergentes se tiene que:
	\begin{equation}\label{limU}
	\forall \epsilon >0, \exists N_1 \in \mathbb{N} \ \text{tal que }\forall n \geq N_1 |u_n-l| \leq \frac{\epsilon}{2} \hspace{0.5cm}\text{y}
	\end{equation}
	\begin{equation}\label{limV}
	\forall \epsilon >0, \exists N_2 \in \mathbb{N} \ \text{tal que }\forall n \geq N_1 |v_n-l'| \leq \frac{\epsilon}{2},
	\end{equation}
	donde hemos denotado \(l\) al límite de \(u\) y \(l'\) al de \(v\).
	
	A continuación, definimos \(N_0 := \text{max}(N_1,N_2)\) y consideramos \(n \geq N_0\), de manera que:
	\begin{equation*}
	|(u_n+v_n)-(l+l')| \stackrel{(*)}{\leq}|u_n-l|+|v_n-l'|\stackrel{(**)}{\leq} \frac{\epsilon}{2}+\frac{\epsilon}{2}=\epsilon,
	\end{equation*}
	donde en \((*)\) se usado la desigualdad triangular y en \((**)\) las desigualdades vistas en (\ref{limU}) y (\ref{limV}).
	
	De manera que finalmente se ha demostrado que \( \forall \epsilon >0, \ \exists N_0 \in \mathbb{N}\) tal que \( \forall n \geq N_0\) se verifica que \( | (u_n+v_n)-(l+l')| \leq \epsilon\), o lo que es lo mismo que la sucesión \((u+v)\) converge a \( (l+l')\).
\end{demostracion}

En Lean esto se formaliza como:
\begin{leancode}
import tuto_lib
variables {u v: ℕ → ℝ} {l l' : ℝ}
example (hu : seq_limit u l) (hv : seq_limit v l') :
seq_limit (u + v) (l + l') :=
begin
intros eps eps_pos,
cases hu (eps/2) (by linarith) with N1 hN1,
cases hv (eps/2) (by linarith) with N2 hN2,
let N0:= max N1 N2,
use N0,
intros n hn,
rw ge_max_iff at hn,
calc
|(u + v) n - (l + l')| = |u n + v n - (l + l')|   : rfl
... = |(u n - l) + (v n - l')| : by congr' 1 ; ring
... ≤ |u n - l| + |v n - l'|   : by apply abs_add
... ≤  eps                     : by linarith [hN1 n (by linarith),
 hN2 n (by linarith)],
end
\end{leancode}

\subsection{Paridad de la suma de funciones}
Como ya se adelantó, los tres resultados que vamos a estudiar a continuación son relativos a la paridad de las funciones. Es por eso, que previo al desarrollo de los resultados veamos las dos siguientes definiciones:
\begin{definicion}\label{funpar}
	Sea \(f: \mathbb{R}\rightarrow \mathbb{R}\), se dirá que \(f\) es una \underline{función par} si para todo \(x \in \mathbb{R}\) se verifica que \(f(-x)=f(x)\).
\end{definicion}

\begin{definicion}\label{funimpar}
	Sea \(f: \mathbb{R}\rightarrow \mathbb{R}\), se dirá que \(f\) es una \underline{función impar} si para todo \(x \in \mathbb{R}\) se verifica que \(f(-x)=-f(x)\).
\end{definicion}

Estas dos definiciones se formalizan en Lean como sigue:
\begin{leancode}
def even_fun (f : ℝ → ℝ) := ∀ x, f (-x) = f x
def odd_fun (f : ℝ → ℝ) := ∀ x, f (-x) = -f x
\end{leancode}

A continuación, ya estamos preparados para ver la demostración del resultado deseado.
\begin{teorema}
	Sean \(f\) y \(g\) dos funciones pares, entonces se tiene que la suma de ambas, \( (f+g)\), es también una función par.
\end{teorema}
\begin{demostracion}
	Por la definición \ref{funpar} de una función par, se tiene que se verifica lo siguiente:
	\begin{equation}\label{fpar}
	f(-x)=f(x)\hspace{1cm}\forall x \in \mathbb{R} \ \text{y}
	\end{equation}
	\begin{equation}\label{gpar}
	g(-x)=g(x)\hspace{1cm}\forall x \in \mathbb{R}.
	\end{equation}
	Como queremos demostrar que \(f+g\) es una función par, se tendría que probar que \((f+g)(-x)=(f+g) (-x) \ \forall x \in \mathbb{R}\). Veámoslo, consideremos \(x \in \mathbb{R}\) arbitrario, entonces:
	\begin{equation*}
	(f+g)(-x)\stackrel{(1)}{=}f(-x)+g(-x)\stackrel{(2)}{=}f(x)+g(x)\stackrel{(3)}{=}(f+g)(x),
	\end{equation*}
	donde tanto en \((1)\) como en \(3\) hemos usado la propia definición de la suma de funciones, mientras que en \((2) \) se han usado las hipótesis de paridad (\ref{fpar}) y (\ref{gpar}). Por tanto, ya tendríamos el resultado deseado.
\end{demostracion}

Esta demostración se formaliza en Lean de la siguiente manera:
\begin{leancode}
import data.real.basic
import algebra.group.pi
example (f g : ℝ → ℝ) : even_fun f → even_fun g →  even_fun (f + g) :=
begin
intros hf hg,
intros x,
calc (f + g) (-x) = f (-x) + g (-x) : rfl
... = f x + g (-x) : by rw hf 
... = f x + g x : by rw hg 
... = (f + g) x : rfl
end
\end{leancode}

\subsection{Paridad de la composición de funciones}
\begin{teorema}
	Sea \(f\) una función par y sea \(g\) una función arbitraria, entonces se tiene que \((g\circ f)\) es también par.
\end{teorema}
\begin{demostracion}
	Al igual que en la demostración anterior, por la definición \ref{funpar}, se tiene que para la función \(f\) se verifica que
	\begin{equation}\label{fpar2}
	f(-x)=f(x) \hspace{1cm}\forall x \in \mathbb{R}.
	\end{equation}
	Consideremos ahora \(x\in \mathbb{R}\) arbitrario y estudiemos la composición de un función arbitraria \(g\) con \(f\):
	\begin{equation*}
	(g\circ f)(-x)\stackrel{(1)}{=} g(f(-x))\stackrel{(2)}{=}g(f(x)) \stackrel{(3)}{=} (g\circ f)(x),
	\end{equation*}
	donde en \((1)\) y \((3)\) se ha usado la definición de la composición; mientras que en \((2)\) se ha hecho uso de la hipótesis de paridad de \(f\), (\ref{fpar2}).
\end{demostracion}

En Lean esto se formalizaría como sigue:
\begin{leancode}
import data.real.basic
import algebra.group.pi
example (f g : ℝ → ℝ) : even_fun f → even_fun (g∘f) :=
begin
intros hf x,
calc (g∘f) (-x) = g (f (-x)) : rfl
... = g (f x) : by rw hf,
end
\end{leancode}

\subsection{Imparidad de la composición de funciones impares}
\begin{teorema}
	Sean \(f\) y \(g\) dos funciones impares, entonces se tiene que la composición de ambas, \( (g \circ f)\), es también una función impar.
\end{teorema}
\begin{demostracion}
	Por la definición \ref{funimpar} de una función impar vista anteriormente, se tiene que se verifica lo siguiente:
	\begin{equation}\label{fimpar}
	f(-x)=-f(x)\hspace{1cm}\forall x \in \mathbb{R} \ \text{y}
	\end{equation}
	\begin{equation}\label{gimpar}
	g(-x)=-g(x)\hspace{1cm}\forall x \in \mathbb{R}.
	\end{equation}
	A continuación, vamos a estudiar la composición de estas dos funciones para un \(x \in \mathbb{R}\) arbitrario:
	\begin{equation*}
	(g \circ f)(-x)=g(f(-x))\stackrel{(1)}{=}g(-f(x))\stackrel{(2)}{=}-g(f(x))=-(g\circ f)(x),
	\end{equation*}
	donde en \((1)\) y en \((2)\) se han usado las hipótesis de funciones impares, (\ref{fimpar}) y (\ref{gimpar}) respectivamente.
	
	De esta forma ya tendríamos la demostración deseada.
\end{demostracion}

Prosigamos con la formalización en Lean de esta demostración:
\begin{leancode}
import data.real.basic
import algebra.group.pi
example (f g : ℝ → ℝ) : odd_fun f → odd_fun g →  odd_fun (g ∘ f) :=
begin
intros hf hg x,
calc (g ∘ f) (-x) = g (f (-x)) : rfl
... = g (-f x): by rw hf
... =- g (f x):by rw hg
...= - (g ∘ f) x: rfl,
end
\end{leancode}
\newpage 
\section{Álgebra Básica}
En esta segunda sección se estudiarán diversos resultados correspondientes a la asignatura de formación básica: Álgebra Básica. Al igual que Cálculo Infinitesimal, también se corresponde al primer año del grado en Matemáticas.

Se van a estudiar dos resultados para los cuales es necesario introducir previamente el siguiente concepto:
\begin{definicion}\label{division}
	Denotaremos la división en \(\mathbb{Z}\) por el símbolo "\( | \)". Entonces se dirá que \(a\) divide a \(b\), \( (a|b)\), si y solamente si existe \(k \in \mathbb{Z}\) tal que \(b=a\cdot k\).
\end{definicion}

A continuación, una vez hemos introducido el concepto, ya estamos listos para ver los resultados en cuestión.

\subsection{Transitividad de la división}
\begin{teorema}
	Sean \(a, b, c \in \mathbb{Z}\) tales que verifican que \(a|b\) y \(b|c\), entonces se tiene que \(a|c\).
\end{teorema}
\begin{demostracion}
	Por la definición \ref{division} de división en \(\mathbb{Z}\) vista anteriormente, como consecuencia de que \(a|b\) se tiene que:
	\begin{equation}\label{div1}
	\exists k_1 \in \mathbb{Z} \ \text{tal que }b=k_1 \cdot a.
	\end{equation}
	Asimismo, como consecuencia de que \(b|c\) se verifica que:
	\begin{equation}\label{div2}
	\exists k_2 \in \mathbb{Z} \ \text{tal que }c=k_2 \cdot b.
	\end{equation}
	De manera que si introducimos la descomposición de \(b\) descrita en (\ref{div1}) en la de \(c\) vista en (\ref{div2}), se tiene que:
	\begin{equation}
	\exists k_1, k_2 \in \mathbb{Z} \ \text{tal que }c=\underbrace{k_1 \cdot k_2}_{k_3} \cdot a \implies \exists k_3\in \mathbb{Z} \ \text{tal que }c=k_3 \cdot a .
	\end{equation}
	Y ya tendríamos el resultado deseado.
\end{demostracion}

Veamos ahora la formalización en Lean:
\begin{leancode}
import data.real.basic
import data.int.parity
variables (a b c : ℤ)
example (h₁ : a ∣ b) (h₂ : b ∣ c) : a ∣ c :=
begin
cases h₁ with k1 hk1,
cases h₂ with k2 hk2,
rw hk1 at hk2,
use k1*k2,
rw ← mul_assoc,
exact hk2,
end
\end{leancode}

\subsection{Aditividad de la división}
\begin{teorema}
	Sean \(a, b, c \in \mathbb{Z}\) tales que verifican que \(a|b\) y \(a|c\), entonces se tiene que \(a|(b+c)\).
\end{teorema}
\begin{demostracion}
	Análogamente a la demostración anterior, por la definición \ref{division} se tiene que:
	\begin{equation}\label{div3}
	a|b \hspace{0.5cm} \implies \hspace{0.5cm}\exists k_1 \in \mathbb{Z} \ \text{tal que }b=k_1 \cdot a \hspace{0.5cm} \text{y}
	\end{equation}
	\begin{equation}\label{div4}
	a|c \hspace{0.5cm} \implies \hspace{0.5cm}\exists k_2 \in \mathbb{Z} \ \text{tal que }c=k_2 \cdot a \hspace{0.5cm}.
	\end{equation}
	De manera que sumando las dos expresiones descritas en (\ref{div3}) y (\ref{div4}) llegamos a que
	\begin{equation*}
	\exists k_1, k_2 \in \mathbb{Z} \ \text{tal que }b+c=(k_1+k_2)\cdot a \implies a | (b+c`)
	\end{equation*}
\end{demostracion}

La formalización de esta demostración en Lean sería:
\begin{leancode}
import data.real.basic
import data.int.parity
variables (a b c : ℤ)
example (h1 : a ∣ b) (h2 : a ∣ c) : a ∣ b+c :=
begin
cases h1 with k1 hk1,
rw hk1,
cases h2 with k2 hk2,
rw hk2,
use k1+k2,
ring,
end
\end{leancode}