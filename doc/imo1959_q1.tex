\section{IMO 1959 Q1}

En esta sección se va a detallar la solución al problema Q1
correspondiente al año 1959. En primer lugar, se presentarán el
enunciado y su solución en lenguaje natural, de manera que ésta última
se asemeje a la solución que posteriormente se formalizará en
Lean. Luego, se analizará de manera detallada dicha formalización del
problema en Lean.

\noindent
\framebox{\parbox{\textwidth}{
  \textbf{Problema 1 (1959--Q1)}. Demostrar que la fracción
  \[\frac{21⋅n+4}{14⋅n+3}\]
  es irreducible para cualquier número natural \(n\).}}

\textbf{Solución:}

Para ver que esta fracción es irreducible para cualquier número natural
\(n\) se demostrará que el numerador y el denominador de ésta son dos
números coprimos independientemente del valor de \(n\).

Antes de proseguir, introduzcamos el concepto de que dos números sean
coprimos:

\begin{definicion}\label{coprimos}
  Se dirá que dos números enteros \(a\) y \(b\) son \textbf{coprimos}
  o \textbf{primos relativos} si no tienen ningún factor primo en común,
  o equivalentemente, si no poseen otro divisor común distinto de \(1\)
  y \(-1\).
\end{definicion}

\comentario{Añadir su definición en Lean.}

A continuación, se va a introducir un lema auxiliar, el cual nos dirá
que si existe un número natural \(k\) que divide al numerador y al
denominador de la fracción anterior; entonces, dicho número \(k\)
dividide a \(1\).

\begin{lema}\label{lema}
  Sean \(k\) y \(n\) dos números naturales tales que \(k\) divide al
  numerador y al denominador de la fracción anterior; es decir,
  \begin{align}
    & k | (21⋅n+4)   \label{k1} \\
    & k | (14⋅n+3).  \label{k2}
  \end{align}
  Entonces se tiene que \(k\) divide a \(1\), para cualquier número
  natural \(n\).
\end{lema}

\begin{demostracion}
  Como consecuencia de que \(k\) divida al numerador de la fracción,
  (\ref{k1}), se tiene que también divide a un múltiplo cualquiera de
  éste; es decir,
  \begin{equation}\label{p1}
    k | 2⋅(21⋅n+4),
  \end{equation}
  donde en particular hemos considerado el múltiplo resultante de
  multiplicar por \(2\).

  Análogamente para el caso del denominador, como se verifica
  (\ref{k2}), se puede escribir que
  \begin{equation}\label{p2}
    k | 3⋅(14⋅n+3),
  \end{equation}
  donde en este caso se ha considerado el múltiplo resultante de
  multiplicar por \(3\).

  Por otro lado, se tiene que desarrollando los productos de (\ref{p1}) y
  (\ref{p2}) como sigue:
  \begin{align}
    & 2⋅(21⋅n+4)=42⋅n+8    \\
    & 3·(14·n+3)=42·n+9,
  \end{align}
  se puede observar que para todo número natural \(n\), se verifica la
  siguiente relación entre el múltiplo del numerador y el del denominador:
  \begin{equation}\label{k}
    3⋅(14⋅n+3)=2⋅(21⋅n+4)+1.
  \end{equation}

  Ahora bien, como consecuencia de que \(k\) sea divisor de
  \(3⋅(14⋅n+4)\), (\ref{p2}), y la igualdad vista en(\ref{k}), se llega
  a que:
  \begin{equation}\label{div1b}
    k | 2⋅(21⋅n+4)+1
  \end{equation}

  Para concluir la demostración de este lema, basta con usar la siguiente
  propiedad de la división en los números naturales:

  \begin{proposicion}\label{divn}
    Sean \(k,m\) y \(n\) tres números naturales. Entonces se tiene que
    \(k\) divide a la suma de \(m\) y \(n\) si y solamente si \(k\)
    divide a \(n\). Simbólicamente:
    \begin{equation}
      ∀k,m,n ∈ ℕ, k | (m+n) ⟺ k|n.
    \end{equation}
  \end{proposicion}

  \comentario{La propiedad anterior es falsa. Por ejemplo, 2 divide a 1+3
  y no divide a 3.}

  \begin{nota}
    Como consecuncia de que la suma es conmutativa, en la proposición
    anterior se puede afirmar que \(k\) también ha de dividir a \(m\).
  \end{nota}

  De esta forma, aplicando la propiedad \ref{divn} sobre (\ref{div1b}),
  se llega a que \(k\) divide a \(1\); que era lo que queríamos
  demostrar.
\end{demostracion}

Destacar que cuando hacemos la factorización de un número, se hace en
función de sus factores primos. Por tanto, el divisor \(k\) considerado
es un número primo.

Finalmente, aplicando el Lema \ref{lema} se puede concluir que para
cualquier número natural \(n\) se verifica que los números \(21⋅n+4\)
y \(14⋅n+3\) son coprimos. O equivalentemente, que la fracción
\begin{equation*}
  \frac{21⋅n+4}{14⋅n+3}
\end{equation*}
es irreducible. Esta afirmación se tiene como consecuencia de la
definición de dos números coprimos, vista en \ref{coprimos}.

A continuación, vamos a analizar la formalización en Lean propuesta:

\begin{leancode}
open nat

lemma calculation (n k : ℕ) (h1 : k ∣ 21 * n + 4) (h2 : k ∣ 14 * n + 3) :
k ∣ 1 :=
have h3 : k ∣ 2 * (21 * n + 4), from dvd_mul_of_dvd_right h1 2,
have h4 : k ∣ 3 * (14 * n + 3), from dvd_mul_of_dvd_right h2 3,
have h5 : 3 * (14 * n + 3) = 2 * (21 * n + 4) + 1, by ring,
(nat.dvd_add_right h3).mp (h5 ▸ h4)

theorem imo1959_q1 : ∀ n : ℕ, coprime (21 * n + 4) (14 * n + 3) :=
assume n, coprime_of_dvd' $ λ k hp h1 h2, calculation n k h1 h2
\end{leancode}

En primer lugar, se puede observar que el lema que en la
formalización se nombra \textit{calculation}, se corresponde con
el lema \ref{lema} de la solución en lenguaje natural.

Finalmente, el teorema que se presenta es el que acaba concluyendo
el resultado deseado. Se comprueba que el numerador y el denominador
de la fracción son coprimos para cualquier número natural \(n\).
Esto lo hace aplicando el lema anterior.

\comentario{Añadir una cita en la bibliografía a la prueba original.}

\comentario{Poner la demostración en Lean del lema a continuación de su
  demostración en lenguaje natural.}

\comentario{Adaptar las pruebas para que la correspondencias entre la
  del lenguaje natural y la de Lean sea máxima. Por ejemplo, usando los
  mismos nombres para los lemas y las etiquetas.}

\comentario{Escribir los lemas usados en la prueba, como se indica a
  continuación.}

En la prueba se han usado los siguientes lemas:
\begin{itemize}
\item \mint{lean}|dvd_mul_of_dvd_right: a ∣ b → ∀ c, a ∣ c * b|
  \index{\url{dvd_mul_of_dvd_right}}
\end{itemize}




%%% Local Variables:
%%% mode: latex
%%% TeX-master: "IMO_en_Lean"
%%% End:
